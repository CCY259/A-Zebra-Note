\section{Introduction to Permutation Group Theory}
\subsection{Notations}
A \textbf{permutation group} of a set $\Omega$ is a subgroup of $\Sym\Omega$. If $G$ is a permutation group on $\Omega$, then $G$ acts on $\Omega$ via the canonical injection and this is a faithful action.  Conversely, if $G$ is a faithful action on $\Omega$, then $G$ can be identified as a permutation group of $\Omega$. For simplicity, we reintroduce notations for notions in group actions. Let $G$ act on $\Omega$.  
\begin{align*}
	\omega^G &= O_G(\omega) =\{\omega^g\mid g\in G\}, \tag{\textbf{Orbit} of $\omega \in \Omega$}
	\\
	G_{\omega} &= \operatorname{Stab}_G(\omega) = \{g\in G\mid \omega^g = \omega\}, \tag{\textbf{Point stabilizer} of $\omega\in \Omega$}
	\\
	G_{X} &=  \{g\in G\mid X^g = X\}, \tag{\textbf{Setwise stabilizer} of $X\subseteq \Omega$}
	\\
	G_{(X)} &=  \{g\in G\mid x^g = x\text{ for all }x\in X\}. \tag{\textbf{Elementwise stabilizer} of $X\subseteq \Omega$}
\end{align*}
\subsection{Isomorphic actions}
\begin{definition}
	Let $G$ and $H$ be groups acting on the sets $\Omega$ and $\Delta$, respectively. The two actions (or the pairs $(G,\Omega)$ and $(H,\Delta)$) are said to be \textbf{permutationally isomorphic} if there exist a bijection $\vartheta: \Omega \to \Delta$ and an isomorphism $\chi: G \to H$ such that
	\[ \vartheta(\omega^g) = \vartheta(\omega)^{\chi(g)} \quad \text{for all } \omega \in \Omega, g \in G. \] In other words, for every $g\in G$ the following diagram commutes.
	% https://q.uiver.app/#q=WzAsNCxbMCwwLCJcXE9tZWdhIl0sWzIsMCwiXFxPbWVnYSJdLFswLDIsIlxcRGVsdGEiXSxbMiwyLCJcXERlbHRhIl0sWzAsMSwiZyJdLFsxLDMsIlxcdmFydGhldGEiXSxbMCwyLCJcXHZhcnRoZXRhIiwyXSxbMiwzLCJcXGNoaShnKSIsMl1d
	\[\begin{tikzcd}
		\Omega && \Omega \\
		\\
		\Delta && \Delta
		\arrow["g", from=1-1, to=1-3]
		\arrow["\vartheta"', from=1-1, to=3-1]
		\arrow["\vartheta", from=1-3, to=3-3]
		\arrow["{\chi(g)}"', from=3-1, to=3-3]
	\end{tikzcd}\]
	If such conditions hold, the pair $(\vartheta, \chi)$ is said to be a \textbf{permutational isomorphism}. Similarly, the pair $(\vartheta, \chi)$ is a \textbf{permutational embedding} of the permutation group $G$ on $\Omega$ into the permutation group $H$ on $\Delta$, if $\chi: G \to H$ is a monomorphism and $(\vartheta, \hat{\chi})$ is a permutational isomorphism, where $\hat{\chi}: G \to \operatorname{im}\chi$ is obtained from $\chi$ by simply restricting the range of $\chi$.
\end{definition}


\begin{proposition}
	Let $G$ act on a set $\Omega$. Let $\Delta$ be a set and let $\vartheta: \Omega \to \Delta$ be a bijection. Define a $G$-action on $\Delta$ by $\delta^g = \vartheta((\vartheta^{-1}(\delta))^g)$. Then $(\vartheta, \operatorname{Id}_G)$ is a permutational isomorphism from the $G$-action on $\Omega$ to the $G$-action on $\Delta$.
\end{proposition}

\begin{sketch}
	It is easy to see that the $G$-action on $\Delta$ is well defined and is permutationally isomorphic to the $G$-action on $\Omega$. 
\end{sketch}

\begin{comment}
	\begin{proposition}
		Let $G_1$ and $G_2$ be groups acting transitively on $\Omega_1$ and $\Omega_2$, respectively. Then the following are equivalent.
		\begin{enumerate}[(1)]
			\item The actions of $G_1$ and $G_2$ on $\Omega_1$ and $\Omega_2$, respectively, are permutationally isomorphic.
			\item There exist $\omega_1 \in \Omega_1$ and $\omega_2 \in \Omega_2$ and an isomorphism $\alpha: G_1 \to G_2$ such that $\alpha(G_1)_{\omega_1} = (G_2)_{\omega_2}$.
			\item For all $\omega_1 \in \Omega_1$ and $\omega_2 \in \Omega_2$, there exists an isomorphism $\alpha: G_1 \to G_2$ such that $\alpha(G_1)_{\omega_1} = (G_2)_{\omega_2}$.
		\end{enumerate}
	\end{proposition}
	
	\begin{sketch}
		(1) $\Rightarrow$ (2) Let $(\gamma: \Omega_1 \to \Omega_2, \alpha: G_1 \to G_2)$ be a permutational isomorphism. Let $\omega_1 \in \Omega_1$ and $g \in (G_1)_{\omega_1}$. Then $\gamma(\omega_1) ^{\alpha (g)} = \gamma(\omega_1^g) = \gamma(\omega_1)$, and so $\alpha(g) \in (G_2)_{\gamma(\omega_1)}$. Therefore $\alpha(G_1)_{\omega_1} \le (G_2)_{\omega_1\gamma}$. 
		
		On the other hand if $g_2 \in (G_2)_{\omega_1\gamma}$ then there is some $g_1 \in G_1$ such that $g_1\alpha = g_2$. Then $\omega_1\gamma = (\omega_1\gamma)g_2 = (\omega_1\gamma)(g_1\alpha) = (\omega_1 g_1)\gamma$. Applying $\gamma^{-1}$ we obtain $\omega_1 = \omega_1 g_1$. Therefore $g_1 \in (G_1)_{\omega_1}$, and hence $g_2 = g_1\alpha \in (G_1)_{\omega_1}\alpha$. This shows that $(G_1)_{\omega_1}\alpha = (G_2)_{\omega_1\gamma}$, and assertion (ii) is valid.
		
		Next we show that (ii) implies (iii). Since both $G_1$ and $G_2$ are transitive, $\{(G_1)_{\omega'_1} \mid \omega'_1 \in \Omega_1\}$ and $\{(G_2)_{\omega'_2} \mid \omega'_2 \in \Omega_2\}$ are conjugacy classes in $G_1$ and $G_2$, respectively, by Lemma 2.1. Let $\alpha: G_1 \to G_2$ be an isomorphism, and $\omega_1 \in \Omega_1, \omega_2 \in \Omega_2$ such that $(G_1)_{\omega_1}\alpha = (G_2)_{\omega_2}$, and let $\omega'_1 \in \Omega_1$ and $\omega'_2 \in \Omega_2$. Then there are $\sigma_1 \in \Inn(G_1)$ and $\sigma_2 \in \Inn(G_2)$ such that $(G_1)_{\omega'_1} = (G_1)_{\omega_1}\sigma_1$ and $(G_2)_{\omega_2}\sigma_2 = (G_2)_{\omega'_2}$. If $\phi = \sigma_1\alpha\sigma_2$ then clearly $(G_1)_{\omega'_1}\phi = (G_2)_{\omega'_2}$.
		
		Finally we show that (iii) implies (i). Let $\omega_1 \in \Omega_1, \omega_2 \in \Omega_2$, and $\alpha: G_1 \to G_2$ be an isomorphism such that $(G_1)_{\omega_1}\alpha = (G_2)_{\omega_2}$. Then define $\gamma: \Omega_1 \to \Omega_2$ by $(\omega_1 g)\gamma = \omega_2(g\alpha)$ for $g \in G_1$. Let us show that $\gamma$ is well defined. Since $G_1$ is transitive, $\omega_1 g$ runs through all elements of $\Omega_1$ as $g$ runs through the elements of $G_1$. If $\omega_1 g_1 = \omega_1 g_2$ for some $g_1, g_2 \in G_1$, then, $g_1g_2^{-1} \in (G_1)_{\omega_1}$, and so $(g_1g_2^{-1})\alpha = (g_1\alpha)(g_2\alpha)^{-1} \in (G_2)_{\omega_2}$. Hence $\omega_2(g_1\alpha) = \omega_2(g_2\alpha)$, and $\gamma$ is well defined. Since $G_2 = G_1\alpha$ is transitive, $\gamma$ is onto. If $(\omega_1 g_1)\gamma = (\omega_1 g_2)\gamma$ for some $g_1, g_2 \in G_1$, then, by the definition of $\gamma$, we have $\omega_2(g_1\alpha) = \omega_2(g_2\alpha)$ and so $(g_1\alpha)(g_2\alpha)^{-1} = (g_1g_2^{-1})\alpha \in (G_2)_{\omega_2}$. Thus $g_1g_2^{-1} \in (G_1)_{\omega_1}$, and so $\omega_1 g_1 = \omega_1 g_2$. Therefore $\gamma$ is one-to-one, and hence $\gamma$ is a bijection. If $\omega \in \Omega_1$ and $g \in G_1$ then there is a $g_1 \in G_1$ such that $\omega = \omega_1 g_1$. Then
		\[ (\omega g)\gamma = ((\omega_1 g_1)g)\gamma = \omega_2((g_1g)\alpha) = \omega_2(g_1\alpha)(g\alpha) = (\omega_1 g_1)\gamma (g\alpha) = (\omega\gamma)(g\alpha). \]
		Thus $(\gamma, \alpha)$ is a permutational isomorphism.
	\end{sketch}

\begin{proposition}
	Let $G $ and $H$ be groups acting transitively on $\Omega $ and $\Delta $, respectively. Then the following are equivalent.
	\begin{enumerate}[(1)]
		\item The actions of $G $ and $H$ on $\Omega $ and $\Delta $, respectively, are permutationally isomorphic.
		\item There exist $\omega_1 \in \Omega $ and $\omega_2 \in \Delta $ and an isomorphism $\varphi: G  \to H$ such that $\varphi(G_{\omega_1}) = H_{\omega_2}$.
		\item For all $\omega_1 \in \Omega $ and $\omega_2 \in \Delta $, there exists an isomorphism $\varphi: G  \to H$ such that $\varphi(G_{\omega_1}) = H_{\omega_2}$.
	\end{enumerate}
\end{proposition}

\begin{sketch}
	(1) $\Rightarrow$ (2) Let $(\vartheta: \Omega  \to \Delta , \varphi: G  \to H)$ be a permutational isomorphism. Let $\omega_1 \in \Omega $ and $g \in G_{\omega_1}$. Then $\vartheta(\omega_1) ^{\varphi (g)} = \vartheta(\omega_1^g) = \vartheta(\omega_1)$, and so $\varphi(g) \in H_{\vartheta(\omega_1)}$. Therefore $\varphi (G_{\omega_1}) \leq H_{\vartheta(\omega_1)}$. 
	
	On the other hand if $g_2 \in H_{\vartheta(\omega_1)}$ then there is some $g_1 \in G $ such that $\varphi(g_1) = g_2$. Then $\vartheta(\omega_1) = \vartheta(\omega_1)^{g_2} = \vartheta(\omega_1)^{\varphi(g_1)} = \vartheta(\omega_1^{g_1})$. Applying $\vartheta^{-1}$ we obtain $\omega_1 = \omega_1^{g_1}$. Therefore $g_1 \in G_{\omega_1}$, and hence $g_2 = \varphi(g_1) \in \varphi(G_{\omega_1})$. This shows that $\varphi(G_{\omega_1} ) = H_{\vartheta(\omega_1)}$.
	
	(2) $\Rightarrow$ (3) Since both $G$ and $H$ are transitive, $\{G_{\omega'_1} \mid \omega'_1 \in \Omega \}$ and $\{H_{\omega'_2} \mid \omega'_2 \in \Delta \}$ are conjugacy classes in $G$ and $H$, respectively, by Proposition \ref{prop-stabilizer}.(ii). Let $\varphi: G  \to H$ be an isomorphism, and $\omega_1 \in \Omega , \omega_2 \in \Delta$ such that $\varphi(G_{\omega_1}) = H_{\omega_2}$, and let $\omega'_1 \in \Omega $ and $\omega'_2 \in \Delta $. Then there are $\sigma_1 \in \Inn G$ and $\sigma_2 \in \Inn H$ such that $\sigma_1(G_{\omega'_1}) = G_{\omega_1}$ and $\sigma_2(H_{\omega_2}) = H_{\omega'_2}$. If $\phi = \sigma_1\varphi\sigma_2$ then clearly $\phi(G_{\omega'_1} ) = H_{\omega'_2}$.
	
	(3) $\Rightarrow$ (2) Let $\omega_1 \in \Omega , \omega_2 \in \Delta $, and $\varphi: G  \to H$ be an isomorphism such that $\varphi(G_{\omega_1} ) = H_{\omega_2}$. Then define $\vartheta: \Omega  \to \Delta $ by $\vartheta (\omega_1^g)= \omega_2^{\varphi(g)}$ for $g \in G $. We show that $\vartheta$ is well defined. Since $G $ is transitive, we have $\Omega = \{\omega_1^g\mid g\in G\}$. If $\omega_1^{g_1} = \omega_1^{g_2}$ for some $g_1, g_2 \in G $, then, $g_1g_2^{-1} \in G_{\omega_1}$, and so $\varphi(g_1g_2^{-1}) = \varphi(g_1)\varphi(g_2)^{-1} \in H_{\omega_2}$. Hence $\omega_2^{\varphi(g_1)} = \omega_2^{\varphi(g_2)}$, and so $\vartheta$ is well defined.  Since $H = \varphi(G)$ is transitive, $\vartheta$ is surjective. For injectivity, suppose that $\vartheta(\omega_1^{g_1} ) = \vartheta(\omega_1^{g_2} )$ for some $g_1, g_2 \in G$, then, by the definition of $\vartheta$, we have $\omega_2^{\varphi(g_1)} = \omega_2^{\varphi(g_2)}$ and so $\varphi(g_1)\varphi(g_2)^{-1} = \varphi(g_1g_2^{-1}) \in H_{\omega_2}$. Thus $g_1g_2^{-1} \in G_{\omega_1}$, and so $\omega_1^{g_1} = \omega_1^{g_2}$. Therefore $\vartheta$ is injective, and hence $\vartheta$ is a bijection. 
	
	If $\omega \in \Omega $ and $g \in G $ then there is a $g_1 \in G $ such that $\omega = \omega_1^{g_1}$. Then
	\begin{equation*}
		\vartheta(\omega^g) = \vartheta((\omega_1^{g_1} )^g) = \omega_2^{\varphi(g_1g)} = \omega_2^{\varphi(g_1)\varphi(g)} = \vartheta(\omega_1^{g_1} )^{\varphi(g)} = \vartheta(\omega)^{\varphi(g)}.
	\end{equation*}
	Thus $(\vartheta, \varphi)$ is a permutational isomorphism.
\end{sketch}
\end{comment}
\begin{proposition}
	Let $G $ and $H$ be groups acting transitively on $\Omega $ and $\Delta $, respectively. Then the following are equivalent.
	\begin{enumerate}[(1)]
		\item The actions of $G $ and $H$ on $\Omega $ and $\Delta $, respectively, are permutationally isomorphic.
		\item There exist $\omega \in \Omega $ and $\delta \in \Delta $ and an isomorphism $\varphi: G  \to H$ such that $\varphi(G_{\omega}) = H_{\delta}$.
		\item For all $\omega \in \Omega $ and $\delta \in \Delta $, there exists an isomorphism $\varphi: G  \to H$ such that $\varphi(G_{\omega}) = H_{\delta}$.
	\end{enumerate}
\end{proposition}

\begin{sketch}
	(1) $\Rightarrow$ (2) Let $(\vartheta: \Omega  \to \Delta , \varphi: G  \to H)$ be a permutational isomorphism. Let $\omega \in \Omega $ and $g \in G_{\omega}$. Then $\vartheta(\omega) ^{\varphi (g)} = \vartheta(\omega^g) = \vartheta(\omega)$, and so $\varphi(g) \in H_{\vartheta(\omega)}$. Therefore $\varphi (G_{\omega}) \leq H_{\vartheta(\omega)}$. 
	
	On the other hand if $g_2 \in H_{\vartheta(\omega)}$ then there is some $g_1 \in G $ such that $\varphi(g_1) = g_2$. Then $\vartheta(\omega) = \vartheta(\omega)^{g_2} = \vartheta(\omega)^{\varphi(g_1)} = \vartheta(\omega^{g_1})$. Applying $\vartheta^{-1}$ we obtain $\omega = \omega^{g_1}$. Therefore $g_1 \in G_{\omega}$, and hence $g_2 = \varphi(g_1) \in \varphi(G_{\omega})$. This shows that $\varphi(G_{\omega} ) = H_{\vartheta(\omega)}$.
	
	(2) $\Rightarrow$ (3) Since both $G$ and $H$ are transitive, $\{G_{\omega'} \mid \omega' \in \Omega \}$ and $\{H_{\delta'} \mid \delta' \in \Delta \}$ are conjugacy classes in $G$ and $H$, respectively, by Proposition \ref{prop-stabilizer}.(ii). Let $\varphi: G  \to H$ be an isomorphism, and $\omega \in \Omega , \delta \in \Delta$ such that $\varphi(G_{\omega}) = H_{\delta}$, and let $\omega' \in \Omega $ and $\delta' \in \Delta $. Then there are $\sigma_1 \in \Inn G$ and $\sigma_2 \in \Inn H$ such that $\sigma_1(G_{\omega'}) = G_{\omega}$ and $\sigma_2(H_{\delta}) = H_{\delta'}$. If $\phi = \sigma_1\varphi\sigma_2$ then clearly $\phi(G_{\omega'} ) = H_{\delta'}$.
	
	(3) $\Rightarrow$ (2) Let $\omega \in \Omega , \delta \in \Delta $, and $\varphi: G  \to H$ be an isomorphism such that $\varphi(G_{\omega} ) = H_{\delta}$. Then define $\vartheta: \Omega  \to \Delta $ by $\vartheta (\omega^g)= \delta^{\varphi(g)}$ for $g \in G $. We show that $\vartheta$ is well defined. Since $G $ is transitive, we have $\Omega = \{\omega^g\mid g\in G\}$. If $\omega^{g_1} = \omega^{g_2}$ for some $g_1, g_2 \in G $, then, $g_1g_2^{-1} \in G_{\omega}$, and so $\varphi(g_1g_2^{-1}) = \varphi(g_1)\varphi(g_2)^{-1} \in H_{\delta}$. Hence $\delta^{\varphi(g_1)} = \delta^{\varphi(g_2)}$, and so $\vartheta$ is well defined.  Since $H = \varphi(G)$ is transitive, $\vartheta$ is surjective. For injectivity, suppose that $\vartheta(\omega^{g_1} ) = \vartheta(\omega^{g_2} )$ for some $g_1, g_2 \in G$, then, by the definition of $\vartheta$, we have $\delta^{\varphi(g_1)} = \delta^{\varphi(g_2)}$ and so $\varphi(g_1)\varphi(g_2)^{-1} = \varphi(g_1g_2^{-1}) \in H_{\delta}$. Thus $g_1g_2^{-1} \in G_{\omega}$, and so $\omega^{g_1} = \omega^{g_2}$. Therefore $\vartheta$ is injective, and hence $\vartheta$ is a bijection. 
	
	If $\omega \in \Omega $ and $g \in G $ then there is a $g_1 \in G $ such that $\omega = \omega^{g_1}$. Then
	\begin{equation*}
		\vartheta(\omega^g) = \vartheta((\omega^{g_1} )^g) = \delta^{\varphi(g_1g)} = \delta^{\varphi(g_1)\varphi(g)} = \vartheta(\omega^{g_1} )^{\varphi(g)} = \vartheta(\omega)^{\varphi(g)}.
	\end{equation*}
	Thus $(\vartheta, \varphi)$ is a permutational isomorphism.
\end{sketch}

\begin{comment}
	\begin{proposition}
		Let $\Omega$ be a set and let $G_1, G_2 \le \Sym\Omega$. Then $G_1$ and $G_2$ are permutationally isomorphic if and only if $G_1$ and $G_2$ are conjugate subgroups of $\Sym\Omega$. Moreover, if $(\gamma, \alpha)$ is a permutational isomorphism, then $\gamma \in \Sym\Omega$ and $g\alpha = \gamma^{-1}g\gamma$, for all $g \in G_1$.
	\end{proposition}
	
	\begin{sketch}
		Assume first that $G_1$ and $G_2$ are permutationally isomorphic, and let $(\gamma, \alpha)$ be a permutational isomorphism. Then, for all $g \in G_1$ and $\omega \in \Omega$, we have $(\omega g)\gamma = (\omega\gamma)(g\alpha)$; that is $\omega g = (\omega\gamma)(g\alpha)\gamma^{-1}$. This shows that $\gamma^{-1}g\gamma = g\alpha$, and so $G_1$ is conjugate to $G_1\alpha = G_2$.
		
		Suppose conversely that $\gamma \in \Sym\Omega$ such that $\gamma^{-1}G_1\gamma = G_2$ and let $\alpha: G_1 \to G_2$ denote the isomorphism induced by conjugating by $\gamma$; that is, for $g \in G_1$, we define $g\alpha = \gamma^{-1}g\gamma$. Then, for $g \in G$ and $\omega \in \Omega$, $(\omega\gamma)(g\alpha) = \omega\gamma\gamma^{-1}g\gamma = \omega g\gamma$. Hence $(\gamma, \alpha)$ is a permutational isomorphism.
	\end{sketch}
\end{comment}


\begin{proposition}
	Let $\Omega$ be a set and let $G_1, G_2 \le \Sym\Omega$. Then $G_1$ and $G_2$ are permutationally isomorphic if and only if they are conjugate in $\Sym\Omega$. Moreover, if $(\vartheta, \varphi)$ is a permutational isomorphism, then $\vartheta \in \Sym\Omega$ and $\varphi(g) = \vartheta^{-1}g\vartheta$, for all $g \in G_1$.
\end{proposition}

\begin{sketch}
	Assume that $G_1$ and $G_2$ are permutationally isomorphic, and let $(\vartheta:\Omega\to \Omega, \varphi:G_1\rightarrow G_2)$ be a permutational isomorphism. Then we have $\vartheta(\omega^g) = \vartheta(\omega)^{\varphi(g)}$ for all $g \in G_1$ and $\omega \in \Omega$, i.e., $\omega^g = \vartheta^{-1}(\vartheta(\omega)^{\varphi(g)})$. This shows that $\vartheta^{-1}g\vartheta = \varphi(g)$, and so $G_1$ is conjugate to $\varphi(G_1) = G_2$.
	
	Suppose conversely that $\vartheta \in \Sym\Omega$ such that $\vartheta^{-1}G_1\vartheta = G_2$ and let $\varphi: G_1 \to G_2$ be the isomorphism defined by  $\varphi(g) = \vartheta^{-1}g\vartheta$. Then  $\vartheta(\omega)^{\varphi(g)} = \vartheta(\omega)^{\vartheta^{-1}g\vartheta} = \vartheta((\vartheta^{-1}(\vartheta(\omega)))^{g}) = \vartheta(\omega^g)$ for $g \in G$ and $\omega \in \Omega$. Hence $(\vartheta, \varphi)$ is a permutational isomorphism.
\end{sketch}

 Recall that if $H$ is a subgroup of  a group $G$, then the right coset action of $G$ on the set $\Gamma_H$ of right cosets of $H$  is defined by $(Hh)^g  = Hhg$ for $h,g\in G$. In view of Theorem \ref{thm-statement-of-kernel-of-left-translation}, this action is transitive. In fact, every transitive action is permutationally isomorphic to a coset action. 
\begin{proposition} \label{prop-transitive-action-perm-isom}
	Let $G$ act transitively on $\Omega$ and let $\omega\in \Omega$. Then the $G$-action on $\Omega$ is permutationally isomorphic to the $G$-action on $\Gamma_{G_\omega}$.
\end{proposition}
\begin{sketch}
	Since $G$ is transitive, we have $\omega^G = \Omega$. Let $\vartheta:\Omega \rightarrow \Gamma_{G_\omega}$ be the bijective function as defined in Lemma \ref{lemma-orbit-partition-and-bijection}.(ii). Then $\vartheta((\omega^g)^h) = \vartheta(\omega^{gh}) = G_\omega gh = (G_\omega g)^h = \vartheta(\omega^g)^h$. This shows that $(\vartheta,\operatorname{Id}_G)$ is a permutational isomorphism.
\end{sketch}

\begin{corollary}
	Let $G$ act transitively on $\Omega$ and let $\omega\in\Omega$. Then a subgroup $H$ of $G$ is transitive if and only if $G = G_\omega H$.
\end{corollary}
\begin{sketch}
	($\Rightarrow$) Since $H$ is transitive, for every $g\in G$, there exists $h\in H$ such that $\omega^g = \omega^h$. By Proposition \ref{prop-transitive-action-perm-isom}, we get $G_\omega g = G_\omega h$. Hence we can write $gh^{-1} = g'\in G$ and thus $g = g'h\in G_\omega H$.
	
	($\Leftarrow$) Reverse the argument in the previous paragraph.
\end{sketch}

\subsection{Blocks}
\begin{definition}
	Let $G$ act transitively on $\Omega$. The nonempty subset $\Delta$ of $\Omega$ is called a \textbf{block} if for every $g\in G$, either $\Delta^g=\Delta$ or $\Delta^g \cap \Delta = \emptyset$. All the singletons of $\Omega$ and the set $\Omega$ itself are blocks, and so they are said to be \textbf{trivial}.
\end{definition}
\begin{comment}
	\begin{proposition} \label{prop-block-and-stab}
		Let $G$ acts transitively on $X$. Let $x\in X$ be fixed. Then there is a one-to-one correspondence between the set of blocks of $X$ containing $x$ and the set of subgroups which contains the stabilizer $G_x$ of $x$. 
	\end{proposition}
	\begin{sketch}
		Let $B$ be a block containing $x \in X$, and consider the set $H_B = \{g \in G \mid gx \in B\}$. We claim that $H_B$ is a subgroup of  $G$. Clearly, $e \in H_B$. Let $g,g' \in H_B$. Since $x$ and $gx$ both lie in $B$, we see that $gB \cap B\neq \emptyset$  and hence that $gB = B$. Now we have $(gg')x = g(g'x) \in gB = B$ and hence $gg' \in H_B$. Also, for $g \in H_B$ we have $gx \in B$ and $g^{-1}(gx) = x \in B$. Thus $g^{-1}B \cap B \neq \emptyset$, which forces $g^{-1}B = B$. In particular,   $g^{-1}x \in B$ and hence  $g^{-1} \in H_B$. Therefore $H_B$ is a subgroup of $G$. 
		
		Observe that $G_x \le H_B$ since $x \in B$. Fix $x\in X$. Let $\mathcal{B}$ be the set of blocks of $X$ containing $x$ and let $\mathcal{H}$ be the set of subgroups containing $G_x$. Let $\theta:\mathcal{B}\rightarrow \mathcal{H}$ be the function defined by $\theta(B)=H_B$. We claim that $\theta$ is bijective. 
		
		Let $B$ and $B'$ be distinct blocks in $\mathcal{B}$. Without loss of generality assume that $B'\not\subseteq B$. Then there exists some  $y \in B'$ with $y \notin B$. Since $G$ acts transitively on $X$, there exists some $g \in G$ such that $gx = y$. So $g \in H_{B'}$. Since $y\not\in B$, we have $g \notin H_B$, and hence $H_B \ne H_{B'}$. This shows that $\theta$ is injective.
		
		Let $H\in\mathcal{H}$. Consider the subset $C = \{hx \mid h \in H\}$ of $X$. We show that $C$ is a block. Clearly $C$ is non-empty and  $gC = C$ for each $g\in H$. Let $g \in G$ be such that $gC \cap C\neq \emptyset$. Then there exist $h_1, h_2 \in H$ such that $gh_1x = h_2x$. This gives $h_2^{-1}gh_1x = x$ and hence $h_2^{-1}gh_1 \in G_x \le H$, and thus $g \in H$. Consequently $gC  = C$. Therefore $C$ is a block. Note that $\theta(C) = H_C = \{g \in G \mid gx \in C\}$. Clearly $H \le H_C$. Let $g \in H_C$. Then $gx = hx$ for some $h \in H$. Hence $h^{-1}gx = x$ and thus $h^{-1}g  \in H$, giving $g \in H$. So $\theta(C) = H$, which shows that $\theta$ is surjective.
	\end{sketch}
\end{comment}

\begin{definition}
	Let $G$ act transitively on $\Omega$. A partition $\Sigma$ of $\Omega$ is called a \textbf{system of blocks} (or a \textbf{system of imprimitivity}) if each $\Delta\in \Sigma$ is a block.
\end{definition}
\begin{proposition}
	Let $G$ act transitively on $\Omega$.  Then the following propositions hold.
	\begin{enumerate}[(i)]
		\item If $\Delta$ is a block of $\Omega$, then $G_{\Delta}$ acts transitively on $\Delta$. 
		\item Let $\Sigma$ be a system of imprimitivity and let $\Delta\in \Sigma$. Then
		$\Sigma  = \{\Delta^g\mid g\in G\}$. In particular, $|\Delta^g| = |\Delta|$.
		\item If $\Delta$ is a subset of $\Omega$, then $\Delta$ is a block if and only if $\{\Delta^g\mid g\in G\}$ forms a partition of $\Omega$.
	\end{enumerate}  
\end{proposition}
\begin{sketch}
	
\end{sketch}

\begin{definition}
	Let $G$ act on $\Omega$. An equivalence relation $\sim$ on $\Omega$ is called a $G$-\textbf{congruence} if  $$\omega_1 \sim \omega_2 \iff \omega_1^g \sim \omega_2^g$$ 
	for all $\omega_1,\omega_2\in \Omega$ and $g\in G$. We also say that $G$ preserves the relation.
\end{definition}
\begin{proposition}
	Let $G$ act transitively on $\Omega$.
	\begin{enumerate}[(i)]
		\item If $\sim$ is a $G$-congruence on $\Omega$, then each equivalence class is a block of $\Omega$.
		\item  If  $\Delta$ is a block, then $\Sigma = \{\Delta^g\mid g\in G\}$ is the set of equivalence classes of a $G$-congruence on $\Omega$. Hence $G$ acts transitively on $\Sigma$.
	\end{enumerate}
\end{proposition}
\begin{sketch}
	
\end{sketch}

\subsection{Primitive Actions}

\begin{definition}
	Let $G$ act transitively on $\Omega$. The action (or $G$-set) is said to be \textbf{primitive} (or $G$ is \textbf{primitive} on $\Omega$) if $G$ has no nontrivial blocks; otherwise, it is \textbf{imprimitive}.
\end{definition}

\begin{proposition} \label{prop-block-and-stab}
	Let $G$ acts transitively on $\Omega$. Let $\omega\in \Omega$ be fixed. Then there is a one-to-one correspondence between the set of blocks of $\Omega$ containing $\omega$ and the set of subgroups which contains the stabilizer $G_\omega$ of $\omega$.
\end{proposition}
\begin{sketch}
	Let $\Delta$ be a block containing $\omega \in \Omega$, and consider the set $H_\Delta = \{g \in G \mid \omega^g \in \Delta\}$. We claim that $H_\Delta$ is a subgroup of  $G$. Clearly, $e \in H_\Delta$. Let $g,g' \in H_\Delta$. Since $\omega$ and $\omega^g$ both lie in $\Delta$, we see that $\Delta^g \cap \Delta\neq \emptyset$ and hence that $\Delta^g = \Delta$. Now we have $\omega^{gg'} = (\omega^g)^{g'} \in \Delta^{g'} = \Delta$ and hence $gg' \in H_\Delta$. Also, for $g \in H_\Delta$ we have $\omega^g \in \Delta$. Since $\omega$ is also in $\Delta$, this means $\Delta \cap \Delta^g \neq \emptyset$, which implies $\Delta^g = \Delta$. Acting on both sides by $g^{-1}$, we get $\Delta^{g^{-1}} = (\Delta^g)^{g^{-1}} = \Delta$. Since $\omega \in \Delta$, it follows that $\omega^{g^{-1}} \in \Delta^{g^{-1}} = \Delta$, and hence $g^{-1} \in H_\Delta$. Therefore $H_\Delta$ is a subgroup of $G$.
	
	Observe that $G_\omega \le H_\Delta$ since for any $s \in G_\omega$, we have $\omega^s=\omega \in \Delta$.  Let $\Sigma$ be the set of blocks of $\Omega$ containing $\omega$ and let $\mathcal{H}$ be the set of subgroups containing $G_\omega$. Let $\theta:\Sigma\rightarrow \mathcal{H}$ be the function defined by $\theta(\Delta)=H_\Delta$. We claim that $\theta$ is bijective.
	
	Let $\Delta$ and $\Delta'$ be distinct blocks in $\Sigma$. Without loss of generality assume that $\Delta'\not\subseteq \Delta$. Then there exists some  $\delta \in \Delta'$ with $\delta \notin \Delta$. Since $G$ acts transitively on $\Omega$, there exists some $g \in G$ such that $\omega^g = \delta$. So $g \in H_{\Delta'}$. Since $\delta\not\in \Delta$, we have $g \notin H_\Delta$, and hence $H_\Delta \ne H_{\Delta'}$. This shows that $\theta$ is injective.
	
	Let $H\in\mathcal{H}$. Consider the subset $C = \{\omega^h \mid h \in H\}$ of $\Omega$. We show that $C$ is a block. Clearly $C$ is non-empty and $C^g = C$ for each $g\in H$. Let $g \in G$ be such that $C^g \cap C\neq \emptyset$. Then there exist $h_1, h_2 \in H$ such that $(\omega^{h_1})^g = \omega^{h_2}$. This gives $\omega^{h_1g} = \omega^{h_2}$, so $\omega^{h_1gh_2^{-1}} = \omega$. Hence $h_1gh_2^{-1} \in G_\omega \le H$, and thus $g \in H$. Consequently $C^g = C$. Therefore $C$ is a block. Note that $\theta(C) = H_C = \{g \in G \mid \omega^g \in C\}$. Clearly $H \le H_C$. Let $g \in H_C$. Then $\omega^g = \omega^h$ for some $h \in H$. Hence $\omega^{gh^{-1}} = \omega$ and thus $gh^{-1} \in G_\omega \le H$, giving $g \in H$. So $\theta(C) = H$, which shows that $\theta$ is surjective.
\end{sketch}
\begin{remark}
	This correspondence is order-preserving, i.e., if $\Delta_1,\Delta_2$ are blocks of $\Omega$ containing $\omega$, then $\Delta_1\subseteq \Delta_2$ if and only if $\theta(\Delta_1)\subseteq \theta(\Delta_2)$. Indeed $\Delta_1\subseteq \Delta_2$ implies that $\omega^g \in \Delta_1\subseteq \Delta_2$ for all $g\in H_{\Delta_1}$. Conversely, suppose that $H_{\Delta_1}\leq H_{\Delta_2}$. Let $\delta\in \Delta_1$. So there exists $g\in G$ such that $\delta = \omega^g$. This implies $g\in H_{\Delta_1}\subseteq H_{\Delta_2}$. So $\delta = \omega^g \in \Delta_2$. This shows that $\theta$ is order-preserving.
\end{remark}

\begin{corollary}
	Let $G$ act transitively on a set $\Omega$. Then $G$ is primitive if and only
	if the stabilizers are maximal subgroups.
\end{corollary}
\begin{sketch}
	Suppose $G$ is primitive on $\Omega$. Let $\omega \in \Omega$.  Since $G$ is primitive, there are  two  blocks containing $\omega$, namely $\{\omega\}$ and $\Omega$.  By Proposition \ref{prop-block-and-stab}, there are only two subgroups of $G$ containing $G_\omega$, namely $G_\omega$ and $G$. So there is no proper subgroup of $G$ which properly contains $G_\omega$. Therefore $G_\omega$ is maximal in $G$.
	
	Conversely, suppose that every stabilizer is a maximal subgroup. Fix $\omega\in \Omega$. Then there are only two subgroups of $G$ containing $G_\omega$, namely $G_\omega$ and $G$. By Proposition \ref{prop-block-and-stab}, we only have two such blocks $\{\omega\}$ and $\Omega$.  Consequently, $\Omega$ can have no other blocks besides itself and its singletons, and so $\Omega$ is primitive.
\end{sketch}

\subsection{Centralizers and Normalizers of Transitive Permutation Groups}
\begin{definition}
	Let $G$ act on a set $\Omega$. We say that $G$ acts \textbf{semiregularly} on $\Omega$ ($G$ or the $G$-set $\Omega$ is \textbf{semiregular}) if nonidentity elements fix no point, i.e., $G_\omega = 1$ for all $\omega\in\Omega$. We say that $G$ acts \textbf{regularly} on $\Omega$ if $G$ is transitive and semiregular.
\end{definition}
\begin{lemma}
	Let $G$ be a group with a subgroup $H$, and put $K := N_G(H)$. Let $\Gamma_H$ denote the set of right cosets of $H$ in $G$, and let $\rho$ and $\lambda$ denote the right and left actions of $G$ and $K$, respectively, on $\Gamma_H$ as defined above. Then the following hold.
\begin{enumerate}[(i)]
\item $\ker \lambda = H$ and $\lambda(K)$ is semiregular.
\item The centralizer $C$ of $\rho(G)$ in $\Sym\Gamma_H$ equals $\lambda(K)$.
\item $H \in \Gamma_H$ has the same orbit under $\lambda(K)$ as under $\rho(K)$.
\item If $\lambda(K)$ is transitive, then $K = G$, and $\lambda(G)$ and $\rho(G)$ are conjugate in $\Sym\Gamma_H$.
\end{enumerate}
\end{lemma}
\begin{sketch}
	
\end{sketch}

\begin{theorem}
	Let $G$ be a transitive subgroup of $\Sym\Omega$, and $\alpha$ a point in $\Omega$. Let $C$ be the centralizer of $G$ in $\Sym\Omega$. Then the following hold.
	\begin{enumerate}[(i)]
		\item $C$ is semiregular, and $C \cong N_G(G_\alpha)/G_\alpha$. In particular, $|C| = |\text{fix}(G_\alpha)|$.
		\item $C$ is transitive if and only if $G$ is regular.
		\item If $C$ is transitive, then it is conjugate to $G$ in $\Sym\Omega$ and hence $C$ is regular.
		\item $C = 1$ if and only if $G_\alpha$ is self-normalizing in $G$, i.e., $N_G(G_\alpha) = G_\alpha$.
		\item If $G$ is abelian, then $C = G$.
		\item If $G$ is primitive and nonabelian, then $C = 1$.
	\end{enumerate}
\end{theorem}
\begin{sketch}
	
\end{sketch}


\begin{theorem}
	Let $G$ be a transitive subgroup of $\Sym\Omega$, let $N$ be the normalizer of $G$ in $\Sym \Omega$ and let $\alpha \in \Omega$. If $\Psi:N\to \Aut G$ is the homomorphism defined by conjugation, and $\sigma \in \Aut G$, then
	$ \sigma \in \operatorname{im} \Psi$ if and only if $(G_\alpha)^\sigma$ is a point stabilizer for $G$, i.e., $(G_\alpha)^\sigma = G_\beta$ for some $\beta\in\Omega$.
\end{theorem}
\begin{sketch}
	Let $\sigma \in\operatorname{im} \Psi$, so $\sigma = \Psi(x)$ for some $x \in N$. Then $(G_\alpha)^\sigma = x^{-1}G_\alpha x = G_\beta$ where $\beta = \alpha^x$. Conversely, suppose that $(G_\alpha)^\sigma = G_\beta$ for some $\beta \in \Omega$. Then the two transitive permutation representations of $G$ into $\Sym\Omega$ given by $x \mapsto x$ and $x \mapsto x^\sigma$ are equivalent because $G_\beta$ is a point stabilizer for each of them. This means that for some $t \in \Sym\Omega$ we have $xt = tx^\sigma$ for all $x \in G$. Clearly $t \in N$. Hence $\sigma = \Psi(t) \in\operatorname{im} \Psi$ as required.
\end{sketch}

\begin{definition}
	Let $G$ be a group. The \textbf{holomorph} of $G$, denoted by $\operatorname{Hol} G$, is the semidirect product $G \rtimes \Aut G$ with respect to the natural action of $\Aut G$ on $G$.
\end{definition}
In the case where $G$ is regular,  the normalizer of $G$ in the symmetric group is the holomorph of $G$.
\begin{corollary}
	Let $G$ be a transitive subgroup of $\Sym\Omega$ and let $N$ be the normalizer of $G$ in $\Sym \Omega$. If $G$ is regular, then $\operatorname{im} \Psi = \Aut G$. In this case $N_\alpha \cong \Aut G$, and $N$ is isomorphic to $\operatorname{Hol}G$.
\end{corollary}
\begin{sketch}
	Since $G$ is regular, therefore $G_\alpha = 1$, and so $\operatorname{im} \Psi = \Aut G$ by theorem xxx. The centralizer $C$ of $G$ in $\text{Sym}(\Omega)$ is regular and isomorphic to $G$ by Theorem 4.2A, and therefore $N = CN_\alpha$ with $C \triangleleft N$ and $C \cap N_\alpha = 1$. Hence $\Aut G =\operatorname{im} \Psi \cong N/\ker \Psi = N/C \cong N_\alpha$. Finally, because $G$ is regular and normal in $N$, therefore $G \cap N_\alpha = 1$ and $N = GN_\alpha \cong G \rtimes \Aut G$.
\end{sketch} 

\paragraph{Main References.} \cite{Praeger2018,Dixon1996,Cameron1999}