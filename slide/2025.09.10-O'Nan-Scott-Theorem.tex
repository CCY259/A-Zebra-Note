\section{O'Nan-Scott Theorem: The Classification of Maximal Subgroups of Symmetric Groups}
\begin{definition}
	Let $G$ be a group. A normal subgroup $N$ of $G$ is said to be \textbf{minimal} if $N\neq 1$ and there is no normal subgroup $M$ of $G$ such that $1<M< N$.
\end{definition}
\begin{remark}
	In a  finite group, every nontrivial normal subgroup contains a minimal normal subgroup, but it is not necessarily true for infinite group ($\mathbb{Z}$ is a counterexample). The argument is similar to Proposition \ref{prop-maximal-grp-exist}.
\end{remark}
\begin{proposition} 
	Every minimal normal subgroup of a finite group is a direct product of isomorphic simple groups.
\end{proposition}

\begin{definition}
	Let $G$ be a group. The product of all minimal normal subgroups of $G$ is called the \textbf{socle} of $G$ and is denoted by $\operatorname{soc}(G)$.
\end{definition}
\subsection{Classes of Groups} \label{sec-classes-of-grps}
\subsubsection{Intransitive Groups}
\begin{proposition}
	Let $G$ be a intransitive permutation group on $\Omega$ of degree $n$. Then $G$ is permutationally isomorphic to a subgroup of $S_k\times S_{n-k}$ for some $1<k< n$.
\end{proposition}
\subsubsection{Transitive Imprimitive Groups}
\begin{proposition}
	Let $G$ be a transitive and imprimitive permutation group on $\Omega$ of degree $n$. Then $G$ is permutationally isomorphic to a subgroup of $S_k \wr S_m$, where $1<k,m<n$ and $n = km$.
\end{proposition}


\subsubsection{Affine Groups}

Type (A)


\begin{proposition}
	Let $ G $ be a primitive permutation group on $ \Omega $ of degree $ n $ with an abelian normal subgroup $ A \neq 1 $. Then $ A = C_G(A) $ is the unique minimal normal subgroup of $ G $ and $ n = |A| = p^m $ for some prime number $ p $. Furthermore, $G$ is the internal direct sum of $A$ by $G_\omega$ for $\omega\in\Omega$ and $G$ is permutationally isomorphic to a subgroup of $\operatorname{Aff}(m,p) $ containing $ \mathbb{F}_p^m$ for $ \omega \in \Omega $.
\end{proposition}
\subsubsection{Almost Simple Groups}
Type (S)
\begin{definition}
	A group $G$ is said to be \textbf{almost simple} if there exists a nonabelian simple group $S$ such that $S\leq G\leq \Aut S$.
\end{definition}
\begin{proposition} 
	Let $G$ be a finite primitive group with a nonabelian simple socle. Then $G$ is almost simple.
\end{proposition}

\subsubsection{Groups of Diagonal Type}
Type (D)

\subsubsection{Groups of Product Type}
Type (P)

\subsubsection{Twisted Wreath Products}
Type (T)

\subsection{Main Result}
\begin{theorem}[Aschbacher-O'Nan-Scott Theorem]
	 Let $H$ be a subgroup of $S_n$. Then one of the following holds.
	\begin{enumerate}[(i)]
		\item $H$ is a subgroup of $S_k \times S_{n-k}$, where $1<k<n$;
		\item $H$ is a subgroup of $S_k \wr S_m$, where $n = km$;
		\item $H$ is one of the types (A), (S), (D), (P) and (T) described in Section \ref{sec-classes-of-grps}.
	\end{enumerate}
\end{theorem}

\paragraph{Main References.} \cite{Wilson2009,Cameron1999,Smith2018,Liebeck1988,Aschbacher1985}