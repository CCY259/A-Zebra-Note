\section{Wreath Products}
\subsection{Construction}
Let $G$ be an arbitrary group and let $H$ be a group acting on a set $\Delta$. Let $G^\Delta$ be the group of all functions from $\Delta$ into $G$ with pointwise multiplication. In other words,  $G^\Delta$ is isomorphic to a direct product of copies of $G$ indexed by $\Delta$. Now we define an action of $H$ on  $G^\Delta$ by 
\begin{equation*}
	f^h(\delta) = f(\delta^{h^{-1}}).
\end{equation*}
\begin{definition}
	The semidirect product $W = G^\Delta \rtimes H$ with respect to the action defined above  is called the \textbf{(complete) wreath product} of $G$ by $H$. This $W$ will be denoted by $G \Wr_{\Delta} H$. We can define the \textbf{restricted wreath product} $G \wr_{\Delta} H$ of $G$ by $H$, by using the same setting, but with $f\in G^{(\Delta)}$, where $G^{(\Delta)}$ is the set of all functions such that $f(y) = e$ for all but finitely many $y\in \Delta$. The wreath product is said to be \textbf{trivial} if either $G$ or $H$ is the trivial group. In $G\Wr_\Delta H$ (resp. $G\wr_\Delta H$), the subgroup $G^\Delta$ (resp. $G^{(\Delta)}$) is called the \textbf{base group}, $G$ is called the \textbf{bottom group} and $H$ is called the \textbf{top group}.
\end{definition}

\begin{remark}
	Let us restrict to the case where $\Delta$ is a finite set. If $\Delta = \{1,\dots,n\}$, then we can think of the base group as $$G^\Delta = \underbrace{G\times \cdots \times G}_{n\text{ times}}.$$
	So the elements of the base group are just an $n$-tuples of elements in $G$. Now the action of $H$ on the base group corresponds to permuting coordinates of elements in $G$ as follows.
	\begin{equation*}
		(g_1,g_2,\dots, g_n)^h = (g_{1^{h^{-1}}},g_{2^{h^{-1}}},\dots, g_{n^{h^{-1}}}).
	\end{equation*}
	Note that it is necessary to introduce $h^{-1}$ rather than $h$ since we are considering right action.
\end{remark}


\begin{definition}
	The \textbf{standard wreath product} $G\Wr H$ (or $G\wr H$) of $G$ by $H$ is the wreath product $G\Wr_H H$ (or $G\wr_H H$) where $G$ and $H$ act on themselves by right regular action.
\end{definition}

\subsection{Imprimitive Action}
\begin{proposition} \label{prop-imprimitive-action-of-WP}
	Let $G$ and $H$ act on $\Omega$ and $\Delta$ respectively. Then $G\Wr_\Delta H$  acts on $\Omega\times \Delta$ via
	\begin{equation*}
		(\omega,\delta)^{(f,h)} = (\omega^{f(\delta)},\delta^h).
	\end{equation*} 
\end{proposition}
\begin{sketch}
	Let $(\omega,\delta)\in \Omega\times \Delta$. 
	Clearly $(\omega,\delta)^{(1,1)} =  (\omega^{1(\delta)},\delta^1) = (\omega^{1},\delta) = (\omega,\delta)$. Let $f_1,f_2\in G^\Delta$ and $h_1,h_2\in H$. Then  
\begin{align*}
	((\omega,\delta)^{(f_1,h_1)})^{(f_2,h_2)} &= (\omega^{f_1(\delta)},\delta^{h_1})^{(f_2,h_2)} 
	\\
	&= ((\omega^{f_1(\delta)})^{f_2(\delta^{h_1})},(\delta^{h_1})^{h_2})
	\\
	&= (\omega^{f_1(\delta)f_2(\delta^{h_1})},\delta^{h_1h_2}),
	\\
	(\omega,\delta)^{(f_1,h_1)(f_2,h_2)} &= (\omega,\delta)^{(f_1f_2^{h_1^{-1}},h_1h_2)}
	\\
	&= (\omega^{(f_1f_2^{h_1^{-1}})(\delta)},\delta^{h_1h_2})
	\\
	&= (\omega^{f_1(\delta)f_2^{h_1^{-1}}(\delta)},\delta^{h_1h_2})
	\\
	&= (\omega^{f_1(\delta)f_2(\delta^{h_1})},\delta^{h_1h_2}).
\end{align*}
So $((\omega,\delta)^{(f_1,h_1)})^{(f_2,h_2)}  = (\omega,\delta)^{(f_1,h_1)(f_2,h_2)}$. This shows that $G\Wr_{\Delta} H$ acts on $\Omega\times \Delta$ with respect to this action.
\end{sketch}
\begin{definition}
	The action defined in Proposition \ref{prop-imprimitive-action-of-WP} is called the \textbf{imprimitive action} of $G\Wr_{\Delta} H$.
\end{definition}

\begin{proposition} \label{prop-imprim-WP}
	Let $G$ and $H$ act on $\Omega$ and $\Delta$ respectively. 
	\begin{enumerate}[(i)]
		\item $G\Wr_\Delta H$ acts transitively on $\Omega\times \Delta$   if and only if both $G$ and $H$ are transitive.
		\item  $G\Wr_\Delta H$ acts faithfully on $\Omega\times \Delta$ if and only if both actions are faithful.
		\item If $|\Omega|,|\Delta|\geq 2$ and $G\Wr_\Delta H$ acts transitively on $\Omega\times \Delta$, then $G\Wr_\Delta H$ is imprimitive on $\Omega\times \Delta$.
		\item In this form the wreath product is associative in the sense that, if $K$ acts on $\Gamma$, then the action of $(G\Wr_\Delta H) \Wr_\Gamma K$ on $(\Omega\times \Delta) \times \Gamma$  is permutationally isomorphic to the action of $G\Wr_{\Delta\times \Gamma} (H \Wr_\Gamma K)$ on $\Omega\times (\Delta \times \Gamma)$.
	\end{enumerate}
\end{proposition}

\begin{sketch}
	(i) Suppose that $G\Wr_\Delta H$ is transitive on $\Omega\times \Delta$. Then for any $\omega_1,\omega_2\in \Omega$ and $\delta_1,\delta_2\in \Delta$, consider the pairs $(\omega_1,\delta_1),(\omega_2,\delta_2) \in \Omega\times\Delta$. 
	
	Conversely, let $(\omega_1,\delta_1),(\omega_2,\delta_2) \in \Omega\times\Delta$. Then we take a function $f:\Delta \to G$ such that $f(\delta_1)=g$ where $\omega_1^g =\omega_2$.
	
	(ii) Suppose that the action of $G\Wr_\Delta H$ on $\Omega\times \Delta$ is faithful. Let $g\in G$ be such that $\omega^g = \omega$ for all $\omega\in \Omega$. There is a function $f:\Delta\to G$ such that $f(\delta) = g$ for all $\delta\in \Delta$. So $\omega^{f(\delta)} = \omega$ for all $\delta\in \Delta$. Note that
	\begin{equation*}
		(\omega,\delta)^{(f,1)} = (\omega^{f(\delta)}, \delta^1) = 	(\omega,\delta).
	\end{equation*} This implies $g  = 1$. We use a similar argument to show that $H$ is faithful on $\Delta$.
	
	Conversely, let $(f,h)\in G\Wr_{\Delta} H$ be such that  $(\omega,\delta)^{(f,h)} = (\omega,\delta)$ for all $(\omega,\delta)\in\Omega\times \Delta$. In particular, $\delta^h = \delta$ for all $\delta\in \Delta$. Since the action of $H$ on $\Delta$ is faithful, we get $h = 1$. Next, note that $\omega^{f(\delta)} = \omega$ for all $\omega\in\Omega$, $\delta\in\Delta$. Since the action of $G$ on $\Omega$ is faithful, we get $f(\delta) = 1$ for all $\delta\in\Delta$.
	
	(iii) Let $\delta\in \Delta$ and consider the subset $\Omega\times \{\delta\}$ of $\Omega\times \Delta$. We show that it is a nontrivial block. Let $(f,h)\in G\Wr_{\Delta} H$ be  such that $(\Omega\times \{\delta\})^{(f,h)} \cap (\Omega\times \{\delta\}) \neq \emptyset$. Then $\delta^h = \delta$ and $\Omega^{f(\delta)} \cap \Omega \neq \emptyset$. Since $\Omega$ itself is a block, it follows that $\Omega^{f(\delta)} =\Omega$. Hence $(\Omega\times \{\delta\})^{(f,h)} = \Omega\times \{\delta\}$.
	
	(iv) The elements of $(G\Wr_\Delta H) \Wr_\Gamma K$ are of the form $(f,k)$ where $f:\Gamma\to G\Wr_\Delta H$ is a function such that $f(\gamma) = (g_\gamma,h_\gamma)$ with a function $g_\gamma:\Delta\to G$ and an element $h_\gamma\in H$. On the other hand, the elements of $G\Wr_{\Delta\times \Gamma} (H \Wr_\Gamma K)$ are of the form $(g,(h,k))$ where $g:\Delta\times\Gamma\to G$ and $h:\Gamma \to H$ are functions. 
	
	Let $\varphi:(G\Wr_\Delta H) \Wr_\Gamma K \to G\Wr_{\Delta\times \Gamma} (H \Wr_\Gamma K)$ be defined by
	\begin{equation*}
		\varphi(f,k) = (g_\varphi,(h_\varphi,k))
	\end{equation*}
	where $f$ is as described above, $g_\varphi:\Delta\times\Gamma\to G$ and $h_\varphi:\Gamma \to H$ are functions respectively defined by
\begin{equation*}
	g_\varphi(\delta,\gamma) = g_\gamma(\delta),\quad 	h_\varphi(\gamma)= h_\gamma.
\end{equation*}
Now we show that $\varphi$ is bijective. Let $(f^{(1)},k_1),(f^{(2)},k_2)\in (G\Wr_\Delta H) \Wr_\Gamma K$ with $f^{(i)}(\gamma) = (g^{(i)}_\gamma, h^{(i)}_\gamma)$  for $i=1,2$.
Assume $\varphi(f^{(1)}, k_1) = \varphi(f^{(2)}, k_2)$, or equivalently,
\[
(g_\varphi^{(1)}, (h_\varphi^{(1)}, k_1)) = (g_\varphi^{(2)}, (h_\varphi^{(2)}, k_2)).
\]
So $k_1 = k_2$,   $h_\varphi^{(1)} = h_\varphi^{(2)}$ and  $g_\varphi^{(1)} = g_\varphi^{(2)}$. Since $h_\varphi^{(1)} = h_\varphi^{(2)}$, we get
\begin{equation*}
	h_{\gamma}^{(1)} = h_\varphi^{(1)}(\gamma) = h_\varphi^{(2)}(\gamma) = 	h_{\gamma}^{(2)}
\end{equation*}
for all $\gamma\in\Gamma$. Since $g_\varphi^{(1)} = g_\varphi^{(2)}$, we get 
\begin{equation*}
	g_\gamma^{(1)}(\delta) = g_{\varphi}^{(1)}(\delta,\gamma) = g_{\varphi}^{(2)}(\delta,\gamma) = g_\gamma^{(2)}(\delta)
\end{equation*} for all $\delta \in \Delta$ and $\gamma \in \Gamma$. This implies that  $g_\gamma^{(1)}=g_\gamma^{(2)}$ for all $\gamma \in \Gamma$. Hence  $f^{(1)}(\gamma) = f^{(2)}(\gamma)$ for all $\gamma \in \Gamma$. Therefore $f^{(1)} = f^{(2)}$. Consequently, $(f^{(1)}, k_1) = (f^{(2)}, k_2)$. Thus $\varphi$ is injective.

Let $(g, (h, k))$ be an arbitrary element of $G \operatorname{Wr}_{\Delta \times \Gamma} (H \operatorname{Wr}_\Gamma K)$.
Recall that $g: \Delta \times \Gamma \to G$ is a function, $h: \Gamma \to H$ is a function, and $k \in K$.  Define the function $f: \Gamma \to G \operatorname{Wr}_\Delta H$ by $f(\gamma) = (g_\gamma, h(\gamma))$, where  $g_\gamma:\Delta\to G$ is a function defined by $g_\gamma(\delta) = g(\delta,\gamma)$. Now we have 
$
\varphi(f, k) = (g_\varphi, (h_\varphi, k))$.
Clearly $h_\varphi(\gamma) = h(\gamma)$ for all $\gamma \in \Gamma$ and   $g_\varphi(\delta, \gamma) = g_\gamma(\delta) = g(\delta, \gamma)$ for all $(\delta, \gamma) \in \Delta \times \Gamma$.   Therefore, $\varphi$ is surjective.

We verify that $\varphi$ is a homomorphism. Let $(f^{(1)},k_1),(f^{(2)},k_2)\in (G\Wr_\Delta H) \Wr_\Gamma K$ with $f^{(i)}(\gamma) = (g^{(i)}_\gamma, h^{(i)}_\gamma)$  for $i=1,2$. Then
\begin{equation*}
	\varphi((f^{(1)},k_1)(f^{(2)},k_2)) = \varphi((f^{(1)}(f^{(2)})^{k_1^{-1}},k_1k_2)).
\end{equation*}
Note that 
\begin{align*}
	f^{(1)}(f^{(2)})^{k_1^{-1}}(\gamma) &= f^{(1)}(\gamma)(f^{(2)})^{k_1^{-1}}(\gamma) 
	\\
	&= 
	f^{(1)}(\gamma)f^{(2)}(\gamma^{k_1}) 
	\\
	&=(g^{(1)}_\gamma, h^{(1)}_\gamma)(g^{(2)}_{\gamma^{k_1}}, h^{(2)}_{\gamma^{k_1}})
	\\
	&= \left(g^{(1)}_\gamma (g^{(2)}_{\gamma^{k_1}})^{(h^{(1)}_\gamma)^{-1}}, h^{(1)}_\gamma h^{(2)}_{\gamma^{k_1}}\right)
\end{align*} for $\gamma\in\Gamma$. So we have $\varphi((f^{(1)},k_1)(f^{(2)},k_2)) = (g',(h',k_1k_2))$ where $g':\Delta\times\Gamma\to G$ and $h':\Gamma \to H$ are functions respectively given by
\begin{equation*}
	g'(\delta,\gamma) = g^{(1)}_\gamma(\delta)g^{(2)}_{\gamma^{k_1}}(\delta^{h_\gamma^{(1)}}),\quad h'(\gamma) = h^{(1)}_\gamma h^{(2)}_{\gamma^{k_1}}.
\end{equation*}
Next, we have 
\begin{align*}
	\varphi(f^{(1)},k_1) \varphi(f^{(2)},k_2) &= \left(g_\varphi^{(1)},(h_\varphi^{(1)},k_1)\right)\left(g_\varphi^{(2)},(h_\varphi^{(2)},k_2)\right)
	\\
	&= \left(g_\varphi^{(1)}(g_\varphi^{(2)})^{(h_\varphi^{(1)},k_1)^{-1}},(h_\varphi^{(1)},k_1)(h_\varphi^{(2)},k_2)\right)
	\\
	&=  \left(g_\varphi^{(1)}(g_\varphi^{(2)})^{(h_\varphi^{(1)},k_1)^{-1}},(h_\varphi^{(1)}(h_\varphi^{(2)})^{k_1^{-1}},k_1k_2)\right)
	\\
	&= (g',(h',k_1k_2))
\end{align*}
because
\begin{align*}
\left[g_\varphi^{(1)}(g_\varphi^{(2)})^{(h_\varphi^{(1)},k_1)^{-1}}\right] (\delta,\gamma) &= g_\varphi^{(1)}(\delta,\gamma) (g_\varphi^{(2)})^{(h_\varphi^{(1)},k_1)^{-1}}(\delta,\gamma)
\\
&= g_\gamma^{(1)}(\delta) g_\varphi^{(2)}((\delta,\gamma)^{(h_\varphi^{(1)},k_1)})
\\
&= g_\gamma^{(1)}(\delta) g_\varphi^{(2)}(\delta^{h_\gamma^{(1)}},\gamma^{k_1})
\\
&= g_\gamma^{(1)}(\delta) g_{\gamma^{k_1}}^{(2)}(\delta^{h_\gamma^{(1)}}),
\\
\left[h_\varphi^{(1)}(h_\varphi^{(2)})^{k_1^{-1}} \right](\gamma)&=  h_\varphi^{(1)}(\gamma)(h_\varphi^{(2)})^{k_1^{-1}}(\gamma)
\\
&= h_\varphi^{(1)}(\gamma)(h_\varphi^{(2)})(\gamma^{k_1})
\\
&= h_\gamma^{(1)}h_{\gamma^{k_1}}^{(2)}.
\end{align*}
Finally, let $\vartheta:(\Omega\times \Delta)\times \Gamma \to \Omega\times (\Delta\times \Gamma)$ be defined by $\vartheta((\omega,\delta),\gamma) = (\omega,(\delta,\gamma))$. We claim that $(\vartheta,\varphi)$ is a permutational isomorphism. Let $((\omega,\delta),\gamma)\in (\Omega\times \Delta)\times \Gamma$ and $(f,k)\in (G\Wr_\Delta H) \Wr_\Gamma K$ be such that $f(\gamma)= (g_\lambda,h_\lambda)$ and $\varphi(f,k) = (g_\varphi,(h_\varphi,k))$. Then
\begin{align*}
	\vartheta\left(((\omega,\delta),\gamma)^{(f,k)}\right) &= \vartheta\left((\omega,\delta)^{f(\gamma)},\gamma^k\right) = \vartheta\left((\omega,\delta)^{(g_\gamma,h_\gamma)},\gamma^k\right)
	 \\
	 &= \vartheta\left((\omega^{g_\gamma(\delta)},\delta^{h_\gamma}),\gamma^k\right)
	 = \left(\omega^{g_\gamma(\delta)},(\delta^{h_\gamma},\gamma^k)\right)
	 \\
	 &= \left(\omega^{g_{\varphi}(\delta,\gamma)},(\delta^{h_\varphi(\gamma)},\gamma^k)\right)
	= \left(\omega^{g_{\varphi}(\delta,\gamma)},(\delta,\gamma)^{(h_\varphi,k)}\right)
	 \\
	 &= (\omega,(\delta,\gamma))^{(g_{\varphi},(h_\varphi,k))}
	  = (\omega,(\delta,\gamma))^{\varphi(f,k)}.
\end{align*}
This completes the proof.
\end{sketch}

\begin{remark}\begin{enumerate}[(i)]
		\item 	Proposition \ref{prop-imprim-WP}.(iii) is the reason why we call the action imprimitive.
		\item The standard wreath product is not associative in general. Let $H$, $K$ and $L$ be three groups. Then
\begin{gather*}
	|H \Wr K| = |H|^{|K|} |K| = |H|^{|K|} |K|,
	\\
	|(H \Wr K) \Wr L| = |H \Wr K|^{|L|} |L| = |H|^{|K||L|} |K|^{|L|} |L|.
\end{gather*}  But $|K \Wr L| = |K|^{|L|} |L|$ and so $$|H \Wr (K \Wr L)| = |H|^{|K \Wr L|} |K \Wr L| = |H|^{|K|^{|L|} |L|} |K|^{|L|} |L|.$$
 Thus $(H \Wr K) \Wr L$ does not necessarily have the same order  as $H \Wr (K \Wr L)$. 
	\end{enumerate}

\end{remark}

\begin{theorem}[Embedding Theorem] \label{thm-embedding}
	Let $G$ act transitively on $\Omega$. Let $\mathcal{B} = \{\Omega_\lambda\mid \lambda\in\Lambda\}$ be a system of imprimitivity of $\Omega$. Fix a block $\Omega_\iota \in\mathcal{B}$ and let $\phi:G\rightarrow \Sym\mathcal{B}$ be the induced representation of $G$ on $\mathcal{B}$. Then $(G,\Omega)$ is permutationally embedded into $(G_{\Omega_\iota} \Wr_\mathcal{B} \phi(G), \Omega_\iota\times \mathcal{B})$.
\end{theorem}
\begin{sketch}
	Since $G$ acts transitively on $\mathcal{B}$, for each $\Omega_\lambda\in \mathcal{B}$ we choose an element $g_\lambda\in G$ such that $\Omega_\lambda = \Omega_\iota^{g_{\lambda}}$. Define $\vartheta: \Omega \to \Omega_\iota \times \mathcal{B}$ by
	\begin{equation*}
		\vartheta(\omega) = (\omega^{g_{\lambda}^{-1}},\Omega_\lambda) 
	\end{equation*}
	where $\omega \in \Omega_\lambda$. Clearly $\vartheta$ is well-defined. We claim that $\vartheta$ is a bijection. Suppose that $\vartheta(\omega_1) = \vartheta(\omega_2)$ for some $\omega_1,\omega_2\in\Omega$. Then $\omega_1,\omega_2\in \Omega_\lambda$ for some $\Omega_\lambda\in\mathcal{B}$ and $\omega_1^{g_{\lambda}^{-1}}=\omega_2^{g_{\lambda}^{-1}}$. Thus we get $\omega_1 = \omega_2$ and $\vartheta$ is injective. Let $(\omega,\Omega_\lambda)\in \Omega_\iota \times \mathcal{B}$. Take the unique element $\delta\in \Omega$ such that $\delta = \omega^{g_\lambda}$. Then, by definition, $\vartheta$ is surjective.
	
	Let $\psi:G \to G_{\Omega_\iota} \Wr_{\mathcal{B}} \phi(G)$ be defined by
	\begin{equation*}
		\psi(x) = (f_x,\phi(x))
	\end{equation*}
	where $f_x$ is a function from $\mathcal{B}$ into $G_{\Omega_\iota}$ with $f_x(\Omega_\lambda) = g_{\lambda}xg_{\lambda^x}^{-1}$ where $g_{\lambda^x}$ is the element corresponding to $\Omega_\lambda^x$ It can be checked that $\Omega_\iota^{g_\lambda xg_{\lambda^x}^{-1}} = \Omega_\iota$ and so $g_\lambda xg_{\lambda^x}^{-1}\in G_{\Omega_\iota}$. We claim that $\psi$ is a monomorphism. Clearly $\psi$ is injective, because $x\in \ker \psi$ implies that  $x =g_\iota^{-1} 1 g_{\iota^x} = g_\iota^{-1} 1 g_{\iota} =1$. Now we show that $\psi$ is a homomorphism. Let $x,y\in G$. Then $
		\psi(x)\psi(y) = (f_x,x)(f_y,y) = (f_xf_y^{x^{-1}},xy)$. Note that for all $\Omega_\lambda \in \mathcal{B}$, 
\begin{align*}
	(f_xf_y^{x^{-1}})(\Omega_\lambda) &= f_x(\Omega_\lambda)f_y^{x^{-1}}(\Omega_\lambda) 
	\\
	&= f_x(\Omega_\lambda)f_y(\Omega_\lambda^x) 
	\\
	&= g_{\lambda}xg_{\lambda^x}^{-1}g_{\lambda^x}yg_{(\lambda^{x})^y}^{-1} 
	\\
	&= g_\lambda xy g_{\lambda^{xy}}^{-1} 
	\\
	&= f_{xy}(\Omega_\lambda).
\end{align*}
So $\psi(x)\psi(y) = \psi(xy)$. Next, we want to prove
\begin{equation*}
	\vartheta(\omega^x) = \vartheta(\omega)^{\psi(x)}
\end{equation*}
for all $\omega\in\Omega$ and $x\in G$.
\begin{align*}
	\vartheta(\omega)^{\psi(x)} &= (\omega^{g_\lambda^{-1}},\Omega_\lambda)^{(f_x,x)} 
	\\
	&= ((\omega^{g_\lambda^{-1}})^{f_x(\Omega_\lambda)},\Omega_\lambda^x) 
	\\
	&= ((\omega^{g_\lambda^{-1}})^{g_\lambda x g_{\lambda^x}^{-1}},\Omega_\lambda^x) 
	\\
	&= ((\omega^x)^{g_{\lambda^x}^{-1}},\Omega_\lambda^x) 
	\\
	&= \vartheta(\omega^x).
\end{align*}
This shows that $(\vartheta,\psi)$ is a permutational embedding.
\end{sketch}

\begin{corollary}[Kaluzhnin–Krasner Embedding Theorem]
	Let $G$ be a group and let $N$ be a subgroup of $G$. Let $H$ be the permutation group induced by $G$ on the set of right cosets of $N$ in $G$. Then $G$ is embedded into $N \Wr H$. If $N$ is normal in $G$, then $G$ is embedded into $N\Wr G/N$.
\end{corollary}
\begin{sketch}
	Apply Theorem \ref{thm-embedding} using the right regular action. 
\end{sketch}


\subsection{Product Action}
\begin{proposition} \label{prop-product-action-of-WP}
	Let $G$ and $H$ act on $\Omega$ and $\Delta$ respectively. Then $G\Wr_\Delta H$  acts on $\Omega^\Delta$ via
	\begin{equation*}
	\varphi^{(f,h)}(\delta) = \varphi(\delta^{h^{-1}})^{f(\delta^{h^{-1}})}.
	\end{equation*} 
\end{proposition}
\begin{sketch}
	Let $\varphi\in \Omega^\Delta$ and let $\delta\in \Delta$. 
	Clearly $\varphi^{(1,1)}(\delta) = \varphi(\delta^{1})^{1(\delta^{1})} = \varphi(\delta)^{1} = \varphi(\delta)$. Let $f_1,f_2\in G^\Delta$ and $h_1,h_2\in H$. Then  
	\begin{align*}
		(\varphi^{(f_1,h_1)})^{(f_2,h_2)}(\delta) &= \varphi^{(f_1,h_1)}(\delta^{h_2^{-1}})^{f_2(\delta^{h_2^{-1}})}
		\\
		&= (\varphi((\delta^{h_2^{-1}})^{h_1^{-1}})^{f_1((\delta^{h_2^{-1}})^{h_1^{-1}})})^{f_2(\delta^{h_2^{-1}})}
		\\
		&= (\varphi(\delta^{(h_1h_2)^{-1}})^{f_1(\delta^{(h_1h_2)^{-1}})})^{f_2(\delta^{h_2^{-1}})}
		\\
		&= \varphi(\delta^{(h_1h_2)^{-1}})^{f_1(\delta^{(h_1h_2)^{-1}})f_2(\delta^{h_2^{-1}})},
		\\
\varphi^{(f_1,h_1)(f_2,h_2)}(\delta)&= \varphi^{(f_1f_2^{h_1^{-1}},h_1h_2)}(\delta)
\\
&= \varphi(\delta^{(h_1h_2)^{-1}})^{(f_1f_2^{h_1^{-1}})(\delta^{(h_1h_2)^{-1}})}
\\
&= \varphi(\delta^{(h_1h_2)^{-1}})^{f_1(\delta^{(h_1h_2)^{-1}})f_2^{h_1^{-1}}(\delta^{(h_1h_2)^{-1}})}
\\
&= \varphi(\delta^{(h_1h_2)^{-1}})^{f_1(\delta^{(h_1h_2)^{-1}})f_2((\delta^{(h_1h_2)^{-1}})^{h_1})}
\\
&= \varphi(\delta^{(h_1h_2)^{-1}})^{f_1(\delta^{(h_1h_2)^{-1}})f_2((\delta^{h_2^{-1}})}.
\end{align*}
	So $(\varphi^{(f_1,h_1)})^{(f_2,h_2)}  =\varphi^{(f_1,h_1)(f_2,h_2)}$. This shows that $G\Wr_{\Delta} H$ acts on $\Omega^\Delta$ with respect to this action.
\end{sketch}
\begin{definition}
	The action defined in Proposition \ref{prop-product-action-of-WP} is called the \textbf{product action} of $G\Wr_{\Delta} H$.
\end{definition}
\begin{proposition} \label{prop-product-action-transitive}
Let $G$ and $H$ act on $\Omega$ and $\Delta$ respectively.  Then $G\Wr_\Delta H$ acts transitively on $\Omega^\Delta$ if and only if $G$ is transitive on $\Omega$.
\end{proposition}
\begin{sketch}
	Suppose that $G\Wr_\Delta H$ acts transitively on $\Omega^\Delta$. Let $\omega_1,\omega_2\in G$. Then take two constant functions $\varphi_1,\varphi_2\in \Omega^\Delta$ whose images are $\omega_1$ and $\omega_2$ respectively. So there is $(f,h)\in G\Wr_{\Delta} H$ such that $\varphi_1^{(f,h)} = \varphi_2$. In particular, $\omega_1^{f(\delta^{h^{-1}})} = \omega_2$ for some $\delta\in\Delta$.
	
	Conversely, suppose that $G$ is transitive on $\Omega$. Let $\varphi_1,\varphi_2\in \Omega^\Delta$. For each $\delta\in\Delta$, there exists $g_\delta\in G$ such that $\varphi_1(\delta)^{g_\delta} = \varphi_2(\delta)$. Let $f:\Delta\to G$ be defined by $f(\delta) = g_\delta$. Then we see that $\varphi_1^{(f,1)}(\delta) = \varphi_1(\delta)^{f(\delta)} = \varphi_1(\delta)^{g_\delta} = \varphi_2(\delta)$ for all $\delta\in\Delta$. Hence $G\Wr_\Delta H$ is transitive on $\Omega^\Delta$.
\end{sketch}
\begin{proposition}
	Let $G$ and $H$ act on $\Omega$ and $\Delta$ respectively. Assume that $|\Omega|\geq 2$. Then $G\Wr_\Delta H$ acts faithfully on $\Omega^\Delta$ if and only if both actions are faithful. 
\end{proposition}
\begin{sketch}
	Suppose that $G\Wr_\Delta H$ acts faithfully on $\Omega^\Delta$. Let $g\in G$ be such that $\omega^g =\omega$ for all $\omega\in\Omega$. There is a function $f:\Delta\to G$ such that $f(\delta) = g$ for all $\delta\in\Delta$. Note that for any $\varphi\in \Omega^\Delta$ and $\delta\in \Delta$, we have
	\begin{equation*}
		\varphi^{(f,1)}(\delta) = \varphi(\delta)^{f(\delta)} = \varphi(\delta)^g = \varphi(\delta).
	\end{equation*}
	Thus $f(\delta) = 1$ for all $\delta\in\Delta$ and so $g = 1$. Now let $h\in H$ be such that $\delta^h = \delta$ for all $\delta\in\Delta$. Then for  any $\varphi\in \Omega^\Delta$ and $\delta\in \Delta$, 
	\begin{equation*}
		\varphi^{(1,h)}(\delta) = \varphi(\delta^{h^{-1}})^{1(\delta^{h^{-1}})} = \varphi(\delta)^1 = \varphi(\delta)
	\end{equation*}
	where $1:\Delta\to G$ is the trivial map. Hence $h=1$. Therefore $G$ and $H$ are faithful on $\Omega$ and $\Delta$ respectively.
	
	Conversely, let $(f,h)\in G\Wr_{\Delta} H$ such that $\varphi^{(f,h)}(\delta) = \varphi(\delta)$ for all $\varphi:\Delta\to \Omega$ and $\delta\in \Delta$. Then $
		\varphi(\delta^{h^{-1}})^{f(\delta^{h^{-1}})} = \varphi(\delta)$ for all $\delta\in \Delta$. Let $\delta\in\Delta$ be fixed. For each $\omega\in \Omega$, we can find a function $\varphi_\omega\in \Omega^\Delta$ such that  $\varphi_\omega(\delta^{h^{-1}}) =\varphi_\omega(\delta) = \omega$. Hence we get $\omega^{f(\delta^{h^{-1}})} = \omega$ for all $\omega\in \Omega$. Since $G$ is faithful on $\Omega$, we get $f(\delta^{h^{-1}}) = 1$ for each $\delta\in \Delta$, which means that $f$ is the trivial map. So $\varphi(\delta^{h^{-1}}) = \varphi(\delta)$ for all $\delta\in \Delta$. Suppose for the sake of contradiction that $\delta^{h^{-1}}\neq \delta$ for some $\delta\in\Delta$. Let $\omega_1,\omega_2\in \Omega$ be two distinct elements. We can consider a function $\phi:\Delta\to \Omega$ such that $\phi(\delta) = \omega_1$, $\phi(\delta^{h^{-1}}) = \omega_2$. This contradicts to the fact that $\phi(\delta) = \phi(\delta^{h^{-1}})$. Hence $H$ is faithful on $\Delta$.
\end{sketch}
\begin{remark}
The result is not true if $|\Omega| = 1$. The groups $G=\{1\}$ and $H = S_2$ provide a counterexample.
\end{remark}

\begin{lemma} \label{lemma-not-regular-iff-stab-self-normalizing}
	Let $G$ be a primitive group on a set $\Omega$ with $|\Omega|\geq 2$. Then either
	\begin{enumerate}[(i)]
		\item all the point stabilizers are self-normalizing; or
		\item the point stabilizers $G_\omega$ are all equal and $G_\omega$ is normal in $G$. Moreover, $|G/G_\omega|$ is a prime.
	\end{enumerate}
\end{lemma}
\begin{sketch}
	By Corollary \ref{cor-prim-iff-stab-is-max}, all point stabilizers are maximal. Note that $N_G(G_\omega) \supseteq G_\omega$ for all $\omega\in \Omega$. So for each $\omega\in \Omega$, we have either $N_G(G_\omega) = G_\omega$ or $N_G(G_\omega) = G$. Now we argue in two cases.
	
\textit{Case 1.} $N_G(G_\omega) = G_\omega$ for some $\omega\in\Omega$. Then for each $x\in G$, we get
\begin{equation*}
	N_G(G_{\omega^x}) = N_G(x^{-1}G_{\omega}x) = x^{-1}N_G(G_{\omega})x = x^{-1}G_{\omega}x =G_{\omega^x}.
\end{equation*}
Since $G$ is transitive, we obtain (i).

\textit{Case 2.} $N_G(G_\omega) = G$ for some $\omega\in G$. By definition, $G_\omega$ is normal in $G$.  Note that $G_{\omega^x} = x^{-1}G_\omega x = G_\omega$ for all $x\in G$. Since $G$ is transitive, the point stabilizers are all equal. Now we prove the last assertion. Since $G_\omega$ is maximal, by Correspondence Theorem, the only subgroups of the quotient group $G/G_\omega$ are $1$ and $G/G_\omega$ itself. Since $G$ is transitive, we know that $G\neq G_\omega$. So $G/G_\omega$ is a nontrivial group which has no proper nontrivial subgroup. It only happens when $G/G_\omega$ is  of prime order. This proves (ii).
\end{sketch}

\begin{proposition}
	Let $G$ and $H$ be nontrivial groups acting faithfully on $\Omega$ and $\Delta$ respectively. Then $G\Wr_{\Delta} H$ is primitive on $\Omega^\Delta$ if and only if the following hold.
\begin{enumerate}[(i)]
	\item $G$ is primitive and the point stabilizers are not normal in $G$.
	\item $\Delta$ is finite and $H$ is transitive on $\Delta$.
\end{enumerate}
\end{proposition}
\begin{sketch}
	Let $W\coloneq G\Wr_{\Delta} H$.  Fix an element $\omega\in\Omega$. Define $\phi_\omega\in \Omega^\Delta$ by $\phi_\omega(\delta) \coloneq \omega$ for all $\delta\in \Delta$. Then we can verify that the stabilizer of $\phi_{\omega}$ is 
	\begin{equation*}
		W_{\phi_{\omega}} = \{(f,h)\in W\mid f(\delta)\in G_\omega \text{ for all }\delta\in\Delta\}.
	\end{equation*}
	Suppose that $W$ is primitive on $\Omega^\Delta$. In view of Corollary \ref{cor-prim-iff-stab-is-max}, $W_{\phi_{\omega}}$ is a maximal subgroup.
Now we show (i) and (ii). The main idea of the proof is to construct a subgroup $S$ so that $W_{\phi_\omega}< S <W$. Such a construction contradicts the maximality of $W_{\phi_\omega}$. Suppose that $G$ is not primitive on $\Omega$. We argue in two cases:
\begin{enumerate}[(1)]
	\item If $G$ is transitive, then, by Corollary \ref{cor-prim-iff-stab-is-max}, there exists a subgroup $R$ with $G_\omega< R < G$ for some $\omega\in\Omega$. Note that 
	\begin{equation*}
		W_{\phi_{\omega}} < \{(f,h)\in W\mid f(\delta)\in R \text{ for all }\delta\in\Delta\} < W.
	\end{equation*}
	\item If $G$ is not transitive, then take an orbit $\Omega'\subsetneq \Omega$. We claim that $W$ is intransitive on $\Omega^\Delta$. Consider a  function $f\in\Omega^{\Delta}$ such that $f(\Delta)\subseteq \Omega'$. Then $f$ cannot be acted by $W$ to a function whose values lie in $\Omega\setminus \Omega'$. Therefore $W$ is intransitive on $\Omega^\Delta$, and hence imprimitive, a contradiction.
\end{enumerate}
Thus $G$ is primitive on $\Omega$. Assume that $G_\omega$ is normal in $G$, then $G_\omega = 1$ by Corollary \ref{cor-normal-subgrp-of-prim-grp}.  Hence the set
\begin{equation*}
	D\coloneq\{(f,1)\in W \mid \text{$f$ is a constant function}\}
\end{equation*}
is normalized by $\overline{H} \coloneq \{(1,h)\in W\mid h\in H\}$ and $W_{\phi_\omega} < D\overline{H} < W$. This completes the proof of (i).

For (ii), if $\Delta$ is infinite, then the set
\begin{equation*}
	B_0 \coloneq \{ (f,1)\in W \mid \text{$f$ has finite support}\}
\end{equation*}
is a normal subgroup of $W$ and satisfies $W_{\phi_\omega}< W_{\phi_\omega} B_0 < W$. If $H$ is intransitive on $\Delta$. Let $\Delta'\subseteq \Delta$ be an orbit of $H$. Note that 
\begin{equation*}
	W_{\phi_\omega} < \{(f,h)\in W\mid f(\delta)\in G_\omega \text{ for all }\delta\in\Delta'\} < W.
\end{equation*}

Conversely, suppose that (i) and (ii) hold. Since $G$ is transitive, $W$ is transitive by Proposition \ref{prop-product-action-transitive}. In view of Corollary \ref{cor-prim-iff-stab-is-max}, it suffices to show that if $S$ properly contains $W_{\phi_\omega}$, then $S = W$. Let $B = G^\Delta$. Note that $W = \overline{B} \overline{H} = \overline{B} W_{\phi_\omega}$, where $\overline{B}  \coloneq \{(f,1)\mid f\in B\}$. Then we can use Dedekind Law to show that $S = SL\cap W = (S\cap B)L$. So we must have $S\cap B > L\cap B$, otherwise $S = (L\cap B)L = L$. Let $(f_0,1)\in S\cap B$ be such that $f_0(\delta_0)\not\in G_\omega$ for some $\delta_0\in\Delta$. By Lemma \ref{lemma-not-regular-iff-stab-self-normalizing}, $G_\omega$ is self-normalizing. Since $f_0(\delta_0)\not\in G_\omega = N_G(G_\omega)$, the element $f_0(\delta_0)$ does not normalize $G_\omega$, and so $f_0(\delta_0)^{-1}g_0 f_0(\delta_0) \not\in G_\omega$ for some $g_0\in G_\omega$. Let $\phi:\Delta\to G$ be defined by $\phi(\delta_0) = g_0$ and $\phi(\delta) = 1$ for all $\delta\in\Delta\setminus\{\delta_0\}$.  Let the function $\psi:\Delta\to G$ be defined by $\psi(\delta) = [\phi(\delta),f_0(\delta)] = \phi(\delta)^{-1}f_0(\delta)^{-1}\phi(\delta)f_0(\delta)$. Then  $\psi(\delta_0)\not\in G_\omega$ and $\psi(\delta) = 1$ for all $\delta\in\Delta\setminus\{\delta_0\}$.  Since $G$ is primitive, $G_\omega$ is maximal by Corollary \ref{cor-prim-iff-stab-is-max}. So $G = \langle G_\omega, \psi(\delta_0)\rangle$. For each $\delta'\in \Delta$, let 
\begin{equation*}
	B_{\delta'} \coloneq \{(f,1)\in W\mid f(\delta) = 1\text{ for all }\delta\in\Delta\setminus\{\delta'\}\}.
\end{equation*}
Note that $(\psi,1)\in S\cap B$. We show that $S$ contains $B_{\delta_0}$. Let $(f,1)\in B_{\delta_0}$ and let $g\coloneq f(\delta_0)$. Since $g\in \langle G_\omega, \psi(\delta_0)\rangle$, it can be written as a product of $g_1,\dots, g_n$ and $ \psi(\delta_0)$. So $(f,1)$ is a product of $(f_{g_1},1),\dots, (f_{g_n},1)$ and $(\psi,1)$ in $S\cap B$, where $f_{g_i}(\delta_0) = g_i$ for all $i$. Hence $B_{\delta_0}$ is contained in $S$. 

Next, we can check that $$(1,x)^{-1}B_{\delta'} (1,x) = B_{\delta'^{x}}$$ for all $(1,x)\in W$. To see this, recall that $(1,x)^{-1}(f,1)(1,x) = (f^x,1)$ and $f^x(\delta) = f(\delta^{x^{-1}})$. Since $\overline{H}\leq S$ and $H$ is transitive on $\Delta$, we obtain $B_{\delta}\subseteq S$ for all $\delta\in \Delta$. Since $\Delta$ is finite, we have $$B = \prod_{\delta\in\Delta} B_{\delta}.$$ Hence $B\leq S$. Therefore $S\geq B\bar{H} = W$. This completes the proof.
\end{sketch}

\paragraph{Main References.} \cite{Meldrum1995,Praeger2018,Dixon1996}
