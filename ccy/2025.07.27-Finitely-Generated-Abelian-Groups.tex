
\section{Structure Theorem for Finitely Generated Abelian Groups}
Throughout this section we write abelian groups additively.
\subsection{Free Abelian Groups}


\begin{definition} \label{def-free-abelian}
	A \textbf{basis} of an abelian group $F$ is a nonempty subset $X$ of $F$ such that $F = \langle X \rangle$ and $X$ is \textbf{linearly independent}, i.e., for distinct $x_1, x_2, \dots, x_k \in X$ and $n_i \in \mathbb{Z}$,
	\begin{equation*}
		n_1x_1 + n_2x_2 + \dots + n_kx_k = 0 \implies n_i = 0 \text{ for every } i.
	\end{equation*}
	The abelian group $F$ is said to be \textbf{free} on the set $X$ if $X$ is a basis of $F$. 
\end{definition}
\begin{remark}
	Depend on the context, the trivial group is considered as a free abelian group on the empty set.
\end{remark}
\begin{lemma} \label{lemma-basis-for-sum-Z}
	Let $I$ be an index set. For each $j \in I$, let $\theta_j$ be the element $(u_i)_{i\in I}$ of $\bigoplus_{i\in I} \mathbb{Z}$, where
	\begin{equation*}
		u_i = \begin{cases}
			1 & \text{if } i = j,\\
			0 & \text{if } i\neq j.
		\end{cases}
	\end{equation*} Then $\{\theta_i \,|\,  i \in I\}$ is a basis of $\bigoplus_{i\in I} \mathbb{Z}$. Furthermore, $\bigoplus_{i\in I} \mathbb{Z}$ is a free object on  $\{\theta_i \,|\, i \in I\}$.
\end{lemma}
\begin{sketch}
	Routine.
\end{sketch}
\begin{proposition} \label{prop-free-abelian-eqv}
	The following conditions on an abelian group $F$ are equivalent.
	\begin{enumerate}[(1)]
		\item $F$ has a basis $X$.
		\item $F$ is the  direct sum of a family of infinite cyclic subgroups.
		\item $F$ is isomorphic to a direct sum of copies of $\mathbb{Z}$.
		\item $F$ is a free object on $X$ in the category of abelian groups.
	\end{enumerate}
\end{proposition}
\begin{sketch}
	(1) $\Rightarrow$ (2) Let $X = \{x_i\,|\, i\in I\}$ be a basis of $F$. Then for each $x \in X$, $nx = 0$ if and only if $n=0$. Hence the subgroup generated by $x_i$ is infinite cyclic. We show that \begin{equation*}
		F = \bigoplus_{i\in I} \langle x_i\rangle.
	\end{equation*} Clearly $F = \sum_{i\in I} \langle x_i\rangle$. If there exists $x_i\in X$ such that $$\langle x_i \rangle \cap \sum_{\substack{j \in I \\ j\neq i}} \langle x_j \rangle \neq \{0\},$$ then for some nonzero $n_i \in \mathbb{Z}$ and $n_{1},\dots, n_{k}\in\mathbb{Z}$ not all of which are zero such that
	$
		nx_i = n_{1} x_{i_1} + \dots + n_k  x_{i_k}
	$ where $x_{i_1}, \dots, x_{i_k}$ are distinct elements of $X$. This contradicts the fact that $X$ is a basis.
	
	(2) $\Rightarrow$ (3) Clearly each infinite cyclic group is isomorphic to $\mathbb{Z}$.
	
	(3) $\Rightarrow$ (1) Suppose $F \cong \bigoplus_{i\in I} \mathbb{Z}$ where the copies of $\mathbb{Z}$ are indexed by a set $I$. By Lemma \ref{lemma-basis-for-sum-Z}, $\{\theta_i \,|\,  i \in I\}$ is a basis of $\bigoplus_{i\in I} \mathbb{Z}$. Then we use the isomorphism $F\cong \bigoplus_{i\in I} \mathbb{Z}$ to obtain a basis of $F$.
	
	(1) $\Rightarrow$ (4) Let $X$ be a basis of $F$ and let $\iota: X \rightarrow F$ be the inclusion map. Suppose we are given a map $f: X \rightarrow G$. If $u \in F = \langle X\rangle$, then 
	\begin{equation*}
	u = n_1 x_1 + \dots + n_k x_k
	\end{equation*} where $n_i \in \mathbb{Z}$ and $x_i \in X$. If $$u = m_1 x_1 + \dots + m_k x_k$$ where $m_i \in \mathbb{Z}$, then $$\sum_{i=1}^k (n_i - m_i) x_i = 0$$ Since $X$ is a basis, $n_i = m_i$ for all $i$. Consequently the map 
	\begin{equation*}
		\bar{f}: F \rightarrow G;\quad \sum_{i=1}^k n_i x_i \mapsto \sum_{i=1}^k n_i f(x_i) 
	\end{equation*}  is a well-defined function such that $\bar{f}\circ \iota = f$. In fact, $\bar{f}$ is a group homomorphism. For the uniqueness, if $g: F \rightarrow G$ is a homomorphism such that $g \iota = f$, then for any $x \in X$, $$g(x) = g(\iota(x)) = f(x) = \bar{f}(x),$$ whence $g(u) = \bar{f}(u)$ for all $u\in F$. Therefore  $F$ is a free object on the set $X$ in the category of abelian groups.
	
	(4) $\Rightarrow$ (3) Suppose that $F$ is free on $X$. Construct the direct sum $\bigoplus_{x\in X} \mathbb{Z}$ with the copies of $\mathbb{Z}$ indexed by $X$. Lemma \ref{lemma-basis-for-sum-Z} shows that  $\bigoplus_{x\in X} \mathbb{Z}$ is a free object on the set $Y = \{\theta_x \,|\, x \in X\}$. Since we have $|X| = |Y|$, it follows that $F \cong \bigoplus_{x\in X} \mathbb{Z}$ by Proposition \ref{prop-free}.
\end{sketch}

\begin{remark}
	From the categorical point of view, one should define a free abelian group using (4). 
\end{remark}


\begin{theorem} \label{thm-free-abelian-same-card}
	Any two bases of a free abelian group $F$ have the same cardinality. 
\end{theorem}

\begin{sketch}
	First suppose $F$ has a basis $X$ of finite cardinality $n$. By Proposition \ref{prop-free-abelian-eqv}, we get $$F \cong \bigoplus_{i=1}^n\mathbb{Z}.$$ Let $p$ be a prime. For any additive group $G$, the set $pG = \{pg \,|\, g \in G\}$ is a subgroup of $G$. Clearly
	\begin{equation*}
		pF \cong \bigoplus_{i=1}^n p\mathbb{Z}.
	\end{equation*} By Proposition \ref{prop-direct-product-normal-subgrp}, we have $$F/pF \cong \bigoplus_{i=1}^n\mathbb{Z}/p\mathbb{Z}  \cong \bigoplus_{i=1}^n \mathbb{Z}_p.$$
	Therefore $|F/pF| = p^n$. Let $Y$ be another basis of $F$ containing at least $r$ elements, then a similar argument shows that $F/pF$ contains a subgroup isomorphic to $\bigoplus_{i=1}^r \mathbb{Z}_p$. Hence we have  $|F/pF| \geq p^r$, whence $p^r \le p^n$ and $r \le n$. It follows that $Y$ contains at most $n$ elements. Let $|Y| = m\leq n$. Then $|F/pF| = p^m$. Therefore $p^m = p^n$ and $|X| = n = m = |Y|$.
	
	By the previous paragraph, the existence of a finite basis would imply every basis is finite. So if one basis of $F$ is infinite, then all bases are infinite. Let $X$ be an infinite basis of $F$. It suffices to show that $|X|=|F|$. Clearly $|X| \le |F|$. Let $ \mathcal{P}_0(X)$ be the set of all finite subsets of $X$.  For each $S=\{x_1, \dots, x_n\} \in \mathcal{P}_0(X)$, let $G_S$ be the subgroup of $F$ generated by $ x_1, \dots, x_n$. Every element in $F$ can be expressed as a linear combination of a finite subset of the basis $X$, and hence belongs to some $G_S$. So 
	\begin{equation*}
		F = \bigcup_{S \in \mathcal{P}_0(X)} G_S.
	\end{equation*}
	Also, we have 
	\begin{equation*}
	G_S \cong \langle x_1 \rangle \oplus \dots \oplus \langle x_n \rangle
	\end{equation*} for each $S = \{x_1,\dots, x_n\}$. Therefore
	\begin{equation*}
		|G_S| = |\langle x_1 \rangle \oplus \dots \oplus \langle x_n \rangle| = \left|\,\bigoplus_{i=1}^n \mathbb{Z}\,\right|= |\mathbb{Z}|^n = |\mathbb{Z}| = \aleph_0
	\end{equation*} by Theorem \ref{thm-cardinal-aleph_0}.(ii).  Hence  we obtain
	\begin{equation*}
		|F| = \left|\,\bigcup_{S \in \mathcal{P}_0(X)} G_S\,\right| \le |\mathcal{P}_0(X)|\aleph_0.
	\end{equation*} by Theorem \ref{thm-cardinal-union}. By Theorem \ref{thm-cardinal-finite-subsets}, we have $|\mathcal{P}_0(X)| = |X|$. By Theorems \ref{thm-cardinal-sum-prod} and \ref{thm-cardinal-weird}.(iii), we get  $$|F| \le  \max\{|X|,\aleph_0\} =   |X|.$$ Therefore $|F| = |X|$ by Theorem \ref{thm-cardinal-Schroder-Bernstein}. 
\end{sketch}
\begin{remark}
	The proof was adapted from Hungerford's Algebra. I feel that $S$ in the original proof is a  bit redundant. Prove me wrawwwwwng.
\end{remark}

\begin{definition}
	The number of elements in a basis of $G$ will be called
	the \textbf{rank} of $G$.
\end{definition}
\begin{theorem}  \label{thm-free-abelian-subgrp}
If $G$ is a subgroup of a free abelian group $F$, then $G$ is a free abelian group. Moreover, $\Rank  G \leq \Rank  F$.
\end{theorem}
\begin{proof}[\textbf{Proof using Zorn's Lemma}]
	Let $\{x_i\}_{i \in I}$ be a basis for $F$. Consider the set $\mathcal{S}$ of all linearly independent sets $Y_J$ in $G$, where $J$ is a subset of the index set $I$, and  the subgroup  generated by $Y_J$ is equal to $G \cap F(Y_J)$, where $F(Y_J)$ is the subgroup of $F$ generated by those $x_i$'s that appear in the expansions of the $y_j$'s.
	
	Consider the partial order on $\mathcal{S}$ by usual set inclusion. Every chain in $\mathcal{S}$ has an upper bound (given by the union of the sets in the chain). By Zorn's Lemma (Lemma \ref{lemma-Zorn}), there exists a maximal element $Y_{J_0}$.  We claim that $Y_{J_0}$ is a basis for $G$. Since $Y_{J_0}$ is linearly independent by definition, it suffices to show that $G = \langle Y_{J_0}\rangle$.
	
	Suppose on the contrary that $G \neq  \langle Y_{J_0}\rangle$. This means $\langle Y_{J_0}\rangle$ is a proper subgroup of $G$. Let $z \in G \setminus \langle Y_{J_0}\rangle$. The expansion of $z$ in terms of the basis $\{x_i\}_{i\in I}$ must have some elements $x_k$'s such that they not involved in the expansions of the $y_j$'s. For convenience, we call such $x_k$ a ``new" element in $z$. For each element $z\in G \setminus \langle Y_{J_0}\rangle$, there is a non-empty set of new  $x_k$'s. We can choose an element $z$ that involves the smallest number of new $x_k$'s. Let these new elements be $x_{k_1},\dots, x_{k_\ell}$. The set of all integers $n\in\mathbb{Z}$ such that there exists $y \in F(Y_{J_0}\cup \{x_{k_2},\dots, x_{k_\ell}\})$ for which $$nx_{k_1} + y\in G$$
	is a nonzero ideal of $\mathbb{Z}$. Since $\mathbb{Z}$ is a principal ideal domain, we can find a generator $\alpha$ for this ideal. Hence we pick an element $z$ so that the coefficient of $x_{k_1}$ is $\alpha$. More precisely, we write
	\begin{equation*}
		z = \alpha x_{k_1} + d_2 x_{k_2} + \cdots + d_\ell x_{k_{\ell}} + y
	\end{equation*}
	for some $d_2,\dots, d_\ell\in \mathbb{Z}$ and $y\in F(Y_{J_0})$.
	
	Let $Y' = Y_{J_0} \cup \{z\}$. Now we show that $Y'$ would contradicts the maximality of $Y_{J_0}$.  First we show that $Y'$ is linearly independent. Suppose that
	\begin{equation*}
		\sum_{j=1}^k n_j y_j + nz = 0.
	\end{equation*}
	where $y_1,\dots, y_k$ are distinct elements in $Y_{J_0}$ and $n_1,\dots, n_k\in \mathbb{Z}$. Remark that the case where linear combinations of elements without $z$ is trivial, and so we omit the argument. If $n\neq 0$, then we have $nz  \in \langle Y_{J_0} \rangle$. Note that the coefficient of $x_{k_1}$ in the expansion of $nz$ is $n\alpha$. Since $nz$ is an element of $\langle Y_{J_0} \rangle$. By definition of new elements, the coefficient of $x_{k_1}$ must be zero. Thus we get $n\alpha = 0$, a contradiction since $\alpha$ is also nonzero. This implies $n = 0$. Then $\sum_{j\in J_0} n_j y_j = 0$. Since $Y_{J_0}$ is independent, $n_j=0$ for all $j$.
	
	Clearly $\langle Y' \rangle \subseteq G \cap F(Y')$. For the reverse direction, let  $w\in G\cap F(Y')$. Then $$w = c_1 x_{k_1} + \cdots +c_\ell x_{k_{\ell}} + y'$$
	for some $c_1,\dots, c_\ell\in\mathbb{Z}$ and $y'\in F(Y_{J_0})$. By definition of $\alpha$, we get $c_1 = \lambda \alpha$ for some $\lambda$. Therefore we have
	\begin{equation*}
		w-\lambda z = (c_2-\lambda d_2) x_{k_2} + \cdots + (c_\ell- \lambda d_\ell) x_{k_{\ell}} + y' -\lambda y.
	\end{equation*}
	The coefficient of $x_{k_1}$ in $w - \lambda z$ is zero. So the number of new elements in $w-\lambda z$ is strictly less than that in $z$. Since $z$ has the smallest number of new elements, it follows that $w-\lambda z$ contains no new element, and so $w-\lambda z\in \langle Y_{J_0} \rangle$. This implies that $w  \in \langle Y_{J_0}\cup \{z\} \rangle = \langle Y' \rangle$. Clearly $Y' \supset Y_{J_0}$, which arrives at a contradiction. Hence $G =  \langle Y_{J_0}\rangle$. Therefore $G$ is a free abelian group.
	
	By the argument above, we have found a basis of $G$ indexed by the subset $J_0$ of the index set $I$. Hence $\Rank  G = |J_0|\leq |I| \leq \Rank  F$. This completes the proof.
\end{proof}
\begin{proof}[\textbf{Proof using Well-Ordering Theorem}]
	Let $\{x_i\}_{i\in I}$ be a basis of $F$. In view of Well Ordering Theorem (Theorem \ref{thm-well-ordering}),  we may assume that $I$ is well-ordered. Let $g$ be any nonzero element in $F$. Then $g$ has a unique expression of the form
	\begin{equation*}
		g = c_1 x_{\alpha_1} + c_2 x_{\alpha_2} + \dots + c_r x_{\alpha_r}
	\end{equation*}
	where $c_1, \ldots, c_r\in\mathbb{Z}$  and $\alpha_1, \ldots, \alpha_r$ are members of $I$. Note that $\{\alpha_1, \ldots, \alpha_r\}$ is a finite set, and thus contains a largest element according to the ordering of $I$. We call this largest element the \textbf{index} of $g$.
	
	We proceed to the proof that $G$ is free. For each $\beta\in I$, let $\mathbb{Z}_\beta$ be the subset of $\mathbb{Z}$ consisting of $0$ and all coefficients of $x_\beta$ that occur in elements of index $\beta$ in $G$.   We claim that $\mathbb{Z}_\beta$ is a subgroup of $\mathbb{Z}$. If $m, n \in \mathbb{Z}_\beta$, then we have
	\begin{align*}
		g_1 &= m x_\beta + \cdots, 
		\\
		g_2 &= n x_\beta + \cdots
	\end{align*}
	for suitable elements $g_1, g_2$ in $G$. If $m =n$, then $m - n = 0\in \mathbb{Z}_\beta$.  For $m \ne n$, the equation
	\begin{equation*}
		g_1 - g_2 = (m - n)u_\beta + \cdots
	\end{equation*}
	proves that $m - n \in \mathbb{Z}_\beta$, because the remaining terms only involve elements $x_\gamma$ with $\gamma < \beta$ (this is where we use the definition of index elements). This proves our claim.
	
	Note that $\mathbb{Z}_\beta$ is cyclic. If $\mathbb{Z}_\beta = \{0\}$, we can ignore the index $\beta$. If $\mathbb{Z}_\beta \ne 0$, then we pick a generator $c_\beta$. By the definition of $\mathbb{Z}_\beta$, there exists an element $y_\beta$ of index $\beta$ in $G$ such that the coefficient of $x_\beta$ in $y_\beta$ is the integer $c_\beta$.  
	
	Finally, we claim that $Y = \{y_\beta\in G\,|\, \mathbb{Z}_\beta \neq 0\}$ is a basis of $G$. Suppose that $$\sum_{i=1}^k n_{i}y_{\alpha_i} = 0$$
	where $y_{\alpha_1}, \dots, y_{\alpha_k}\in Y$ and $n_1,\dots, n_k\in\mathbb{Z}$. Without loss of generality, assume that $\alpha_1<\alpha_2<\cdots < \alpha_k$. By the construction of $Y$, we see that $y_{\alpha_i}$ and $y_{\alpha_j}$ have different index whenever $i\neq j$. We express each $y_{\alpha_i}$ in terms of $x_i$'s. Then the coefficient of $x_{\alpha_k}$ in the equation above is $n_kc_{\alpha_k}$. Since $x_i$'s are linearly independent, we get $nc_{\alpha_k} = 0$ and thus $n_k = 0$. Now the equation becomes $\sum_{i=1}^{k-1} n_{i}y_{\alpha_i} = 0$. Repeating this argument, we obtain $n_i = 0$ for all $i$.  Hence $Y$ is linearly independent. Suppose that  $\langle Y\rangle \ne G$. Among all the elements of $G$ that are not in $\langle Y\rangle$ we can pick one of minimal index (since $I$ is well-ordered), let say $g$ is that element and that $\gamma$ is its index. Write
	\[ g = d x_\gamma + \cdots. \]
	Now $d$ is an element of $Z_\gamma$ and therefore is a multiple of the generator $c_{\gamma}$ of   $Z_\gamma$. Let 
	\begin{equation*}
		g' = g - \frac{d}{c_{\gamma}}y_\gamma.
	\end{equation*}  Since $y_\gamma \in \langle Y\rangle$ and $g\not\in \langle Y\rangle$, we have $g'\not\in \langle Y\rangle$. The coefficient of $x_\gamma$ in the right hand side is $0$, and thus the index of $g'$ is smaller than $\gamma$, a contradiction. Hence $G$ is a free abelian group. 
	
	The second assertion follows immediately since the basis $Y$ constructed above is indexed by a subset of $I$.
\end{proof}
\begin{proof}[\textbf{Proof using Transfinite Induction}]
	Let $X =  \{x_\alpha\,|\,\alpha < \beta\}$ be a basis of $F$ where the index set is assumed to be well-ordered and order-isomorphic to $\beta$ according to Counting Theorem (Theorem \ref{thm-counting}). Thus $F = \bigoplus_{\gamma < \beta} \langle x_\gamma \rangle$. Let $G$ be a given subgroup. For each ordinal $\alpha$, define 
	\begin{align*}
	F_\alpha &= 
	\begin{cases}
		\bigoplus_{\gamma < \alpha} \langle x_\gamma \rangle &\text{if } \alpha<\beta,\\
		F &\text{otherwise}. 
	\end{cases}
	\\
	G_\alpha &= G \cap F_\alpha.
	\end{align*}
	We use transfinite induction (Theorem \ref{thm-trans-induction}) to show that $G_\alpha$ is free for all ordinals $\alpha$. With this result,  we can take an ordinal $\alpha\geq \beta$ to obtain that $G = G_{\alpha}$ is free.
	
	For $\alpha=0$, the set $\{\gamma\,|\,\gamma < \alpha\}$ is empty. The direct sum over an empty set is the trivial group. So $G_0 = \{0\}$ is a free abelian group. (Let's assume in this case the trivial group is free, see the remark below Definition \ref{def-free-abelian}).
	
	Assume that the result holds  for a given ordinal $\alpha$. The case $\alpha\geq \beta$ is trivial since $G_{\alpha+1} = G\cap F_{\alpha+1} = G\cap F = G\cap F_{\alpha} = G_{\alpha}$. Suppose that $\alpha < \beta$. Clearly, $G_\alpha = G_{\alpha+1} \cap F_\alpha$. By the Second Isomorphism Theorem, we have 
	\begin{equation*}
		\frac{G_{\alpha+1}}{G_\alpha} = \frac{G_{\alpha+1}}{G_{\alpha+1} \cap F_\alpha} \cong \frac{G_{\alpha+1} + F_\alpha}{F_\alpha}.
	\end{equation*}  The last quotient group is a subgroup of $F_{\alpha+1}/F_\alpha \cong \langle x_\alpha \rangle$, thus either $G_{\alpha+1} = G_\alpha$ or $G_{\alpha+1}/G_\alpha$ is an infinite cyclic group (which is free). If $G_{\alpha+1} = G_\alpha$, then we are done. If $G_{\alpha+1}/G_\alpha$ is free, then   we conclude from Lemma \ref{lemma-FGAG} that $G_{\alpha+1} = G_\alpha \oplus \langle g_\alpha \rangle$ for some nonzero $g_\alpha \in G_{\alpha+1}$. By inductive hypothesis, $G_{\alpha}$ is free. So the union of a basis of $G_{\alpha}$ and $\{g_{\alpha}\}$ is a basis of $G_{\alpha+1}$. Hence $G_{\alpha+1}$ is free.
	
  Suppose the result holds all $\alpha < \lambda$, where $\lambda$ is a nonzero limit ordinal. 
Note that 
$$F_\lambda = \bigcup_{\alpha < \lambda} F_\alpha.$$ 
Thus 
$$G_\lambda = G \cap F_\lambda = G \cap\bigcup_{\alpha < \lambda} F_\alpha = \bigcup_{\alpha < \lambda} G_\alpha.$$
For each $\alpha < \lambda$, let $Y_\alpha$ be the basis for $G_\alpha$ constructed in the successor step, i.e., $Y_{\alpha} = \{g_{\gamma}\,|\, \gamma<\alpha, \, g_{\gamma}\neq 0\}$. Let $Y = \bigcup_{\alpha < \lambda} Y_\alpha$. We claim $Y$ is a basis for $G_\lambda$. Clearly $G_{\lambda} = \langle Y\rangle$ since every $g$ in $G_{\lambda}$ belongs in some $G_\alpha$, so $g$ is a finite linear combination of elements in $Y_{\alpha}$. Suppose that $\sum_{i=1}^k n_i y_i = 0$ where $y_i \in Y$ are distinct elements and $n_i\in\mathbb{Z}$. By the assumption on $Y_{\alpha}$, each $y_i$ can be identified as $g_{\alpha_i}$ for some index $\alpha_i < \lambda$ (Note that $\alpha_i$ is a successor ordinal because $g_{\mu}$ does not make sense in the construction above). Assume that $\alpha_1<\dots< \alpha_k$. Then the  elements $y_1, \dots, y_k$ belong to  $Y_{\alpha_k}$. Since $Y_{\alpha_k}$ is a basis, we get $n_i = 0$ for all $i$. Therefore $G_\lambda$ is a free abelian group.

In the successor step, we construct the basis for $G_{\alpha}$ by at most adding one new basis element $g_\alpha$ corresponding to the basis element $x_\alpha$ of $F$. In the limit ordinal step, we do not add new basis element. Hence $\Rank G_{\alpha} \leq\Rank  F$ for all ordinals $\alpha$.
\end{proof}

	%We argue in two cases either $F$ is finitely generated or not. Suppose that $F$ is generated by $n$ elements. Let  $\{x_1, \dots, x_n\}$ be the basis of $F$.  We prove by induction on $n$. The case $n=1$ is trivial. Let $n\geq 2$ and the result holds for $n-1$. Let $G$ be a subgroup of $F$. By Proposition \ref{prop-free-abelian-eqv}, we have \[F = \langle x_1 \rangle \oplus \dots \oplus \langle x_n \rangle.\]Let $f: F \to \langle x_1 \rangle$ be the projection. Let $G_1$ be the kernel of $f|_G$. Then $G_1$ is a subgroup of the free subgroup $\langle x_2, \dots, x_n \rangle$. By the inductive hypothesis, $G_1$ is free and has a basis with at most $n-1$ elements. By Lemma \ref{lemma-FGAG}, there exists a subgroup $H_1$ isomorphic to the image of $f|_G$ (which is a subgroup of $\langle  x_1 \rangle$) such that \[G = G_1 \oplus H_1.\]Since $f(G)$ is either $0$ or infinite cyclic, i.e. free on one generator, this proves that $H_1$ is free and hence $G$ is free with a basis $X$ having at most $n$ elements.
	
	%Let $G$ be a subgroup of $F$. Let $S$ be the set of all subsets of $G$ that are linearly independent. Clearly every chain in $S$ has an upper bound in $S$. By Zorn's Lemma (Lemma \ref{lemma-Zorn}), it has a maximal linearly independent subset $X$ of $G$. Now we claim that $G = \langle X\rangle$.  If $X$ does not generate $G$, then there is an element $g\in G$ such that $g\not\in \langle X\rangle$. Then $X\cup\{g\}$ is a linearly independent set, contradicts to the maximality of $X$. Therefore $X$ is a basis of $G$ and thus $G$ is free. We proceed to show that $|X|$ is less than the cardinality of a basis of $F$. Let $T$ be the set of all subsets containing $X$ that are linearly independent. Then we can apply Zorn's Lemma again to obtain a basis $Y$ of $F$. By definition, we have $X\subseteq Y$. Hence $|X|\leq |Y|$, as claimed.

\begin{corollary} \label{cor-free-abelian-isom}
	Let $F_1$ be the free abelian group on the set $X_1$ and $F_2$ the free abelian group on the set $X_2$. Then $F_1 \cong F_2$ if and only if $F_1$ and $F_2$ have the same rank.
\end{corollary}
\begin{sketch}
	Let  $\alpha : F_1\rightarrow F_2$ be an isomorphism. Then $\alpha(X_1)$ is a basis of $F_2$. By Theorem \ref{thm-free-abelian-same-card}, we have $|X_1| = |\alpha(X_1)| = |X_2|$. The converse follows from Proposition \ref{prop-free}. 
\end{sketch}

\begin{corollary} \label{cor-free-abelian-grp-homomorphic-image}
	Every abelian group $G$ is the homomorphic image of a free abelian group of rank $|X|$, where $X$ is a set of generators of $G$.
\end{corollary}
\begin{sketch}
	Let $F$ be the free abelian group on the set $X$. Then $F$ is of rank $|X|$. The inclusion map $f:X \to G$ induces a homomorphism $\bar{f} \colon F \to G$ such that $\bar{f}\circ\iota = f$. In particular, $\bar{f}(x) = f(x) = x$ for all $x\in X$, whence $X \subseteq \Im \bar{f}$. Since $X$ generates $G$ we must have $\Im \bar{f} = G$.
\end{sketch}

\begin{corollary} \label{cor-subgrp-FGAG-is-FG}
	Every subgroup of finitely generated abelian group is finitely generated.
\end{corollary}
\begin{sketch}
	Let $G$ be finitely generated by $n$ elements. By Corollary \ref{cor-free-abelian-grp-homomorphic-image},  we can find a free abelian group $F$ on $n$ generators and a surjective homomorphism
	$\varphi: F \to G $. 
	
	Let $H$ be a subgroup of $G$. The subgroup $\varphi^{-1}(H)$ of $F$ is finitely generated by Theorem \ref{thm-free-abelian-subgrp}. Hence $H$ itself is finitely generated.
\end{sketch}

\subsection{Structure Theorem}
\begin{definition}
	Let $G$ be an abelian group. An element $x \in G$ is said to be \textbf{torsion} if it has finite order. The subset $G_t$ of all torsion elements of $G$ is a subgroup of $G$ called the \textbf{torsion subgroup} of $G$.  An abelian group is said to be \textbf{torsion free} if the only torsion element is the identity.
\end{definition}

\begin{lemma} \label{lemma-FGAG-torsion-free}
	Let $G$ be a finitely generated torsion-free abelian group. Then $G$ is a free abelian group of finite rank.
\end{lemma}
\begin{sketch}
	Assume $G$ is not a trivial group. Since $G$ is finitely generated, there exists a finite subset $S$ of $G$ for which $G = \langle S\rangle$. Let $X = \{x_1, \dots, x_n\}$ be a maximal subset of $S$ having the property that $X$ is linearly independent. Note that $n \ge 1$ since $G \neq 0$. Let $F$ be the subgroup generated by $x_1, \dots, x_n$. Then $F$ is free. Given $y \in G$ there exist integers $m_1, \dots, m_n, m$ not all zero such that
	$$ my + m_1 x_1 + \dots + m_n x_n = 0, $$
	by the assumption of maximality on $X$. Furthermore, $m \neq 0$; otherwise all $m_i = 0$. Hence $my$ lies in $F$. This is true for every $y\in S$, whence there exists an integer $m \neq 0$ such that $mG = m\langle S\rangle \subseteq F$ (by considering the least common multiple). The map
	$ x \mapsto mx $
	of $G$ into itself is a homomorphism, having trivial kernel since $G$ is torsion free. Hence it is an isomorphism of $G$ onto $mG$. Since $mG$ is a subgroup of $F$, it follows from  Theorem \ref{thm-free-abelian-subgrp} that   $mG$ is a free abelian group of finite rank. This completes the proof.
\end{sketch}

\begin{lemma} \label{lemma-FGAG}
	Let $\varphi:G \to F$ be a surjective homomorphism of abelian groups, where $F$ is free.  Then there exists a subgroup $H$ of $G$ such that the restriction of $\varphi$ to $H$ induces an isomorphism of $H$ with $F$ , i.e., $H\cong F$, and $$G = \Ker \varphi \oplus H.$$
\end{lemma}
\begin{sketch}
	Let $\{x_i'\}_{i \in I}$ be a basis of $F$, and for each $i \in I$, let $x_i$ be an element of $G$ such that $\varphi(x_i) = x_i'$. Let $H$ be the subgroup of $G$ generated by all elements $x_i, i \in I$, i.e., \begin{equation*}
		H = \langle x_i \,|\, i\in I \rangle.
	\end{equation*} We claim that $\{x_i\}_{i\in I}$ is a basis of $H$. Suppose that
	\begin{equation*}
		\sum_{i \in I} n_i x_i = 0
	\end{equation*}
	with $n_i\in\mathbb{Z}$ and $n_i = 0$ for almost all $i$. Applying $\varphi$ yields
	\begin{equation*}
		0 = \sum_{i \in I} n_i \varphi(x_i) = \sum_{i \in I} n_i x_i'.
	\end{equation*}
	Since $\{x_i'\}_{i \in I}$ is a basis for $F$, we get $n_i = 0$ for all $i$. Hence  $\{x_i\}_{i \in I}$ is a basis of $H$. 
	
	Now we show that $G$ is an internal direct sum of $\Ker \varphi$ and $H$. Let $z \in \Ker \varphi \cap H$. Then $z = \sum_{i\in I}n_ix_i$ and $\varphi(z) = 0$ for some appropriate $n_i\in \mathbb{Z}$. Similar argument as above shows that $\Ker \varphi \cap H = 0$. Let $x \in G$. Since $\varphi(x) \in F$ we can find some appropriate integers $n_i$ such that
	\begin{equation*}
		\varphi(x) = \sum_{i \in I} n_i x_i'.
	\end{equation*}
	From this, we obtain
	\begin{equation*}
		0=\varphi(x) -  \sum_{i \in I} n_i x_i' = \varphi(x) -  \sum_{i \in I} n_i \varphi(x_i) = \varphi\left(x - \sum_{i \in I} n_i x_i\right).
	\end{equation*}
	Hence $x - \sum_{i \in I} n_i x_i \in \Ker\varphi$, whence
	\begin{equation*}
		x = x -  \sum_{i \in I} n_i x_i +  \sum_{i \in I} n_i x_i \in \Ker \varphi + H.
	\end{equation*}  So the lemma follows.
\end{sketch}

\begin{theorem} \label{thm-FGAG-1}
	If $G$ is a finitely generated abelian group, then $G_t$ is finite and $G = G_t\oplus F$, where $F$ is a free abelian group of finite rank and $F\cong G/G_t$.
\end{theorem}
\begin{sketch}
By Corollary \ref{cor-subgrp-FGAG-is-FG}, the subgroup $G_t$ is  finitely generated,  and thus finite because each element of $G_t$ has finite order.
	
	Next, we prove that $G/G_{t}$ is torsion free. Let $x+G_t$ be an element of $G/G_{t}$ such that $m(x+G_t) = 0$ for some integer $m \neq 0$. Then $mx \in G_{t}$, whence $qmx = 0$ for some integer $q \neq 0$. Then $x \in G_{t}$, so $x+G_t = G_t$, and $G/G_{t}$ is torsion free. By Lemma \ref{lemma-FGAG-torsion-free}, $G/G_{t}$ is free and has finite rank. By applying   Lemma \ref{lemma-FGAG} to the canonical projection $\pi: G \rightarrow G/G_t$, we obtain
	$
		G = G_t \oplus F
	$
	where $F$ is a subgroup of $G$ such that $F\cong G/G_t$.  By Corollary \ref{cor-free-abelian-isom}, $F$ has finite rank.
\end{sketch}

\begin{lemma}\label{lemma-FGAG-coprime}
	For each positive integer $m$, let $G_m$ be  the subgroup of a group $G$ consisting of elements $x\in G$ such that $mx = 0$. Then for any positive coprime integers $r$ and $s$, 
	\begin{equation*}
		G_{rs} = G_r \oplus G_s.
	\end{equation*}
\end{lemma}
\begin{sketch}
	Clearly $G_r + G_s\subseteq G_{rs}$. For the reverse direction, there exist integers $u, v$ such that $ur + vs = 1$. Let $x\in G_{rs}$. Then $x = urx + vsx$, and $urx \in G_s$ while $vsx \in G_r$, and $G_{rs} = G_r + G_s$.
	
	Let $x\in G_r\cap G_s$. Then $rx = 0 =sx$. So $dx = 0$ where $d = \gcd(r,s)$. Since $\gcd(r,s) = 1$, we get $x = 0$. This proves the assertion.
\end{sketch}
\begin{remark}
	Let $m$ and $n$ be coprime positive integers. If we choose $G = \mathbb{Z}_{mn}$, then $G_{mn} = \mathbb{Z}_{mn}$. To show that $G_m \cong \mathbb{Z}_m$, we note that $G_m$ can be generated by the element $n$ in $\mathbb{Z}_{mn}$, which is of order $m$. Similarly we obtain $G_{n} \cong \mathbb{Z}_n$ and thus
	\begin{equation*}
		\mathbb{Z}_{mn} \cong \mathbb{Z}_{m} \oplus \mathbb{Z}_{n}.
	\end{equation*}
\end{remark}

\begin{theorem} \label{thm-1.3.9}
	Let $G$ be a torsion abelian group. For each prime $p$, let $G(p)$ be the set of elements of $G$ whose order is a power of $p$, i.e., $G(p) = \{x\in G\,|\, |x|=p^n\text{ for some }n\geq 0\}$.
	\begin{enumerate}[(i)]
		\item $G(p)$ is a subgroup of $G$ for each prime $p$. If $G(p)$ is finite, then it is a $p$-group.
		\item (Primary Decomposition) $G =  \bigoplus_{p\text{ is prime}} G(p)$. If $G$ is finitely generated, then only finitely many of the $G(p)$ are nonzero.
	\end{enumerate}
\end{theorem}
\begin{sketch}
	(i) Trivial.
	
	(ii)  We have a homomorphism
	\begin{align*}
		\varphi: \bigoplus_p G(p) &\rightarrow G
		\\
		(x_p) &\mapsto \sum_p x_p.
	\end{align*}
 We prove that this homomorphism is bijective. Let $x = (x_p)\in \Ker \varphi$. Then $\sum_p x_p = 0$. Let $q$ be a prime.  Then
	$$ x_q =  \sum_{p \neq q} (-x_p). $$
	Let $m$ be the least common multiple of the orders of elements $x_p$ on the right-hand side. Then $mx_q = 0$. By definition of $G(q)$, we also have $q^r x_q = 0$ for some integer $r\geq 0$. Let $d$ be the greatest common divisor of $m$ and $q^r$. Then $dx_q = 0$. But $q$ does not divide $m$, so $d=1$ and thus $x_q = 0$. Hence $x_q = 0$ for every prime $q$, which means that the kernel is trivial, and $\varphi$ is injective.
	
	As for the surjectivity, for each positive integer $m\geq 2$, we let $m = p_1^{\ell_1}\cdots p_k^{\ell_k}$ be the prime factorization of $m$.    Repeating Lemma \ref{lemma-FGAG-coprime} inductively, we conclude that every element in $G_m$ can be expressed as a sum of elements in $G_{p_i^{\ell_i}}$ (which is a subgroup of $G(p_i)$), i.e., 
	$$  G_m = G_{p_1^{\ell_1}} \oplus \cdots \oplus G_{p_k^{\ell_k}} =\bigoplus_{i=1}^k G_{p_i^{\ell_i}}. $$
	Since $G = \bigcup_{m\geq 1} G_m$, the map $\varphi$ is surjective.
	
	If $G$ is finitely generated, then $G$ is finite by Theorem \ref{thm-FGAG-1}. Hence $G$ must be a direct sum of a finitely many groups $G(p)$.
\end{sketch}

\begin{lemma} \label{lemma-FGAG-order}
	Let $G$ be an abelian $p$-group. Let $g$ be a nonzero element of $G$. If $p^kg$ is nonzero and has order $p^\ell$, then $g$ has order $p^{k+\ell}$.  
\end{lemma}
\begin{sketch}
	From $p^\ell p^kg$, we have $|g|\leq p^{k+\ell}$. 
	
	Now let $|g| = p^n$ for some integer $n\geq 1$. Then $p^k g\neq 0$ implies that $k< n$. So $n-k > 0$. Since
	\begin{equation*}
		p^{n-k}p^k b  = 0,
	\end{equation*}
	we have $p^\ell \leq p^{n-k}$. Hence $|g|  = p^n \geq p^{k+\ell}$. Therefore $|g|  =  p^{k+\ell}$.
\end{sketch}




\begin{lemma} \label{lemma-FGAG-order-represent}
	Let $G$ be a $p$-group and let $x$ be an element of maximal order in $G$. Let $\bar{g}$ be an element of $G/\langle x \rangle$, of order $p^r$. Then there exists a representative $g$ of $\bar{g}$ in $G$ which also has order $p^r$.
\end{lemma}
\begin{sketch}
	Let $g$ be any representative of $\bar{g}$. Assume that the order of $x$ is $p^\ell$. Then $p^rg+ \langle x \rangle = \langle x \rangle$ and so $p^rg \in \langle x \rangle$. So $$p^r g = nx$$ for some integer $0\leq n < p^{\ell}$. Note that the order of $\bar{g}$ is at most the order of $g$. If $n=0$, then $g$ has order $p^r$ and we are done. Otherwise write $n = p^k m$ where $m$ is coprime to $p$ and $k\geq 0$. Then $m x$ is also a generator of $\langle x \rangle$, and hence has order $|x| = p^{\ell}$.  Now we have $k <  \ell$ since $n<p^\ell$. Then $p^k m x$ has order $p^{\ell-k}$ (recall that $|\langle a^m \rangle| = |a|/\gcd(|a|,m)$). By Lemma \ref{lemma-FGAG-order}, the element $g$ has order $ p^{r+\ell-k} $. Since $x$ has maximal order we have $|x|\geq |g|$, whence $r+\ell-k \le \ell$ and $r \le k$. Let $g'= p^{k-r}mx$. Then   $g' \in \langle x \rangle$ and $p^r g = p^r g'$. Let $g'' = g-g'$. Then $g''+\langle x \rangle = g+\langle x \rangle = \bar{g}$ because $g''-g = -g'\in \langle x \rangle$. It follows that $$|g''|\geq |g''+\langle x \rangle|= |\bar{g}| = p^r.$$ Since $|g''| \le p^r$ by $p^rg'' = 0$,  we conclude that $g''$ has order  $p^r$.
\end{sketch}

\begin{lemma} \label{lemma-FGAG-small-1}
	Let $m$ and $n$ be integers such that $1\leq m<n$. Let $p$ be a prime number. Then $$p^m\mathbb{Z}_{p^n}\cong \mathbb{Z}_{p^{n-m}}.$$
\end{lemma}
\begin{sketch}
	Note that $p^m$ has order $p^{n-m}$. So we have $p^m \mathbb{Z}_{p^n} = \langle p^m \rangle  \cong \mathbb{Z}_{p^{n-m}}$.
\end{sketch}
\begin{lemma} \label{lemma-FGAG-small-2}
	Let $G$ an abelian group and let $m$ be an integer. If $G$ is the direct sum of  subgroups $G_i$ ($i\in I$) then
	\begin{equation*}
		mG = \bigoplus_{i\in I}mG_i.
	\end{equation*}
\end{lemma}
\begin{sketch}
	There is an isomorphism $\varphi: G\rightarrow \bigoplus_{i\in I}G_i$. Let $\psi:mG \rightarrow \bigoplus_{i\in I}G_i$ be defined by $\psi(x) = \varphi(x)$. Clearly $\Ker \psi = \{0\}$ and $\Im \psi = \bigoplus_{i\in I}mG_i$.
\end{sketch}

\begin{theorem} \label{thm-1.3.11}
	Let $G$ be a finite abelian $p$-group. Then $G$ is an (internal) direct sum of cyclic groups of orders $p^{n_1},\dots, p^{n_k}$ respectively, with $n_1\geq n_2\geq\cdots\geq n_k\geq 1$. In particular,
	\begin{equation*}
		G \cong \mathbb{Z}_{p^{n_1}} \oplus \mathbb{Z}_{p^{n_2}} \oplus \cdots \oplus \mathbb{Z}_{p^{n_k}}.
	\end{equation*}
	In this case, we say that $G$ is of type $(p^{n_1},\dots, p^{n_k})$. The integers $n_1,\dots, n_k$ are uniquely determined.
\end{theorem}
\begin{sketch}
 We prove by induction on order of $G$. If $G$ is cyclic, then we are done. Assume that $G$ is not cyclic. Let $x_1 \in G$ be an element of maximal order. Let $G_1$ be the cyclic subgroup generated by $x_1$, of order $p^{r_1}$. 
	
By inductive hypothesis, the quotient group $G/G_1$ is an internal direct sum of cyclic subgroups $\overline{G}_2,\dots, \overline{G}_k$ of orders $p^{n_2}, \dots, p^{n_k}$ respectively with $n_2 \ge \dots \ge n_k$. More precisely, 
	$$ G/G_1 = \overline{G}_2 \oplus \dots \oplus \overline{G}_k \cong \mathbb{Z}_{p^{n_2}}\oplus \cdots \oplus \mathbb{Z}_{p^{n_k}}.$$
Let $\bar{x}_i$ be a generator of $\overline{G}_i$ for $i=2, \dots, k$. Since $\bar{x}_i \in G/G_1$, Lemma \ref{lemma-FGAG-order-represent} guarantees that we can choose  a representative $x_i$  in $G$ such that $|x_i | = |\bar{x}_i|$. Let $G_i$ be the cyclic subgroup generated by $x_i$. We claim that $G = G_1\oplus G_2\oplus \cdots \oplus G_k$.
	
	Given $x \in G$, let $\bar{x} = x + G_1\in G/G_1$. There exist integers $m_2,\dots, m_k$ such that
	$$ \bar{x} = m_2 \bar{x}_2 + \dots + m_k \bar{x}_k. $$
	Hence 
	\begin{equation*}
		x - m_2 x_2 - \dots - m_k x_k \in G_1.
	\end{equation*} So there exists an integer $m_1$ such that
	$$ x = m_1 x_1 + m_2 x_2 + \dots + m_k x_k. $$
	Hence $G_1 + \dots + G_k = G$.
	
To verify the sum is an internal direct sum, it suffices to show that $x_1,\dots, x_k$ are linearly independent, because it would imply that $$ (G_1 + \dots + G_i) \cap G_{i+1} = 0 $$
for each $i = 1,\dots,k-1$. Suppose that 
	$$ m_1 x_1 + \dots + m_k x_k = 0$$
	where $m_1,\dots, m_k$ are integers. Then we have 
	$$m_2 \bar{x}_2 + \dots + m_k \bar{x}_k = G_1.$$
	Since $G/G_1$ is a direct sum of $\overline{G}_2, \dots, \overline{G}_k$,  we conclude from Theorem \ref{thm-internal-direct-prod} that each $m_i=0$ for $i=2, \dots, k$. Hence $m_1 x_1 = 0$ and so   $m_1=0$. Therefore $G$ is the direct sum of $G_1, \dots, G_k$.
	
	We prove uniqueness by induction. Suppose that $G$ is written in two ways as a direct sum of cyclic groups, say of type
	$$ (p^{n_1}, \dots, p^{n_k}) \quad \text{and} \quad (p^{m_1}, \dots, p^{m_s}) $$
	with $r_1 \ge \dots \ge r_k \ge 1$ and $m_1 \ge \dots \ge m_s \ge 1$. In view of Lemmas \ref{lemma-FGAG-small-1} and \ref{lemma-FGAG-small-2}, the subgroup $pG$  is of type
	$$ (p^{r_1-1}, \dots, p^{r_k-1}) \quad \text{and} \quad (p^{m_1-1}, \dots, p^{m_s-1}). $$
By induction, the subsequence of
	$$ (r_1-1, \dots, r_k-1) $$
	consisting of those integers at least $1$ is uniquely determined (those with $0$ correspond to the trivial subgroup), and is the same as the corresponding subsequence of
	$$ (m_1-1, \dots, m_s-1). $$
	In other words, there exists an integer $\ell \geq 1$ such that $r_i-1 = m_i-1\geq 1$ for all $i=1,\dots, \ell$. Hence $r_i = m_i$ for all $i=1,\dots, \ell$, and the two sequences
	$$ (p^{r_1}, \dots, p^{r_k}) \quad \text{and} \quad (p^{m_1}, \dots, p^{m_s}) $$
	can differ only in their last components which can be equal to $p$. Hence $G$ is of type
	$$ (p^{r_1}, \dots, p^{r_\ell}, \underbrace{p, \dots, p}_{k-\ell \text{ times}}) \quad \text{and} \quad (p^{r_1}, \dots, p^{r_\ell}, \underbrace{p, \dots, p}_{s-\ell \text{ times}}). $$
	Thus the order of $G$ is equal to
	$$ p^{r_1 + \dots + r_\ell} p^{k-\ell} = p^{r_1 + \dots + r_\ell} p^{s-\ell}, $$
	whence $k = s$, and our theorem is proved.
\end{sketch}

By Theorems \ref{thm-FGAG-1}, \ref{thm-1.3.9} and \ref{thm-1.3.11}, we have the following decomposition of finitely generated abelian groups.
\begin{theorem} \label{thm-decomposition-of-AG-elem-div}
	A finitely generated abelian group $G$ is the direct sum of a free abelian group $F$ of finite rank and a finite number of
	cyclic groups. The cyclic summands (if any) are of orders $p_1^{s_1},\dots, p_k^{s_k}$ 
	where $p_1, \dots , p_k$ are (not necessarily distinct) prime numbers and $s_1, \dots , s_k$ are (not
	necessarily distinct) positive integers. The rank of $F$ and the prime powers $p_1^{s_1},\dots, p_k^{s_k}$ are uniquely determined by $G$, up to their order). In particular,
	\begin{align*}
		G&= \langle x_1 \rangle\oplus \cdots \oplus \langle x_k \rangle\oplus F\\
		&\cong \mathbb{Z}_{p_1^{s_1}}\oplus \cdots \oplus \mathbb{Z}_{p_k^{s_k}}\oplus \underbrace{\mathbb{Z} \oplus \cdots \oplus \mathbb{Z}}_{\Rank  F \text{ summands}}
	\end{align*}
	where $x_1,\dots, x_k\in G$ with $|x_i| = p_i^{s_i}$ for all $i=1,\dots, k$.
\end{theorem}
\begin{remark}
	To emphasize the ordering of powers of primes, we can express $G$ as
	\begin{equation*}
		\bigoplus_{j=1}^\ell \bigoplus_{i=1}^{k_{j}} \mathbb{Z}_{p_j^{n_{ij}}}
	\end{equation*}
	where $p_1,p_2,\dots, p_\ell$ are distinct primes and $n_{1j}\geq n_{2j}\geq \cdots \geq n_{k_j j}\geq 1$ for all $j$.
\end{remark}


\begin{definition}
	The prime powers $p_1^{s_1},\dots, p_k^{s_k}$ in Theorem \ref{thm-decomposition-of-AG-elem-div} are called the \textbf{elementary divisors} of $G$.
\end{definition}
Since the order of the primary cyclic factors may vary, the decomposition is not entirely unique. To resolve this issue, there is another way to decompose a group without introducing elementary divisors.
\begin{theorem} \label{thm-decomposition-of-modules-inv-fac}
	A finitely generated abelian group $G$ is the direct sum of a free abelian group $F$ of finite rank and a finite number of
	cyclic  groups. The cyclic  summands (if any) are of orders $m_1,\dots, m_r$ where $m_1,\dots, m_r$ are integers greater than $1$ such that $m_1|m_2|\cdots |m_r$. The rank of $F$ and the integers $m_1,\dots, m_r$ are uniquely determined by $G$. In particular,
	\begin{align*}
		G&= \langle x_1 \rangle\oplus \cdots \oplus \langle x_r \rangle\oplus F\\
		&\cong \mathbb{Z}_{m_1}\oplus \cdots \oplus \mathbb{Z}_{m_r}\oplus \underbrace{\mathbb{Z} \oplus \cdots \oplus \mathbb{Z}}_{\Rank  F \text{ summands}}
	\end{align*}
	where $x_1,\dots, x_r\in G$ with $|x_i| = m_i$ for all $i=1,\dots, r$.
\end{theorem}
\begin{sketch}  We arrange the elementary divisors of $G$ and insert $p_i^0 = 1$ as follows.
	$$
	\begin{array}{ccccc}
		p_1^{n_{11}} & p_2^{n_{12}} & \dots & p_\ell^{n_{1\ell}} \smallskip\\
		p_1^{n_{21}} & p_2^{n_{22}} & \dots & p_\ell^{n_{2\ell}} \smallskip\\
		\vdots & \vdots & & \vdots \smallskip\\
		p_1^{n_{k1}} & p_2^{n_{k2}} & \dots & p_\ell^{n_{k\ell}}
	\end{array}
	$$
	where $p_1, \dots, p_r$ are distinct primes, $0 \le n_{1j} \le n_{2j} \le \dots \le n_{tj}$ for each $j=1,2,\dots,\ell$ and $n_{1j} \neq 0$ for some $j$ (the last condition is to ensure that we do not insert excessive $p_i^0$). By Theorem \ref{thm-decomposition-of-AG-elem-div}, we get (after adding trivial groups in the decomposition)
	\begin{equation*}
		G \cong \bigoplus_{j=1}^\ell \bigoplus_{i=1}^{k} \mathbb{Z}_{p_j^{n_{ij}}} \oplus F
	\end{equation*}  where $F$ is a  free abelian group of finite rank. For each $i=1,2,\dots,k$, let $m_i$ be the product of elements in the $i$th row in the array above, i.e., 
	\begin{equation*}
		m_i = p_1^{n_{i1}} p_2^{n_{i2}} \dots p_\ell^{n_{i\ell}}.
	\end{equation*} Since some $n_{1j} \neq 0$ for some $j$, we have $m_1 > 1$. Clearly $m_1 | m_2 | \dots | m_t$. Since $p_1,\dots, p_\ell$ are distinct primes, by Lemma \ref{lemma-FGAG-coprime}, we have
	$$ \bigoplus_{j=1}^\ell \mathbb{Z}_{p_j^{n_{ij}}}  \cong  \mathbb{Z}_{m_i}$$
	for each $i = 1,\dots, k$. Hence we have
	\begin{equation*}
		G \cong \bigoplus_{j=1}^\ell \bigoplus_{i=1}^{k} \mathbb{Z}_{p_j^{n_{ij}}} \oplus F \cong  \bigoplus_{i=1}^{k} \bigoplus_{j=1}^\ell \mathbb{Z}_{p_j^{n_{ij}}} \oplus F \cong \bigoplus_{i=1}^{k} \mathbb{Z}_{m_i} \oplus F.
	\end{equation*}
	
	Now we show the uniqueness. Suppose that \begin{equation*}
		G \cong \bigoplus_{i=1}^{k} \mathbb{Z}_{m_i} \oplus F_1 \cong \bigoplus_{i=1}^t \mathbb{Z}_{r_i} \oplus F_2
	\end{equation*}
	where $F_1$ and $F_2$ are free abelian, $m_i,r_i$ are integers such that $m_1,r_1>1$, $m_1|m_2|\cdots| m_k$ and $r_1|r_2|\cdots | r_t$. Clearly $F_1 \cong F_2$. Now we write
	\begin{align*}
		m_i &= p_1^{c_{i1}} p_2^{c_{i2}} \dots p_\ell^{c_{i\ell}} \quad \text{ for } i=1,\dots, k,
		\\
		r_i &= p_1^{d_{i1}} p_2^{d_{i2}} \dots p_\ell^{d_{i\ell}} \quad \text{ for } i=1,\dots, t
	\end{align*}
	where $p_1,\dots, p_\ell$ are distinct primes and $c_{1j} \neq 0$ for some $j$ and $d_{1j'}\neq 0$ for some $j'$ since $m_1,r_1>1$. Also, we have $0\leq c_{1j} \leq c_{2j} \leq \cdots \leq c_{kj}$ and $0\leq d_{1j} \leq d_{2j} \leq \cdots \leq d_{tj}$ for all $j$. Note that the direct sums of cyclic groups are isomorphic to $G_t$. By Lemma \ref{lemma-FGAG-coprime}, we get
	\begin{equation*}
		\bigoplus_{i=1}^{k}  \bigoplus_{j=1}^{\ell} \mathbb{Z}_{p_j^{c_{ij}}} \cong \bigoplus_{i=1}^k \mathbb{Z}_{m_i} \cong G_t \cong \bigoplus_{i=1}^t \mathbb{Z}_{r_i} \cong \bigoplus_{i=1}^{t}  \bigoplus_{j=1}^{\ell} \mathbb{Z}_{p_j^{d_{ij}}}.
	\end{equation*}
	By Theorem \ref{thm-decomposition-of-AG-elem-div}, both direct sums must have same number of nonzero summands (elementary divisors), so $k = t$ and $c_{ij} = d_{ij}$ for all $i,j$. Therefore $m_i = r_i$ for all $i = 1,\dots, k$.
\end{sketch}
\begin{definition}
	The integers $m_1,\dots, m_r$ in Theorem \ref{thm-decomposition-of-modules-inv-fac} are called the \textbf{invariant factors} of $G$.
\end{definition}
The isomorphism between finitely generated abelian groups can be studied using invariant factors (resp. elementary divisors).
\begin{corollary}
	Let $G$ and $H$ be finitely generated abelian groups. Then $G\cong H$ if and only if $\Rank (G/ G_t)=\Rank (H/ H_t)$ and $G$ and $H$ have
	the same invariant factors (resp. elementary divisors). 
\end{corollary}
\begin{sketch}
	The result follows from the uniqueness.
\end{sketch}

\begin{remark}
	To count the number of nonisomorphic abelian groups of order $n$, it suffices to count the number of nonisomorphic abelian groups of order the highest prime power $p^k$ appeared in the prime factorization of $n$. It is the number of partitions of $k$, denoted $a(k)$. If $n = p_1^{k_1}\cdots p_\ell^{k_\ell}$, then the number of nonisomorphic abelian groups is given by
	$
		\prod_{i=1}^\ell a(k_i)
	$. There is no explicit formula for $a(k)$, making it difficult to count the groups of large order.
\end{remark}

\paragraph{Main References.} \cite{Lang2002,Hungerford1974,Kaplansky1954,Kaplansky1977,Robinson1982}