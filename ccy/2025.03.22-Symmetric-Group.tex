\part{Group Theory}
\section{Symmetric Groups} \label{sec-symmetric-groups}
Throughout the note, we let $[n] = \{1,\dots, n\}$.
\subsection{Basic Definitions}
\begin{definition}
	The group $S_n$ of all bijections $[n]\rightarrow [n]$ is called the \textbf{symmetric group}. The elements of $S_n$ are called \textbf{permutations}. A permutation $\sigma$ can be expressed by
	\begin{equation*}
		\sigma = \begin{pmatrix}
			1 & 2 & \cdots & n\\
			\sigma(1) &  \sigma(2) & \cdots & \sigma(n)
		\end{pmatrix}.
	\end{equation*}
\end{definition}
\begin{definition}
	Let $a_1,\dots, a_r$, $(r\leq n)$ be distinct elements in $[n]$. Then the $r$-\textbf{cycle} $(a_1a_2\cdots a_r)$ is the permutation $\sigma\in S_n$ defined by
	\begin{gather*}
		\sigma(a_i) = a_{i+1},\quad \forall i\in\mathbb{Z}_r,
		\\
		\sigma(x) = x,\quad \forall x\not\in\{a_1,\dots, a_r\}.
	\end{gather*} 
	Here, $r$ is called the \textbf{length} of the cycle; a $2$-cycle is called a \textbf{transposition}.
\end{definition}
\begin{definition}
	The permutations $\sigma_1,\dots, \sigma_r\in S_n$ are said to be \textbf{disjoint} if for each $1\leq i\leq r$, and every $k\in[n]$, 
	\begin{equation*}
		\sigma_i(k)\neq k\implies \sigma_j(k)=k\quad \forall j\neq i.
	\end{equation*}
	In particular, two cycles $(a_1a_2\cdots a_r)$ and $(b_1b_2\cdots b_s)$ are disjoint if $\{a_1,a_2,\dots, a_r\}\cap \{b_1,b_2,\break\dots ,b_s\} = \emptyset$.
\end{definition}

\begin{proposition} \label{01-prop-1}
	\hfill
	\begin{enumerate}[(i)]
		\item $|S_n| = n!$;
		\item $(a_1a_2\cdots a_r) = (a_2a_3\cdots a_ra_1) = \cdots = (a_ra_1\cdots a_{r-2}a_{r-1})$;
		\item Any $1$-cycle is the identity permutation, and hence it can be omitted when expressing any product of cycles;
		\item $|(a_1a_2\cdots a_r)| = r$;
		\item $(a_1a_2\cdots a_r)^{-1} = (a_ra_{r-1}\cdots a_1)$;
		\item If $\sigma,\tau\in S_n$ are disjoint, then $\sigma\tau = \tau\sigma$. Furthermore, $\sigma \tau = \text{Id}$ implies $\sigma = \tau = \text{Id}$.
	\end{enumerate}
\end{proposition}
The action of conjugation on $S_n$ has a nice property.
\begin{proposition} \label{prop-conjugation-on-S_n}
	Let $\tau = (a_1\cdots a_r)$ be an $r$-cycle and $\sigma\in S_n$. Then \begin{equation*}
		\sigma\tau\sigma^{-1} = \sigma(a_1\cdots a_r)\sigma^{-1} = (\sigma(a_1)\sigma(a_2)\cdots \sigma(a_r)).
	\end{equation*}
\end{proposition}
\begin{sketch}
	Write $[n] = \{\sigma(a_1),\dots, \sigma(a_r), \sigma(b_{r+1}),\dots, \sigma(b_n)\}$ where $b_{r+1},\dots, b_n\not\in \{a_1,\dots,\break a_r\}$. The following
	\begin{gather*}
		\sigma(a_i)\overset{\sigma^{-1}}{\longmapsto}  a_i \overset{\tau}{\longmapsto} a_{i+1}   \overset{\sigma}{\longmapsto}  \sigma(a_{i+1}),
		\\
		\sigma(b_i)\overset{\sigma^{-1}}{\longmapsto}  b_i \overset{\tau}{\longmapsto} b_i   \overset{\sigma}{\longmapsto}  	\sigma(b_i)
	\end{gather*}
	prove the assertion.
\end{sketch}

\subsection{Cycle Decomposition}
\begin{theorem} \label{thm-cycle-decomposition}
	Every nonidentity permutation in $S_n$ can be decomposed into a product of disjoint cycles, each of which has length at least $2$. This decomposition is unique
	up to the order of the cycles in the product.
\end{theorem} 
\begin{sketch} Let $\sigma$ be the nonidentity permutation.
	\begin{enumerate}
		\item The set $[n]$ can be partitioned into the orbits of $\langle \sigma\rangle$ on $[n]$, i.e., $[n] = \bigcup_i \langle \sigma\rangle x_i$ (will be studied in Group Actions).
		\item If $\langle \sigma\rangle x_i = \{x_i\}$, then it corresponds to the identity permutation.
		\item If $\langle \sigma\rangle x_i \neq \{x_i\}$, then there must be $\sigma^{k+\ell} x_i = \sigma^{\ell} x_i$ for some positive integer $k$ and some nonnegative integer $\ell$, since $\langle\sigma\rangle x_i$ is finite. This implies there is the least positive integer  $m$ such that $\sigma^m x_i = x_i$. Therefore $\langle \sigma\rangle x_i = \{x_i,\sigma x_i, \sigma^2 x_i,\dots, \sigma^{m-1}x_i\}$ and it induces a cycle $\tau_i = (x_i(\sigma x_i)\cdots (\sigma^{m-1}x_i))$.
		\item  The induced cycles are disjoint since the orbits are disjoint. Also, $\sigma x = \tau_i x$ if $x\in \langle \sigma\rangle x_i$, completing the proof, i.e., $\sigma = \tau_1\cdots \tau_r$ for some $r$. Remark that any cycle of length $1$ is omitted in the product.
		\item To prove the uniqueness,  consider $\sigma  = \phi_1\cdots \phi_s$ for some $s$. Take $x\in[n]$ such that $\sigma(x)\neq x$. Then there is a unique $\phi_j$ such that $\phi_j(x) = \sigma(x)$. It follows from $\phi_j\sigma  = \sigma\phi_j$ that $\phi^k_j(x) = \sigma^k(x)$ and hence the orbit of $x$ under $\phi_j$ is one of  the orbits of $\sigma$. By Proposition \ref{01-prop-1}.(ii), we get  $\phi_j = \tau_i$. So $s=r$ and $\phi_i =\tau_i$ after reindexing. \qedhere
	\end{enumerate}
\end{sketch}

\begin{corollary}
	The order of a permutation $\sigma\in S_n$ is the least common multiple of the orders of its disjoint cycles.
\end{corollary}
\begin{sketch}Write $\sigma = \tau_1\cdots \tau_r$. Then
	$\sigma^m = \text{Id}$ if and only if $\tau_i^m = \text{Id}$ (by Proposition \ref{01-prop-1}.(vi)) if and only if $|\tau_i|\big|m$ for all $i$. By definition, $\operatorname{lcm}(|\tau_1|,\dots,|\tau_r|)$ is the least integer that is divisible by all $|\tau_i|$.  
\end{sketch}

\subsection{Generators of Symmetric Groups}
\begin{proposition}
The following sets are generators of $S_n$:
	\begin{enumerate}[(i)] 
		\item the set of all transpositions;
		\item $\{(12),(13),(14),\dots,(1n)\}$;
		\item $\{(12),(23),(34),\dots, (n-1\,\, n)\}$;
		\item $\{(12),(123\cdots n)\}$;
		\item $\{(12),(23\cdots n)\}$;
		\item  if $n=p$ where $p$ is a prime , then $\{(rs),(123\cdots p)\}$ where $(rs)$ is any transposition.
	\end{enumerate}
\end{proposition}
\begin{sketch}
	\begin{enumerate}[(i)]
		\item Note that 
		\begin{align*}
			(x_1) &= (x_1x_2)(x_2x_1),
			\\
			(x_1x_2x_3\cdots x_r) &= (x_1x_r)(x_1x_{r-1})\cdots (x_1x_3)(x_1x_2).
		\end{align*} Then use Theorem \ref{thm-cycle-decomposition}.
		\item Note that $(ij) = (1i)(1j)(1i)$. Then use (i).
		\item Note that $(1j) = (1\,\, j-1)(j-1\,\, j)(1\,\, j-1)$. Then use (ii).
		\item Let $\tau = (12)$ and $\sigma = (123\cdots n)$. Then use Proposition \ref{prop-conjugation-on-S_n} and (iii).
		\item Let $\tau = (12)$ and $\sigma = (23\cdots n)$. Then use Proposition \ref{prop-conjugation-on-S_n} and (ii).
		\item Let $\tau = (rs)$ and $\sigma = (123\cdots p)$. Set $d = s-r$. By Proposition \ref{prop-conjugation-on-S_n}, $\langle \tau,  \sigma\rangle$ contains $\{(k\,\, k+d)\,|\,k\in\mathbb{Z}_p\}$.  Note that $(k+d\,\, k+2d)(k\,\, k+d)(k+d\,\, k+2d) = (k\,\, k+2d)$ for all $k\in \mathbb{Z}_p$. Inductively, $(k+id\,\, k+(i+1)d)(k\,\, k+id)(k+id\,\, k+(i+1)d) = (k\,\, k+(i+1)d)$ for all $k\in\mathbb{Z}_p$ and $i\in \mathbb{Z}_p\setminus\{0,p-1\}$. Since $d\in \mathbb{Z}_p^*$, there exists $d^{-1}\in \mathbb{Z}_p^*$ such that $d^{-1}d = 1$. In particular, $\langle \tau,  \sigma\rangle$ contains $(k,k+d^{-1}d) = (k,k+1)$ for all $k\in\mathbb{Z}_p$. Hence the assertion is proved by (iii). \qedhere
	\end{enumerate}
\end{sketch}

\subsection{Sign of a Permutation}
\begin{definition}
	A permutation in $S_n$ is said to be \textbf{even} (resp. \textbf{odd}) if it can be written as a product of an even (resp. odd) number of transpositions.
\end{definition}
As discussed in the previous section, any permutation can
be decomposed into a product of transpositions. The decomposition into a
product of transpositions is not unique, but the numbers of transpositions 
appearing in these decompositions are always all even or all odd.
\begin{theorem}
	A permutation in $S_n$ ($n\geq 2$) cannot be both even and odd.
\end{theorem}
\begin{sketch}
	A permutation $\sigma\in S_n$ induces a permutation matrix $P_{\sigma}\in \operatorname{GL}(n,\mathbb{R})$, i.e. $P_{\sigma} = (p_{ij})$ where 
	\begin{equation*}
		p_{ij} = \begin{cases}
			1 & \text{if } \sigma(j) = i,
			\\
			0 & \text{if } \sigma(j) \neq  i.
		\end{cases}
	\end{equation*} So the map $S_n\rightarrow \operatorname{GL}(n,\mathbb{R}); \sigma\mapsto P_{\sigma}$ is a group homomorphism. The
	transposition $\tau$ is mapped to a matrix formed by swapping two columns of the identity matrix, and thus $\det P_\tau = -1$. Hence $\sigma$ cannot be both even and odd, since $\det P_\sigma$ can only take one value.
\end{sketch}
\begin{definition}
	The \textbf{sign} of a permutation $\tau\in S_n$ is the group homomorphism  $\operatorname{sgn}:S_n\rightarrow \{-1,1\}$ (here $\{-1,1\}$ is a multiplicative group) defined by 
	\begin{equation*}
		\operatorname{sgn}(\tau) = \begin{cases}
			1 &\text{if }\tau\text{ is even},
			\\
			-1 &\text{if }\tau\text{ is odd}.
		\end{cases}
	\end{equation*}
\end{definition}

\subsection{Alternating Group}
\begin{definition}
	The group of all even permutations of $S_n$ is called the \textbf{alternating group} of degree $n$ and is denoted $A_n$.
\end{definition}

\begin{lemma} \label{01-lemma-A_n-by-3-cycles}
	When $n\geq 3$, $A_n$ is generated by the set of all $3$-cycles.
\end{lemma}
\begin{sketch}
	Any $\sigma\in A_n$ is a product of terms of the form $(ab)(cd)$. Observe that
	\begin{equation*}
		(ab)(cd) = \begin{cases}
			1 &\text{if }\{a,b\} = \{c,d\}, \\
			3\text{-cycle} &\text{if }|\{a,b\} \cap \{c,d\}|=1,
			\\
			(acb)(acd) & \text{if }\{a,b\} \cap \{c,d\} = \emptyset. 
		\end{cases}\qedhere
	\end{equation*}
\end{sketch}
\begin{theorem}
	Let $n\geq 2$. Then $A_n$ is a normal subgroup of $S_n$ of index $2$ and order $n!/2$. Furthermore $A_n$ is the only subgroup of $S_n$ of index $2$.
\end{theorem}
\begin{sketch}
	The first assertion is proved by the First Isomorphism Theorem on the homomorphism $\operatorname{sgn}$. To prove the second assertion, suppose that $H\neq A_n$ is a subgroup of index $2$. By Lemma \ref{01-lemma-A_n-by-3-cycles}, there exists a $3$-cycle $(abc)\not\in H$ (otherwise $A_n\subseteq H$ and thus $A_n = H$ because $|A_n|= |H|$). However $H$, $(abc)H$ and $(abc)^{-1}H$ are distinct cosets, a contradiction.
\end{sketch}
\begin{lemma} \label{lemma-A_n-gen-by-n-2-cycles}
	Let $r,s$ be distinct elements of $[n]$. Then $A_n$  $(n \geq 3)$ is generated by the $n-2$ cycles $(rsk)$, $1\leq k\leq n$ with $k\neq r,s$.
\end{lemma}
\begin{sketch}
	Let $a,b,c$ be distinct elements and $a,b,c\neq r,s$. The decompositions of any $3$-cycles into a product of the $n-2$ cycles mentioned above are presented without any motivation:
	\begin{align*}
		(rsa)& \text{ is trivial},
		\\
		(ras) &= (rsa)^2,
		\\
		(rab) &= (rsb)(rsa)^2,
		\\
		(sab) &= (rsb)^2(rsa),
		\\
		(abc) &= (rsa)^2(rsc)(rsb)^2(rsa).
	\end{align*}
	Hence the result follows from Lemma \ref{01-lemma-A_n-by-3-cycles}.
\end{sketch}
\begin{lemma} \label{lemma-A_n-contains-3-cycle}
	If $N$ is a normal subgroup of $A_n$ $(n\geq 3)$ and $N$ contains a $3$-cycle, then $N=A_n$.
\end{lemma}
\begin{sketch}
	Without loss of generality, assume that  $(123)\in N$. Then $(213) = (123)^2\in N$. Since $N$ is normal in $A_n$, $N$ contains all the conjugates $\sigma (213)\sigma^{-1}$ $(\sigma\in A_n)$. In particular, if we choose $\sigma = (12)(3k) $, where $k\geq 4$, then $\sigma (213)\sigma^{-1} = (12k)$. The result follows by Lemma \ref{lemma-A_n-gen-by-n-2-cycles}.
\end{sketch}

\begin{theorem}
	The alternating group $A_n$ is simple if and only if $n\neq 4$.
\end{theorem}
\begin{sketch}\hfill
	\begin{enumerate}
		\item $A_2$ only contains the identity permutation.
		\item The order of $A_3$ is $3$ (which is a prime) and hence a cyclic group.
		\item $\{(1),(12)(34),(13)(24),(14)(23)\}\lhd A_4$ provides a counterexample.
		\item For $n\geq 5$, let $N$ be a nontrivial normal subgroup of $A_n$.  For any $\sigma \in S_n$, we say that $i \in [n]$ is a \textbf{fixed point} of $\sigma$ if $\sigma(i) = i$. The number of fixed points of $\sigma$ is denoted by $[n]_\sigma$. 
		\begin{enumerate}
			\item Take a permutation $\sigma\in N$ so that $[n]_\sigma$ is the largest among the permutations in $N$. Two cases to consider: $\sigma$ is a product of disjoint transpositions; and $\sigma$ is a cycle of length at least $3$.
			\item For the first case, we can find two disjoint transpositions $(ab)$ and $(cd)$. Then we take $x\neq a,b,c,d$ and define
			\begin{align*}
				\tau &= (cdx)\in A_n, \\
				\sigma' &= [\tau,\sigma] = \tau\sigma\tau^{-1}\sigma^{-1} \in N.
			\end{align*}
			It can be checked that we see that $\sigma'(c) = x$ (and thus nontrivial) and  $\sigma'$ fixes $a,b$ and $[n]_{\sigma}\setminus\{x\}$. This means that $[n]_{\sigma'}>[n]_{\sigma}$, a contradiction.
			\item For the second case, if it is of length $3$, then we are done. If it is of length $4$, then we arrive at a contradiction because it is an odd permutation. If it is of length at least $5$, i.e., $\sigma = (abcdx\cdots)$, then using the same construction in (b) one can see that $\sigma'(c) = d$ and $\sigma'$ fixes $b$ and $[n]_\sigma$. So $[n]_{\sigma'}>[n]_{\sigma}$, again a contradiction.
		\end{enumerate}
		\item Therefore $\sigma $ is a cycle of length $3$. Hence Lemma \ref{lemma-A_n-contains-3-cycle} gives the result. \qedhere
	\end{enumerate}
	
\end{sketch}

\paragraph{Main References.} \cite{Suzuki1982,Lang2002,Hungerford1974,Li2025}

\newpage
\section{Dihedral Groups and the Quaternion Group}
\subsection{Dihedral Groups}
\begin{definition}
	  Let $n\geq 3$ and let $d:\mathbb{R}^2\times \mathbb{R}^2\rightarrow \mathbb{R}_{\geq 0}$ be the Euclidean distance. A \textbf{symmetry} $s$ of a regular polygon $P_n$ of $n$ sides  is a bijection that preserves distances, i.e., if $x,y\in P_n$, then $d(x,y)\implies d(s(x),s(y))$.    The \textbf{dihedral group} $D_{n}$ is  the group of
	symmetries of  $P_n$.
\end{definition}
\begin{proposition}
	Let $r$ be a rotations of degree $360^\circ/n$ clockwise around the center of the polygon $P_n$ and let $s$ be a fixed reflection about a line through the center and one vertex $v$. Then $D_{n} = \langle r,s\rangle$ and hence $|D_{n}|  = 2n$.
\end{proposition}
\begin{sketch}
	Let $\sigma$ be a symmetry of $P_n = \{1,2,\dots, n\}\subseteq \mathbb{Z}_n$, labelled in clockwise direction and assume that $n = v$. We observe the following:
	\begin{enumerate}[(i)]
		\item $r^{\ell}(k) = k+\ell$ for all $k,\ell\in \mathbb{Z}_n$;
		\item $s(k) = -k$ for all $k\in \mathbb{Z}_n$. 
	\end{enumerate} Suppose that $\sigma$ maps $1$ to some $i$. The possible values of $\sigma(2)$ are $i-1$ and $i+1$. If $\sigma(2) = i+1$, then $\sigma(k) = k+i-1$ for all $k\in\mathbb{Z}_n$ and thus $\sigma$ is equal to  $r^{i-1}$.   If $\sigma(2) = i-1$, then $\sigma(k) = i+1-k$  for all $k\in\mathbb{Z}_n$. This implies $\sigma(k) = r^{i+1}(-k) = r^{i+1}s(k)$ for all $k\in \mathbb{Z}_n$ and thus $\sigma = r^{i+1}s$. Since $s^2=e$, the first assertion is proved. For the second assertion, it suffices to verify that
	\begin{enumerate}[(i)]
		\item $|r| = n$ and $|s| = 2$;
		\item $r^is = sr^{-i}$ for all $0 \leq i \leq n$;
		\item $s\neq r_i$ for all $i$.
	\end{enumerate}
	This leads to $D_{n}= \{1,r,r^2\cdots, r^{n-1},s,sr,sr^2\cdots, sr^{n-1}\}$ where the elements in the set are distinct.
\end{sketch}
\subsection{Presentations of Dihedral Groups}
A way to describe a group is by using a \textbf{presentation}. Informally, it is an expression
of the form $\langle X | R\rangle$, where $X$ is a set of ``generators", and $R$ is  a set
of ``relations". The precise definition will be introduced after we have studied free groups. For $n\geq 1$, $D_{n}$ has a usual presentation $\langle r,s\,|\, r^n=s^2=1, rs=sr^{-1}\rangle$. Clearly $D_{1}\cong \mathbb{Z}_2$ and $D_{2}\cong \mathbb{Z}_2\times \mathbb{Z}_2$. This extends the definition of dihedral groups.
\begin{comment}
	\begin{theorem}
		Let $n\geq 3$.
		\begin{enumerate}[(i)]
			\item If $n$ is odd, then $Z(D_n) = \{1\}$.
			\item If $n$ is even, then $Z(D_n) = \{1,r^{n/2}\}$.
		\end{enumerate}
	\end{theorem}
	\begin{sketch}
		For any $i$, $sr^i$ is not in $Z(D_n)$, otherwise $r^2=1$, a contradiction. 
		
		If $r^i\in Z(D_n)$, then $r^is = sr^i$ implies $r^{2i}=1$. This implies $2i = n$ or $2i = 0$.
	\end{sketch}
\end{comment}

The following are some of the presentations commonly used to express $D_n$:
\begin{enumerate}[(i)]
	\item A subgroup of $S_n$ generated by $(123\cdots n)$ and $\begin{pmatrix}
		1 & 2 & 3 & \cdots & n-1 & n\\
	1	 & n & n-2 & \cdots & 3 & 2
	\end{pmatrix}$.
	\item A subgroup of $\operatorname{GL}(2,\mathbb{C})$ generated by $\begin{pmatrix}
		\cos\frac{2\pi}{n} & -\sin\frac{2\pi}{n}
		\\
		\sin\frac{2\pi}{n} & \cos\frac{2\pi}{n}
	\end{pmatrix}$ and $\begin{pmatrix}
	1 & 0 \\
	0 & -1
	\end{pmatrix}$.
	\item A subgroup of $\operatorname{GL}(2,\mathbb{C})$ generated by $\begin{pmatrix}
		e^{2\pi i/n} & 0
		\\
		0 & e^{-2\pi i/n}
	\end{pmatrix}$ and $\begin{pmatrix}
		0 & 1 \\
		1 & 0
	\end{pmatrix}$.
\end{enumerate}

\subsection{Subgroups of Dihedral Groups}
We can give the complete description of subgroups of dihedral groups. The proof below depends on Theorem \ref{thm-Van-Dyck} (will be studied in Section \ref{sec-group-presentations}).


\begin{theorem} \label{thm-subgrp-of-Dn}
	 Every subgroup of a dihedral group $D_{n}$ is cyclic or dihedral. In fact, every subgroup of $D_n = \langle r,s\rangle$ belongs to one of the following lists:
	 \begin{enumerate}[(i)]
	 	\item $\langle r^d \rangle$, where $d$ is a positive divisor of $n$;
	 	\item $\langle r^d, r^is\rangle$, where $d$ is a positive divisor of $n$ and $0\leq i\leq d-1$.
	 \end{enumerate}
\end{theorem}
\begin{sketch}
	Clearly $D_n$ contains a cyclic group $C$ of order $n$ consisting of rotations.  Let $H$ be any subgroup of $D_n$. If $H  \leq C$,
	then $H$ is cyclic. If not, then note that $H \cap C$ is a proper subgroup of $C$. Let $C = \langle r\rangle$. Then $H\cap C = \langle r^d\rangle$ for some positive divisor $d$ of $n$. Take $r^i s\in H$. Then we can verify that $H = \langle r^d, r^is \rangle$. Since $|H| = 2|r^d|$,   $|r^d| = n/d$, $|r^i s| = 2$ and $r^{d+i}s = sr^{-i-d} = r^{i}sr^{-d}$, we have   $H\cong D_{n/d}$.
\end{sketch}

\begin{theorem} \label{thm-normal-subgrp-of-Dn}
	If $n$ is odd, then the proper normal subgroups of $D_n$ are those in Theorem \ref{thm-subgrp-of-Dn}.(i). If $n$ is even, the proper normal subgroups of $D_n$ are those in Theorem \ref{thm-subgrp-of-Dn}.(i), together with $\langle r^2,s\rangle$ and $\langle r^2,rs\rangle$.
\end{theorem}
\begin{sketch}
	If $H$ is a cyclic subgroup of $G$ that is normal in $G$, then the subgroups of $H$ is also normal in $G$ (see Hungerford Exercises I.5.11). So the subgroups in Theorem \ref{thm-subgrp-of-Dn}.(i) are normal in $D_n$.
	
	Note that for all $k\in \mathbb{Z}$, we have $r(r^ks)r^{-1} = r^{k+2}s$. So $r^ks$ and $r^{k+2}s$ must belong to the same  conjugacy class (will be studied in Group Actions). Let $N$ be a normal subgroup in $D_n$ containing at least a reflection (those without reflections have been classified in the first paragraph). Now we argue in two cases.
	
	\noindent\textit{Case I.} If $n$ is odd, then the conjugacy classes containing a reflection is $[s]=\{r^ks\,|\, k\in\mathbb{Z}\}$. So $N$ must contain this entire set. So $|N| > n$ since there are $n$ reflections and an identity element. Hence we have $N = D_{n}$ by Lagrange's Theorem.
	
	\noindent\textit{Case II.} If $n$ is even, we have two conjugacy classes containing reflections, i.e., $[s] = \{r^ks\,|\, k\in\mathbb{Z}\text{ is even}\}$ and $[rs] =\{r^ks\,|\, k\in\mathbb{Z}\text{ is odd}\}$. If $N$ contains $[s]$ and $[rs]$, then $N = D_n$. If $N$ contains exactly one of them, then $|N|>n/2$. So $[D_n:N] < 4$. Also, the element $r^is\not\in N$ implies that $r^isN$ has order two in the quotient group $D_n/N$. Thus $|D_n/N|=[D_n:N]$ must be even and we get $|D_n/N| = 2$. In particular, $r^2\in N$ since $N= rNrN = r^2N$. If $r\in N$, then $N = D_n$. If $r\not\in N$, then 
	\begin{equation*}
		N = \begin{cases}
			\langle r^2, s\rangle &\text{if }N\text{ contains }[s],
			\\
			\langle r^2, rs\rangle &\text{if }N\text{ contains }[rs].
		\end{cases}
	\end{equation*}
\end{sketch}

\subsection{Quaternion Group}
\begin{definition}
	The \textbf{quaterion group} is the group $Q_8 = \langle i,j\,|\, i^4=1, i^2=j^2, ij = ji^{-1}  \rangle$. Another presentation is $\langle -1, i,j,k\,|\, (-1)^2=1, i^2=j^2=k^2 = ijk = -1 \rangle$.
\end{definition}
\begin{proposition}
	The order of $Q_8$ is $8$.
\end{proposition}
\begin{sketch}
	Let $Q_8 = \langle i,j\,|\, i^4=1, i^2=j^2, ij = ji^{-1}  \rangle$. The relation $ij = ji^{-1}$ forces every elements in $Q_8$ can be written as the form $i^sj^t$ $(s,t\in\mathbb{Z})$. From   $i^4=1$ and  $i^2=j^2$, one can restrict $s,t$ to  $s \in\{0,1,2,3\}$ and $t\in \{0,1\}$. Observe that $\langle i \rangle \cap \langle j \rangle = \langle i^2\rangle$. If $i^{s_1}j^{t_1} = i^{s_2}j^{t_2}$, then $j^{t_1-t_2}\in \langle i^2\rangle$ and so $t_1=t_2$. Hence $s_1=s_2$. This shows that $i^sj^t$ $(s \in\{0,1,2,3\},t\in \{0,1\})$ are distinct elements in $Q_8$.
\end{sketch}
The following are some usual presentations of $Q_8$: 
\begin{enumerate}[(i)]
	\item A subgroup of $\operatorname{GL}(2,\mathbb{C})$ generated by $\begin{pmatrix}
		0 & i\\
		i & 0
	\end{pmatrix}$ and $\begin{pmatrix}
	0 & 1\\
	-1 & 0
	\end{pmatrix}$.
	\item A subgroup of $\operatorname{GL}(2,\mathbb{C})$ generated by $\begin{pmatrix}
		e^{2\pi i/n} & 0
		\\
		0 & e^{-2\pi i/n}
	\end{pmatrix}$ and $\begin{pmatrix}
		0 & 1 \\
		-1 & 0
	\end{pmatrix}$.
\end{enumerate}

\begin{theorem} \label{thm-subgrp-of-Q8-are-normal}
	All subgroups of $Q_8$ are normal.
\end{theorem}
\begin{sketch}
	Let $Q_8 = \langle i,j\,|\, i^4=1, i^2=j^2, ij = ji^{-1}  \rangle$. Direct verification shows that $\langle i\rangle$, $\langle j\rangle$ and $\langle ij\rangle$ are the only subgroups of order $4$, and hence they are normal. Also, the only subgroup of order $2$ is $\langle i^2 \rangle = \langle j^2 \rangle = \langle (ij)^2 \rangle$. We can check that $\langle i^2\rangle$ is normal. 
\end{sketch}
If a group $G$ is abelian, then all the subgroups of $G$ are normal in $G$. Theorem \ref{thm-subgrp-of-Q8-are-normal} provides counterexamples to the converse of this statement. Moreover, $D_4\not\cong Q_8$, which can be seen from their normal subgroups.

\begin{theorem} \label{thm-nonabelian-order-8}
	Let $G$ be a nonabelian group of order $8$. Then either $G \cong D_4$ or $G\cong Q_8$.
\end{theorem}
\begin{sketch}
	There are no elements of order $8$, and there is an element $g$ of order $4$ (otherwise $x^2=1$ for all $x\in G$ implies $G$ is abelian). So $N=\langle g\rangle$ is a subgroup of index $2$, and hence a normal subgroup in $G$. 
	
	Take a nonidentity element $h\not\in N$. Then $h^2\in N$, otherwise,  $h^2 \not\in N$ implies $h^2\in hN$ and so $h\in N$. Also, $h^{-1}gh = g^{-1}$ since $N$ is normal and $G$ is not abelian.
	
	If $|h| = 2$, then $h^2= 1$ and hence $G\cong D_4$.
	
	If $|h| = 4$, then $h^2 = g^2$ ($g^2$ is the only element of order $2$ in $N$) and hence $G\cong Q_8$.
\end{sketch}

\paragraph{Main References.} \cite{DummitFoote2004,Artin1991,Hungerford1974}

