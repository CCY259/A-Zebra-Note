\section{Solvable and Nilpotent Groups}
\subsection{Solvable Groups}
\begin{definition}
	A group $G$ is said to be \textbf{solvable} if there exists a subnormal series
	\begin{equation*}
		\{e\} = G_0 \leq G_1 \leq \cdots \leq G_n = G
	\end{equation*}
	such that $
		G_{i+1}/G_{i}$ is abelian
	for $0\leq i<n$.
\end{definition}


We give a characterization of solvability is in terms of the ``derived series".  

Recall that the commutator subgroup $[G,G]$ of $G$ is the subgroup generated by all commutators $[x, y] = xy  x^{-1} y^{-1}$ for $x, y \in G$. It is the unique smallest normal subgroup of $G$ such that the corresponding quotient group is abelian.

Let  $G^{(0)} = G$ and  general, $G^{(n)} = [G^{(n-1)},G^{(n-1)}]$ for $n > 0$. Clearly $G^{(i)} \geq G^{(i+1)}$ and we can use induction to show that $G^{(i)}\lhd G$ for all $i$. Hence $G^{(0)} \geq G^{(1)} \geq G^{(2)} \geq \cdots$ constitutes the \textbf{derived series}  of $G$.

\begin{theorem} \label{thm-solvable}
	A group $G$ is solvable if and only if  $G^{(n)} = \{e\}$ for some integer $n$.
\end{theorem}
\begin{sketch}
	If $G^{(n)} = \{e\}$ for some integer $n$, then the derived series
	$$\{e\} = G^{(n)} \leq G^{(n-1)} \leq \cdots \leq G^{(1)} \leq G^{(0)} = G$$
	shows that $G$ is solvable.
	
	Conversely, suppose $G$ is solvable. Then we have subgroups $G_i \lhd G$ with
	$$\{e\} = G_0 \leq G_1 \leq\cdots \leq G_n = G$$
	and such that $G_{i+1}/G_i$ is abelian for $0 \leq i < n$. Thus $[G_{i+1},G_{i+1}] \subseteq G_i$ for all $i$ by the definition of commutator subgroups. In particular, $G^{(1)} = [G_n,G_n] \subseteq G_{n-1}$. Inductively, we obtain $G^{(i)} \subseteq G_{n-i}$ for $0 \leq i \leq n$. Consequently, $G^{(n)} \subseteq G_0 = \{e\}$.
\end{sketch}

\begin{corollary} \label{cor-solvable}
	Subgroups and quotient groups of solvable groups are solvable.
\end{corollary}
\begin{sketch}
	Let $G$ be a solvable group. Then for some $n$ we have $G^{(n)} = \{e\}$.
	
	Let $H$ be a subgroup of  $G$. Then we have $H^{(k)} \subseteq G^{(k)}$ for all $k$ (using induction), and hence $H$ is solvable.
	
	If $N \lhd G$, then the canonical projection $\rho : G \rightarrow G/N$ yields that $(G/N)^{(k)} = \varphi(G^{(k)})$ for all $k$, and so   $(G/N)^{(n)} = \varphi(G^{n}) =\{e\}$.
\end{sketch}

\begin{proposition}
	If $G$ and $H$ are solvable groups, then $G\rtimes H$ is solvable.
\end{proposition}
\begin{sketch}
	BRUH
\end{sketch}




\begin{theorem}
	Let
	$$\{e\} = H_0 < H_1 < \cdots < H_n = G$$
	be a composition series for $G$. Then $G$ is solvable if and only if each composition factor $H_{i+1}/H_i$ has prime order.
\end{theorem}

\begin{sketch}
	Assume that $G$ is solvable. Then each composition factor $H_{i+1}/H_i$ is solvable by Corollary \ref{cor-solvable}. Since these factors are simple, it suffices to show that a solvable simple group must have prime order. If $G$ is a solvable simple group, then $[G,G] =\{e\}$ (we cannot have $[G,G] = G$ by Theorem \ref{thm-solvable}) and $G$ is abelian. It follows from Proposition \ref{prop-abelian-is-simple-iff-prime-order} that simple abelian groups have prime order. 
	
	
Conversely, assume that the composition factors have prime order. These factors $H_{i+1}/H_i$ are abelian since every group of prime order is cyclic.  Therefore $G^{(i)} \subseteq H_{n-i}$ for all $i$ by induction.  It follows that $G^{(n)} = \{e\}$ and $G$ is solvable.
\end{sketch} 
\begin{definition}
	Let $G$ be a  solvable group. The \textbf{derived length} $\operatorname{dl}(G)$ of $G$ is the smallest integer $n$ such that $G^{(n)} = \{e\}$.
\end{definition}
\begin{corollary}
	Let $N$ be a normal subgroup of  a group $G$. Then $G$ is solvable if and only if both $N$ and $G/N$ are solvable. Moreover, $$\text{dl}(G) \leq \text{dl}(N) + \text{dl}(G/N).$$
\end{corollary}
\begin{sketch}
	 The group $G$ is solvable implies that $N$ and $G/N$ are solvable by Corollary \ref{cor-solvable}.
	 
	 Conversely, assume that both $N$ and $G/N$ are solvable. Let $\operatorname{dl}(N) = n$ and $\operatorname{dl}(G/N) = m$. Let  $\rho:G \rightarrow G/N$ be the canonical projection. Since $\rho(G^{(m)}) = (G/N)^{(m)} = \{e\}$, we get $G^{(m)} \subseteq \ker \rho = N$. Thus $G^{(m+n)} = (G^{(m)})^{(n)} \subseteq N^{(n)} = 1$, and the result follows.
\end{sketch}





\subsection{Nilpotent Groups}
\begin{definition}
	A group $G$ is said to be \textbf{nilpotent} if there exists a subnormal series
	\begin{equation*}
		\{e\} = G_0 \leq G_1 \leq \cdots \leq G_n = G
	\end{equation*}
	such that
	\begin{equation*}
		G_{i+1}/G_{i} \subseteq Z(G/G_{i})
	\end{equation*}
	for $0\leq i<n$.
\end{definition}

The series above is also called a \textbf{central series} for a group $G$.
In general, how can we generate a central series for $ G $? One way to do this is by using some of the existing tools: 
\begin{enumerate}[(1)]
	\item The center of a group is always a normal subgroup of the group, and hence we can always find a quotient group.
	\item The Correspondence Theorem helps us to find a normal subgroup of a group corresponding to a subgroup of its quotient group.
\end{enumerate}

Let $ G $ be a group. The center $Z_1(G) =  Z(G) $ of $ G $ is a normal subgroup. Let $ Z_{2}(G) $ be the subgroup of $G$ corresponding to $ Z(G/Z_1(G)) $, i.e., $
	Z_2(G) / Z_1(G) = Z(G/Z_1(G))$. The Correspondence Theorem shows that $Z_1(G) \lhd Z_2(G) \lhd G$.  Continue this process by defining inductively:  $Z_{i}(G)$ is the subgroup of $G$ corresponding to $ Z(G/Z_{i-1}(G)) $, i.e., $Z_i(G) / Z_{i-1}(G) = Z(G/Z_{i-1}(G))$. This leads to the following definition.

\begin{definition}
 	The \textbf{ascending central series} of a group $G$ is a sequence of normal subgroups of $G$ given by  $$ \langle e \rangle \leq Z_{1}(G) \leq  Z_{2}(G) \leq  \cdots .$$
\end{definition} 
 Clearly if $Z_n (G) = G$ for some integer $n$, then $G$ is nilpotent. The converse of this statement is also true. Before proving the converse, we shall introduce the descending central series. Let $G^0 = G$ and $G^i = [G^{i-1},G]$ for $i\geq 1$. Note that $G_i$ is a subgroup of $G_{i-1}$. This is proved by using the fact that if $H\subseteq K \subseteq G$, then $[H,G]\subseteq [K,G]$. Also, we can induction to show that $G^i\lhd G$  for all $i$. It remains to show that $G_{i-1}/G_i\subseteq Z(G/G^i)$  for all $i$. We show it by using a lemma below.
 \begin{lemma} \label{lemma-for-nilpotent}
 	Let $G$ be a group. Let $X$ be a subgroup of $G$ and let $Y$ be a normal subgroup of $G$. Then $[X,G]\subseteq Y$ if and only if $XY/Y\subseteq Z(G/Y)$.
 \end{lemma}
 \begin{sketch}
 	Consider the canonical projection $G\rightarrow G/Y$.
 \end{sketch}
 Set $X = G^{i-1}$ and $Y = G^i$. Then we are done. So the following definition makes sense.
 \begin{definition}
 	The \textbf{descending central series} of a group $G$ is a sequence of normal subgroups of $G$ given by  $$ G = G^0  \geq G^1 \geq  G^2 \geq  \cdots .$$
 \end{definition}
 
 \begin{theorem} \label{thm-nilpotent}
 	 Let $G$ be a group. Then $G$ is nilpotent if and only if  $Z_n(G) = G$ for some integer $n$.
 \end{theorem}
 \begin{sketch}
  	If $Z_n(G) = G$, then $G$ is nilpotent.
  	
  	Conversely,  suppose that $G$ is nilpotent.  Let $\{e\} = N_0 \leq \cdots \leq N_k = G$ be a central series for $G$. We write $Z_i = Z_i(G)$ and show that $N_i \subseteq Z_i$ for all $i$. This certainly holds for $i = 0$. We prove by induction.  Assume that $N_{i} \subseteq Z_{i}$ for some $i\geq 0$. Since $N_{i+1}/N_i \subseteq Z(G/N_i)$, by Lemma \ref{lemma-for-nilpotent}, we have $[N_{i+1}, G] \subseteq N_{i} \subseteq Z_{i}$. So applying Lemma \ref{lemma-for-nilpotent} again gives
 	\[ N_{i+1} Z_{i}/Z_{i} \subseteq Z(G/Z_{i}) = Z_{i+1}/Z_{i}.\]
 	Hence $N_{i+1} \subseteq Z_{i+1}$.   It follows that $Z_k(G) = G$.
 \end{sketch}
 
 
 
\begin{corollary} \label{cor-nilpotent}
	Let $G$ be any group and suppose $n \geq 1$. Then $G^{n} = \{e\}$ if and only if $Z_n(G) = G$.
\end{corollary}
\begin{sketch}
	Suppose that $G^{n} = \{e\}$. Let 
	$G=G^0 \geq  G^{1}\geq \cdots \geq G^{n-1} \geq G^n = \{e\}$  be the descending central series of $G$. If we let $N_{i} = G^{n-i}$ in the proof of Theorem \ref{thm-nilpotent}, then we see that $G^{n-i}\subseteq Z_i(G)$ for all $i$. In particular, $G = G^0 = Z_n(G)$.
	
	Suppose that $Z_n(G) =G$. Let $\{e\} = Z_0(G) \leq Z_1(G) \leq \cdots \leq Z_n(G) = G$ be the ascending central series for $G$. Write $Z_i = Z_i(G)$. Since $Z_{i+1}/Z_i = Z(G/Z_i)$, it follows from Lemma \ref{lemma-for-nilpotent} that  $[Z_{i+1}, G] \subseteq Z_i$ for all $i$. Note that $G^i \subseteq Z_{k-i}$ and so $G^{n} = Z_0 = \{e\}$.
\end{sketch}



\begin{corollary}
	Subgroups and quotient groups of nilpotent groups are nilpotent.
\end{corollary}
\begin{sketch}
	Since $G$ is nilpotent, we have $G^n = \{e\}$ for some integer $n$ by Theorem \ref{thm-nilpotent} and Corollary \ref{cor-nilpotent}. 
	
	If $H \subseteq G$, then $H^n \subseteq G^n$and so $H^n = \{e\}$.
	
	To prove the result for quotient groups, let $N \lhd G$ and let $\rho : G \rightarrow G/N$ be the canonical projection. Then $\rho(G^n) = (G/N)^n$. Therefore we have $(G/N)^n = \rho(\{e\}) =\{e\}$. 
\end{sketch}

\begin{lemma}
	Let $G$ be a group. Then $G^{(k)}\subseteq G^k$ for all $k\geq 1$.
\end{lemma}
\begin{sketch}
	Use induction and note that $G^{(k+1)} = [G^{(k)},G^{(k)}] \subseteq [G^{(k)},G]  = G^{k+1}$.
\end{sketch}
This immediately yields the following consequence.
\begin{proposition} \label{prop-nil-implies-sol}
	Every nilpotent group is solvable.
\end{proposition}
\begin{sketch}
	Trivial.
\end{sketch}

\begin{definition}
	Let $G$ be a nilpotent group. The smallest integer $n$ such that $Z_n(G) = G$ is called the \textbf{class} of $G$.
\end{definition}

 
\begin{proposition} \label{prop-nilpotent-class}
	Let $G$ be a nilpotent group of class $n$. Then the following propositions holds.
	\begin{enumerate}[(i)]
		\item We have $G^n = \{e\}$. If $n \neq 0$, then $G^{n-1} \neq \{e\}$.
		\item Any central series of $G$ has at least $n+1$ terms, and there is a central series with exactly $n+1$ terms.
		\item The quotient group $G/Z(G)$ is a nilpotent group of class $n-1$. The converse is also true.
	\end{enumerate}
\end{proposition}
\begin{remark}
	We can remember the proposition above  using the following diagram:
	\begin{equation*}
		\begin{array}{ccccccccccc}
			G&=&Z_n(G) & \geq & Z_{n-1}(G) & \geq & \dots & \geq & Z_0(G) &=& \{e\} \\
			&&\shortparallel & & \cup & & & & \shortparallel&&\\
			G &= &G_n & \geq & G_{n-1} & \geq & \dots & \geq & G_0 &=& \{e\} \\
			&&\shortparallel & & \cup & & & & \shortparallel &&\\
			G &= &G^0 & \geq & G^1 & \geq & \dots & \geq & G^n &=& \{e\}
		\end{array}
	\end{equation*}
\end{remark}

\begin{proposition}
	Let $H$ be a nilpotent group of class $c$, and let $K$ be nilpotent of class $d$. Then the direct product $H \times K$ is a nilpotent group of class $e$, where $e = \max\{c, d\}$.
\end{proposition}
\begin{sketch}
	If any one of the groups is $\{e\}$, then it is trivial. So
	we may assume that $H \neq \{e\}$, $K \neq \{e\}$, and $c \ge d > 0$. 
	
	We prove by induction on $c$. If $c=1$, then $Z(H) = Z_1(H) = H$. This implies that both $H$ and $K$ are abelian. So the direct product $H \times K$ is also abelian and hence  is a nilpotent group of class $1$.  Suppose that $c > 1$. Let $G = H \times K$. By Corollary \ref{cor-center-of-direct-product}, we have $Z(G) = Z(H) \times Z(K)$. By Proposition \ref{prop-direct-product-normal-subgrp}, we obtain 
	\begin{equation*}
		G/Z(G) = (H/Z(H)) \times (K/Z(K)).
	\end{equation*}
	In view of Proposition \ref{prop-nilpotent-class}.(iii), the class of the nilpotent group $H/Z(H)$ is $c-1$. By the inductive hypothesis, $G/Z(G)$ is a nilpotent group of class $c-1$. By Proposition \ref{prop-nilpotent-class}.(iii), we conclude that $G$ is a nilpotent group of class $c$.
\end{sketch}
\begin{corollary} \label{cor-direct-prod-nil-is-nil}
	The direct product of a finite number of nilpotent groups is nilpotent.
\end{corollary}

\subsection{Examples}
\subsubsection{Triangular Matrices}
\begin{example} \label{exp-nil-sol-matrices}
	Let $\operatorname{GL}(n, \mathbb{F})$ be the group of $n \times n$ invertible matrices over a field $\mathbb{F}$ under matrix multiplication. Consider the following subgroups of $\operatorname{GL}(n, \mathbb{F})$.
	\begin{gather*}
		T = \left\{ (a_{ij}) \in \operatorname{GL}(n, \mathbb{F})\,|\, a_{ij} = 0\text{ for }j-i<0\text{ and }a_{ii}\neq 0\text{ for }1\leq i\leq n  \right\},
		\\
		U = \left\{(a_{ij}) \in T\,|\, a_{ii}\neq 0\text{ for }1\leq i\leq n\right\}.
	\end{gather*}
	Then $T$ is a solvable group, and $U$ is a nilpotent group.
\end{example}
\begin{example}
	Let $V$ be an $n$-dimensional vector space over a field $\mathbb{F}$. Let 
	$$V = V_0 \geq  V_1 \geq \cdots \geq V_n = \{0\}$$
	be a chain of subspaces such that 
	$
	\dim V_i/V_{i+1} = 1 
	$
	for $0\leq i< n$.
	For convenience, if $m > n$, then we set $V_m = \{0\}$. Let $\operatorname{GL}(V)$ be the group of vector space automorphisms of $V$. For each integer $r\geq 1$, we  define
	\begin{equation*}
		U_r = \{x \in \operatorname{GL}(V) \,|\,  (x-1)(V_i) \subseteq V_{i+r}\text{ for all }i\}.
	\end{equation*}
	where $1\in  \operatorname{GL}(V)$ is the identity map. It can be checked that:
	\begin{enumerate}[(i)]
		\item each $U_r$ is a subgroup;
		\item $U_1 \geq U_2 \geq \cdots \geq U_n = \{1\}$;
		\item $U_r$ is closed under multiplication;
		\item $U_r$ is closed under inverses.
	\end{enumerate}
	Let $x, y \in U_r$. Then 
	\begin{equation*}
		(xy-1)(V_i) = ((x-1)y)(V_i) + (y-1)(V_i)\subseteq V_{i+r}
	\end{equation*} and so $U_r$ is closed under multiplication. Assume that $x=1-X \in U_r$, where $X \in \text{End}_{\mathbb{F}}(V)$ is an vector space endomorphism with $X^n=0$. Then we have
	$$x^{-1} = (1-X)^{-1} = \sum_{k=0}^{n-1} X^k = 1+X\sum_{k=1}^{n-1}X^{k-1} \in U_r,$$
	which shows that $U_r$ is closed under inverses. 
	
	Now we claim that for all $r, s$, we have $[U_r, U_s] \subseteq U_{r+s}$. Let $x = 1-X \in U_r$ and $y = 1-Y \in U_s$. Expanding $[x,y] = xyx^{-1}y^{-1}$ as in $\text{End}_{\mathbb{F}}(V)$. To collect terms according to $X$ and $Y$, we get
	\begin{align*}
		(1-X)	(1-Y)(1-X)^{-1}(1-Y)^{-1} &= (1-X)(1-Y) \sum_{k=0}^{n-1} X^k\sum_{h=0}^{n-1}  Y^h
		\\
		&= c + p(X) + q(Y) + r(X,Y)
	\end{align*}
	where $c$ is the constant term, $P(X)$ is a sum of terms containing $X$, $q(Y)$ is a sum of terms containing $Y$, and $r(X,Y)$ is a sum of terms containing $X,Y$. It can be verified that $c = 1$, $p(X) = q(Y) = 0$. Hence we have $r(X,Y) = [x,y]-1$. Since $r(X,Y)$ have $X$ and $Y$, such terms must map $V_i$ to $V_{i+r+s}$. Therefore $[x,y]\in U_{r+s}$, and thus we have proved that $[U_r, U_s] \subseteq U_{r+s}$. As a corollary, we have $U_r \lhd U_s$ whenever $r \geq s$. To see this, note that  $xyx^{-1} = [x,y]y\in U_r$ for all $x\in U_s$, $y\in U_r$. 
	
	Set $U = U^1 = U_1$ and $U^i = U_i$ for $2\leq i\leq n$. It follows that $U_r \lhd U$ and $[U_r, U] \subseteq U_{r+1}$.  Hence the series
	\begin{equation*}
		U = U^1 \geq U^2\geq \cdots \geq U^n = \{1\}
	\end{equation*} implies that $U$ is a nilpotent group.
	
	Let
	\begin{equation*}
		T = \{x \in \operatorname{GL}(V) \,|\,  x(V_i) \subseteq V_i\text{ for all }i\}.
	\end{equation*} We have a surjective group homomorphism
	\begin{align*}
		\Phi:T&\rightarrow \bigoplus_{i=0}^{n-1} \operatorname{GL}(V_i/V_{i+1}) \cong (\mathbb{F}^\times)^n,
		\\
		x &\mapsto (x|_{V_i/V_{i+1}})_{i=0}^{n-1},
	\end{align*}
	where  $x|_{V_i/V_{i+1}}$ is the induced automorphism on each quotient $V_i/V_{i+1}$ defined by $x|_{V_i/V_{i+1}}(v+V_{i+1}) = x(v) + V_{i+1}$. We see that $\operatorname{ker}(\Phi) = U$, so $U \lhd T$ and $T/U \cong \text{im}(\Phi)$ is abelian (recall that the direct product of abelian groups is abelian). Since $T/U$ is abelian (and hence solvable) and $U$ is solvable (by Proposition \ref{prop-nil-implies-sol}), it follows from Corollary \ref{cor-solvable} that $T$ is solvable.
	
	Take an ordered basis $\mathcal{B} = \{e_1, \ldots, e_n\}$ for $V$ such that
	$$V_i = \bigoplus_{j=1  }^{n-i} \mathbb{F}e_j$$
	and write elements of $\operatorname{GL}(V)$ as matrices with respect  to $\mathcal{B}$. Then we recover Example \ref{exp-nil-sol-matrices}, where each $x|_{V_i/V_{i+1}}$ is represented by the $i$th diagonal element. 
\end{example}

\subsubsection{$p$-groups}
\begin{proposition} \label{prop-p-groups-have-nontrivial-center}
	Let $G$ be a $p$-group and let $H$ be a nontrivial normal subgroup of $G$. Then  $H \cap Z(G) \neq \{e\}$. In particular, $Z(G) \neq \{e\}$.
\end{proposition}
\begin{sketch}
	Clearly $H$ is a $p$-group. Since $H \lhd G$, the $p$-group $G$ acts on  $H$ by conjugation. By Fixed point lemma (Lemma \ref{lemma-fixed-point-lemma}), $H_G$ contains an element $x\in H$ other than $e$. By the definition of $H_G$, we have $x = g^{-1}xg$ for all $g \in G$. Thus $x\in Z(G)$ and so $ H \cap Z(G) \neq \{e\}$. The last assertion is obtained by setting $G=H$.
\end{sketch}
\begin{proposition} \label{prop-p-group-is-nil}
	Let $p$ be a prime. Then any $p$-group is nilpotent.
\end{proposition}
\begin{sketch}
	Any $p$-group $G$ and all its nontrivial quotient groups are $p$-groups and thus have nontrivial centers by Proposition \ref{prop-p-groups-have-nontrivial-center}. If $G \neq Z_i(G)$, then $Z_i(G)$ is strictly contained in $Z_{i+1}(G)$.  Since $G$ is finite, $Z_m( G)$ must be $G$ for some $m$. 
\end{sketch}

\subsubsection{Symmetric Groups}
\begin{proposition}
	The symmetric group $S_n$ is solvable for $n=1,2,3,4$.
\end{proposition}
\begin{sketch} It is trivial for $S_1$.
	

	The group $S_2$ is solvable because it is an abelian group.
	
	The group $S_3$ is solvable because we have the following series
	\begin{equation*}
		1 \lhd A_3 \lhd S_3.
	\end{equation*}
	Clearly  $S_3/A_3\cong \mathbb{Z}_2$ and $A_3/\{e\} \cong \mathbb{Z}_3$. Thus they are abelian.  
	
	The group $S_4$ is solvable because we have the following subnormal series
	\begin{equation*}
		\{e\} \lhd V_4 \lhd A_4\lhd S_4
	\end{equation*}
	where $V_4$ is the Klein $4$-group.
	
	\url{https://math.stackexchange.com/questions/120429/prove-that-s-3-and-s-4-are-solvable-groups}
\end{sketch}
\begin{proposition}
	The symmetric group $S_n$ is not solvable for $n\geq 5$.
\end{proposition}
\begin{sketch}
	Note that $A_n$ is a nonabelian and simple group. Hence $[A_n,A_n] = A_n$ and so $A_n$ is not solvable by Theorem \ref{thm-solvable}. Then we use Corollary \ref{cor-solvable} to obtain the conclusion.
\end{sketch}

\subsection{Characterization of Finite Nilpotent Groups}



\begin{lemma} \label{lemma-H=N_G(H)}
	Let $S$ be a Sylow $p$-subgroup of a group $G$. If a subgroup $H$ of $G$ contains $N_G(S)$, then we have $H = N_G(H)$.
\end{lemma}
\begin{sketch}
	By assumption, $S \subseteq N_G(S) \subseteq H$. Note that $H$ is a normal subgroup of $N_G(H)$. By applying Frattini's argument (Lemma \ref{lemma-G=N_G(S)H}) to $N_G(H)$, we  get
	$$N_G(H) = N_{N_G(H)}(S)H \subseteq \langle N_G(S), H \rangle = H.$$
	Clearly, $H \subseteq N_G(H)$, and so we have $H = N_G(H)$.
\end{sketch}

The following theorem gives some of the main properties which characterize  finite nilpotent groups. We include Theorem \ref{thm-nilpotent} and Corollary \ref{cor-nilpotent} for convenience.
\begin{theorem}
	Let $G$ be a finite group. Then the following are equivalent.
	\begin{enumerate}[(1)]
		\item There exists an integer $n$ such that $Z_n(G)=G$.
		\item There is at least one central series of $G$.
		\item There exists an integer $n$ such that $G^n=\{e\}$.
		\item The normalizer of any proper subgroup $H$ is strictly larger than $H$, i.e., $N_G(H) \neq  H$.
		\item Any maximal subgroup is normal.
		\item For any prime number $p$, every Sylow $p$-subgroup is normal.
		\item The group $G$ is a direct product of Sylow subgroups.
		\item The group $G$ is a direct product of groups of prime power order.
		%\item For any set $\pi$ of prime numbers, the group $G$ is a direct product of a $\pi$-group and a $\pi'$-group.
		%\item If $x$ and $y$ are two elements of $G$ having relatively prime orders, then $x$ commutes with $y$.
	\end{enumerate}
\end{theorem}
\begin{sketch} We have already established that (1), (2) and (3) are equivalent.
	
 (2) $\Rightarrow$ (4) Let 
 \begin{equation*}
 	\{e\}= G_0 \leq G_1\leq \cdots \leq G_n = G
 \end{equation*} be a central series for $G$.   Let $H$ be a proper subgroup of $G$. Let $k$ be the integer such that $G_{k+1} \not\subseteq H$ and $G_{k} \subseteq H$. Such integer must exist since $H$ contains $G_0$. Then
 \begin{equation*}
 	[G_{k+1}, H] \subseteq [G_{k+1}, G] \subseteq G_{k} \subseteq H.
 \end{equation*}
 So $G_{k+1} \subseteq N_G(H)$ and hence we have $N_G(H) \neq H$.
 
	(4) $\Rightarrow$ (5) Let $M$ be a maximal subgroup of $G$. Since $M$ is proper subgroup, we have $M<N_G(M)$. By the definition of maximal subgroups, we get $N_G(M) = G$. Hence $M\lhd G$.
	
	(5) $\Rightarrow$ (6) Suppose on the contrary that a Sylow $p$-subgroup $P$ of $G$ is not normal. Then $N_G(P)< G$. In view of Proposition \ref{prop-maximal-grp-exist}, we can take a maximal subgroup $M$  which contains $N_G(P)$. By Lemma \ref{lemma-H=N_G(H)}, we have $N_G(M) = M$. Hence $M$ is not normal, since $N_G(M)<G$. This is a contradiction.
	
	(6) $\Rightarrow$ (7) By Corollary \ref{cor-direct-product-isom-to-product-subgrps}.
	
	(7) $\Rightarrow$ (8) Trivial.
	
	(8) $\Rightarrow$ (1) Any group of prime power order is nilpotent by Proposition \ref{prop-p-group-is-nil}. The result follows from Corollary \ref{cor-direct-prod-nil-is-nil}.
	%(7) $\Rightarrow$ (9) Let $H$ be the product of $S_p$-subgroups for $p \in \pi$, and let $K$ be the product of all other Sylow subgroups of $G$. Then, clearly we have $G = H \times K$, and $H$ is a $\pi$-group and $K$ is a $\pi'$-group.
	%(9) $\Rightarrow$ (10) Let $\pi$ be the set of prime numbers which divide the order of the element $x$. Then, by (9), $G$ is a direct product of a $\pi$-group $H$ and a $\pi'$-group $K$. Since $x$ is a $\pi$-element, its image in $G/H$ is a $\pi$-element. But, $G/H \cong K$ is a $\pi'$-group. This implies that $x \in H$. The element $y$ is a $\pi'$-element. So, we have $y \in K$. This proves that $x$ commutes with $y$.
	%(10) $\Rightarrow$ (6) Let $p$ be any prime number and $S$ be an $S_p$-subgroup of $G$. If $q$ is a prime number different from $p$, any element of an $S_q$-subgroup centralizes $S$. Thus, the normalizer of $S$ contains $S_q$-subgroups of $G$ for any prime number $q$. This implies that $N_G(S) = G$. So, (6) holds.
\end{sketch}



 
