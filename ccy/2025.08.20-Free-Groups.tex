\section{Free Groups}
\subsection{Definition}
\begin{definition} \label{def-free-group}
	Let $X$ be a set and let $F$ be a group. We call $F$ a \textbf{free group} on $X$ if $F$ is a free object on $X$ in the category of groups, i.e., there is a set function $u:X\rightarrow F$ such that the pair $(F,u)$ satisfies the following universal property: for every group $G$ and a set function $f:X\rightarrow G$, there is a unique group homomorphism $f':F\rightarrow G$ such that $f'\circ u = f$, as shown in the following commutative diagram.
	\[\begin{tikzcd}
		X && F && F \\
		\\
		&& G && G
		\arrow["u", from=1-1, to=1-3]
		\arrow["f"', from=1-1, to=3-3]
		\arrow["{f'}", dashed, from=1-3, to=3-3]
		\arrow["{f'}", dashed, from=1-5, to=3-5]
	\end{tikzcd}\]
\end{definition}
\begin{remark}
	We observe that if $(F,u:X\to F)$ satisfies the universal property and $\phi:F\to H$ is a group isomorphism, then $(H,\phi\circ u:X\to H)$ also satisfies the universal property. So $H$ is  free on $X$.
\end{remark}
\begin{proposition}
	Let $u:X\rightarrow F$ be the function defined in Definition \ref{def-free-group}. Then the following propositions hold.
	\begin{enumerate}[(i)]
		\item The elements $u(x)$ ($x\in X$) generate $F$.
		\item The function $u$ is injective.
	\end{enumerate}
\end{proposition}
\begin{sketch}
	(i) Let $G$ be the subgroup of $F$ generated by the elements $u(x)$ ($x\in X$). Let $f:X\rightarrow G$ be the function defined by $f(x) = u(x)$. Then there exists a unique group homomorphism $f':F\rightarrow G$ such that $f'\circ u = f$. Let $\iota:G\rightarrow F$ be the inclusion mapping. Consider the function $\iota\circ f$.  By the universal property, there exists a unique group homomorphism $\varphi:F\rightarrow F$ such that $f''\circ u = \iota\circ f$. Note that left multiplication of $\iota$ on the equality $f'\circ u = f$ yields $(\iota\circ f') \circ u = \iota\circ f$. By the uniqueness, we obtain $f'' =\iota\circ f'$. On the other hand, we see that $1\circ u = \iota\circ f$, where $1:F\rightarrow F$ is the identity mapping. Hence we get $f''= 1$. Thus $\iota\circ f' = 1$. So $\iota$ has a right inverse. This implies that $\iota$ is surjective (recall that surjectivity $\Leftrightarrow$ having right inverse), whence  $F = \iota(G) = G = \langle u(x)\,|\, x\in X\rangle$.
	
	(ii) Let $x,y\in X$ be distinct elements, i.e., $x\neq y$. Let $f:X\rightarrow \mathbb{Z}_2$ be defined by $f(x) = 0$, $f(z) = 1$ for all $z\in X\setminus\{x\}$. By the universal property, there is a unique $f':F\rightarrow \mathbb{Z}_2$ such that $f'\circ u = f$. In particular, $f'(u(x)) = (f'\circ u)(x) = f(x) = 0$ and $f'(u(y)) =  f(y) = 1$. Hence $f'(u(x)) \neq f'(u(y))$. Since $f'$ is well-defined, we get $u(x) \neq u(y)$. Therefore  $u$ is injective.
\end{sketch}
\begin{remark}
	Since $u$ is an injective function, we shall identify $X$ as a subset of $F$. Simply speaking, one usually omits $u$ and writes $x$ for its image $u(x)$. 
\end{remark}
\subsection{Three Ways of Constructing Free Groups}
\begin{theorem}
	For any set $X$, there exists a unique (up to isomorphism) free group on $X$.
\end{theorem}
The uniqueness follows from Proposition \ref{prop-free}. Now we prove the existence in three different ways.
\subsubsection{Groups of Words: a Beginner's Favourite}
Let $X$ be a set. We introduce the set of ``words" on $X$ as
\begin{equation*}
	W(X) \coloneq \bigcup_{n\geq 0} X^n.
\end{equation*}
By convention, we assume that $X^0 \coloneq \{1\}$. Here $1$ is called the \textbf{empty word}.
An element of $X^n\subseteq W(X)$ is called a \textbf{word of length} $n$ over $X$.  To simplify the notation, we write $x_1\cdots x_n$ for the $n$-tuple $(x_1,\dots, x_n)$ in $X^n$. Now we define a binary operation (called the \textbf{concatenation of words}) on $W(X)$ as follows.
\begin{align*}
	W(X)\times W(X)&\to W(X);
	\\
	(x_1\cdots x_n,y_1\cdots y_m) &\mapsto (x_1\cdots x_ny_1\cdots y_m)
\end{align*}
for $n,m\geq 0$. Clearly this operation is associative. To make $W(X)$ a monoid, we assume that the empty word $1$ is the identity element.
\begin{proposition} \label{prop-set-func-to-monoid-homo}
	Let $X$ be a set and let $M$ be a monoid. Then any set function $f:X\to M$ can be extended uniquely to a monoid homomorphism $\hat{f}:W(X)\to M$.
\end{proposition}
\begin{sketch}
	Let $f:X\to M$ be a function. To prove the existence, we define $g:W(X) \to M$ by $g(1) = 1$ and $g(x_1\cdots x_n) = f(x_1)\cdots f(x_n)$ for $n\geq 1$. Clearly $g$ is a monoid homomorphism. Suppose that there is a monoid homomorphism $g':W(X)\to M$ such that $g'(x) = f(x)$ for all $x\in X$. Then we have $g'(1)=1$ by the definition of monoid homomorphism. For $n\geq 1$, we have
	\begin{equation*}
		g'(x_1 \cdots x_n) = g'(x_1) \cdots g(x_n) = f(x_1) \cdots f(x_n) =g(x_1\cdots x_n)
	\end{equation*}
	Hence $g' = g$.
\end{sketch}
We now set $X^{\pm} = X \bigcup \hat{X}$, where $X\to \hat{X}$ is a bijection and $X\cap \hat{X} = \emptyset$. We have two natural inclusions $X\to X^{\pm}$. Every element $x$ in $X$ (resp. $\hat{X}$) has a corresponding element in $\hat{X}$ (resp. $X$), denoted by $x^{-1}$. Hence we have an involution $X^{\pm}\to X^{\pm}; x\mapsto x^{-1}$.

Let $m$ and $m'$ be two words on $X^{\pm}$, we say that $m$ is an \textbf{elementary contraction} of $m'$, and we write $mCm'$, if there exist $n_1,n_2\in W(X^{\pm})$ and $x\in X^{\pm}$ such that $m = n_1n_2$ and $m' = n_1xx^{-1}n_2$. Let $R$ be the equivalence relation on $W(X^{\pm})$ generated by $C$. We denote the set of equivalence classes with respect to $R$ by
\begin{equation*}
	F(X) \coloneq W(X^{\pm})/R.
\end{equation*}
Note that the concatenation induces a composition $F(X)\times F(X) \to F(X)$ via $[m][n]\coloneq[mn]$. To see that this composition is well-defined. Let $m,m',n,n'\in M$ be such that $mRm'$ and $nRn'$. Since $R$ is generated by $C$, there is a sequence of elements $m_1,\dots, m_r\in M$ for which $m_1=m$, $m_r = m'$ and either $m_iCm_{i+1}$ or  $m_{i+1}Cm_i$ for all $1\leq i<r$. In either case, we get $m_inRm_{i+1}n$ for all $1\leq i<r$. By the transitivity of $R$, we have $mnRm'n$. By a similar argument, we can show that $m'nRm'n'$. So we get $mnRm'n'$, proving that the composition is well-defined. 

The associativity of this composition is inherited from the associativity of the operation on $W(X^{\pm})$. Clearly $[1]$ is the identity element. By definition of $C$, we have $[x][x^{-1}] = [xx^{-1}] = [1]$ for all $x\in X^{\pm}$. Since $X^{\pm}$ generates $W(X^{\pm})$, the elements $[x]$ ($x\in X^{\pm}$) generate $F(X)$. Hence every element in $F(X)$ has an inverse, and thus making $F(X)$ a group.

\begin{proposition}
	The group $F(X)$ defined above is free on the set $X$.
\end{proposition}
\begin{sketch}
	Let $u:X\to F(X)$ be defined by $u(x) = [x]$. We show that the pair $(F(X),u)$ satisfies the universal property. Let $G$ be a group and let $f:X\to G$ be a set function.  First we extend $f$ to a function $f^{\pm}:X^{\pm}\to G$ by setting $f^{\pm}(x^{-1}) = f(x)^{-1}$ for $x\in X$. By Proposition \ref{prop-set-func-to-monoid-homo}, $f^{\pm}$ can be extended uniquely to a monoid homomorphism $\hat{f}:W(X^{\pm})\to G$. Now we define $f':F(X)\to G$ by 
	\begin{equation*}
		f'([m]) = \hat{f}(m)
	\end{equation*}
	for all $m\in W(X^{\pm})$. Clearly $f'\circ u = f$. We now verify that $f'$ is well-defined. Let $m,m'\in F(X)$ be such that $mRm'$.  It suffices to show that if $m,m'\in F(X)$ satisfy $mCm'$, then $\hat{f}(m) = \hat{f}(m')$. Since $mCm'$, we have $m' = n_1 x x^{-1} n_2$ and $m = n_1 n_2$ for some $x \in X^{\pm}$ and $n_1, n_2\in W(X^{\pm})$. Hence
	\begin{align*}
		f'([m']) &= \hat{f}(m')
		= \hat{f}(n_1 x x^{-1} n_2)
		=\hat{f}(n_1 )\hat{f}(x)\hat{f}( x^{-1} )\hat{f}(n_2)
		\\
		&= \hat{f}(n_1 )f(x)f(x)^{-1}\hat{f}(n_2)
		= \hat{f}(n_1 )\hat{f}(n_2)
		= \hat{f}(n_1 n_2)
		= \hat{f}(m)
		= f'([m]).
	\end{align*}
	Since $F(X)$ is a group,  for $[m],[m']\in F(X)$, we have 
	\begin{equation*}
		f'([m][m]') = f'([mm]') = \hat{f}(mm') = \hat{f}(m)\hat{f}(m') = f'([m])f'([m]').
	\end{equation*}
	Thus $f'$ is a group homomorphism.
	
	The homomorphism $f'$ is unique  because the elements $[x]$ with $x \in X$ generate the group $F(X)$. 
\end{sketch}


\begin{definition}
	A word in $W(X^{\pm})$ is said to be \textbf{reduced} if it is the empty word or of the form $x_1 \cdots x_n$ with $x_i \in X^{\pm}$ for $1 \leq i \leq n$ and $x_{i+1} \neq x_i^{-1}$ for $1 \leq i < n$.
\end{definition}
Note that a word of length $1$ is trivially reduced.  It is clear that every word in $W(X^{\pm})$ is equivalent to a reduced word. Now we establish the following result which gives us a natural representative for each equivalence class on $W(X^{\pm})$.
\begin{theorem} \label{thm-bijection-red-words-and-free-group}
	The function 
	\begin{align*}
		f:W(X^{\pm}) &\to F_X, 
		\\
		m &\mapsto [m]
	\end{align*}  induces a bijection between the set of reduced words in $W(X^\pm)$ and $F(X)$.
\end{theorem}
\begin{sketch}
	Clearly $f$ is surjective. To prove injectivity, for each $x \in X^{\pm}$, we define $L_x:W(X^\pm) \to W(X^\pm)$ by
	\begin{equation*}
		L_x(m) = \begin{cases}
			xm &\text{if $m$ does not start with $x^{-1}$},
			\\
			n & \text{if $m = x^{-1}n$ for a unique word $n$}.
		\end{cases}
	\end{equation*}  
	Let  $\Omega \subseteq \operatorname{Words}(X^{\pm})$ denote the  set of reduced words. Observe that:
	\begin{itemize}
		\item if $m\in \Omega$, then $L_x(m)\in \Omega$.
		\item $L_{x^{-1}}(L_x(m)) = m$ for any $m\in\Omega$. Indeed, if $m = x^{-1}n$ we have $L_{x^{-1}}(L_x(m)) = L_{x^{-1}}(n) = x^{-1}n = m$ since $n$ does not start with $x$; if $m$ does not start with $x^{-1}$ we have $L_{x^{-1}}(L_x(m)) = L_{x^{-1}}(xm) = m$.
	\end{itemize}    
	By the observations above, the set $S_\Omega\coloneq\{L_x:\Omega \to \Omega\mid x\in X^\pm\}$ forms a group under function composition. This induces a set function
	\begin{align*}
		g:X^{\pm} &\to S_\Omega,
		\\
		x & \mapsto L_x.
	\end{align*}
	By the universal property, there exists a group homomorphism $g': F(X) \to S_{\Omega}$ such that $g'([x])=L_x$ for all $x \in X$. Since $[x]^{-1} = [x^{-1}]$ and $(L_x)^{-1} = L_{x^{-1}}$ for all $x\in X$, we get $g'([x])=L_x$ for all $x \in X^\pm$.
	
	Let $m = x_1 \cdots x_r \in \Omega$. We have $[m] = [x_1] \cdots [x_r]$ in $F(X)$, and thus $g'([m]) = L_{x_1} \circ \dots \circ L_{x_r}$. Since  $x_i \neq x_{i-1}^{-1}$ for $1 < i \le r$, we obtain
	\begin{equation*}
		g'([m])(1) = (L_{x_1} \circ \dots \circ L_{x_r})(1) = (L_{x_1} \circ \dots \circ L_{x_{r-1}})(x_r) = \cdots = x_1 \cdots x_r = m.
	\end{equation*}
	Suppose that $f(m) = f(m')$ where $m,m'\in\Omega$. Then $[m] = [m']$. Applying $g'$ on this equation gives $g'([m]) = g'([m'])$. In particular, we have $m = g'([m])(1) = g'([m'])(1) = m'$.  
\end{sketch}
\begin{remark}
	One can also define $F(X)$ as the set of all reduced words. The operation on $F(X)$ is roughly defined as concatenation followed by discarding the pair $xx^{-1}$ in words if any.
\end{remark}
\begin{example}
	If $X = \emptyset$, we have $F(X) = \{1\}$.
	
	If $X = \{x\}$, then the set of reduced words on $X^{\pm}$ is $\{x^n\mid n\in\mathbb{Z}\}$. By Theorem \ref{thm-bijection-red-words-and-free-group}, we have an isomorphism $\mathbb{Z}\to F(X)$, $n\mapsto [x^n]$.
	
	If $|X|\geq 2$, the group $F(X)$ is nonabelian because for any distinct elements $a,b\in X$, the reduced words $ab$ and $ba$ are distinct.
\end{example}

 

\subsubsection{Construction from Term Algebras: a Viewpoint from Universal Algebra}
\begin{definition}
	Let $X$ be a set. A \textbf{group-theoretic term} is a finite string of symbols from $X$ using formal group operations $\cdot$, $e$ and $\iota$, where  parentheses are introduced among the symbols.
\end{definition}

\begin{definition}
 The set of all \textbf{group-theoretic terms} in the elements of $X$ under the formal group operations $\cdot$, $e$, $\iota$ is a set $T$ satisfying the following: There are some functions
 \[
 \text{symb}_T: X \to T, \quad \cdot_T: T\times T \to T, \quad e_T: T^0 \to T, \quad \text{and} \quad \iota_T: T \to T,
 \]
 where $T^0$ is a set of one distinguished element, such that
\begin{enumerate}[(i)]
	\item  each of these maps is one-to-one;
	\item  their images are disjoint and $T$ is the union of those images; 
	\item $T$ is generated by $\text{symb}_T(X)$ under the operations $\cdot_T$, $e_T$, and $\iota_T$;
	\item $T$ has no proper subset which contains $\text{symb}_T(X)$ and is closed under those operations.
\end{enumerate}
\end{definition}

Now we construct a free group on a given set $X$. Let $T$ be the set of all group-theoretic terms in the elements of $X$ under $\cdot$, $\iota$, $e$. Let $\sim$ be the \textbf{smallest} relation on $T$ that satisfy:

\noindent(Group axioms)
\begin{align}
	(\forall p, q, r \in T) \quad &(p \cdot q) \cdot r \sim p \cdot (q \cdot r),  \label{eq-G1} \tag{G1}\\
	(\forall p \in T) \quad &(p \cdot e \sim p) \land (e \cdot p \sim p), \tag{G2}\\
	(\forall p \in T) \quad &(p \cdot p^{-1} \sim e) \land (p^{-1} \cdot p \sim e).  \label{eq-G3}\tag{G3}
\end{align}
(Compatibility)
\begin{align}
	(\forall p, p', q \in T) \quad &(p \sim p') \implies ((p \cdot q \sim p' \cdot q) \land (q \cdot p \sim q \cdot p')), \label{eq-C1}\tag{C1}
	\\
	(\forall p,p'\in T) \quad &(p\sim p') \implies (p^{-1} \sim p'^{-1}). \label{eq-C2}\tag{C2}
\end{align}
(Equivalence relations)
\begin{align}
	(\forall p \in T) \quad &p \sim p, \label{eq-R1} \tag{R1} \\
	(\forall p, q \in T) \quad &(p \sim q) \implies (q \sim p), \tag{R2}\\
	(\forall p, q, r \in T) \quad &((p \sim q) \land (q \sim r)) \implies (p \sim r).\label{eq-R3} \tag{R3}
\end{align}
This smallest relation on $T$ can be constructed by forming the intersection of all relations on $T$ satisfying the conditions above. By (\ref{eq-R1})--(\ref{eq-R3}), the relation $\sim$ is an equivalence relation. Let $F$ be the equivalence classes of $\sim$, i.e.,
\begin{equation*}
	F = \frac{T}{\sim} =\{[p]\,|\, p\in T\}.
\end{equation*}
Let $u:X\rightarrow F$ be the function defined by
\begin{equation*}
	u(x) = [x].
\end{equation*}
Define operation $\cdot$, ${}^{-1}$ and $e$ on $F$ by
\begin{equation*}
	[p]\cdot [q] = [p\cdot q].
\end{equation*}
Then the operation is well-defined by (\ref{eq-WD}). It can be verified that $[e]$ is the identity and $[p]^{-1} = [p^{-1}]$ for all $p\in T$. It follows that $F$ is a group by (\ref{eq-G1})--(\ref{eq-G3}). Now we claim that $(F,u)$ satisfies the universal property.



\subsubsection{Crazy Construction from Direct Products: Only Serge Lang's Favourite}
\subsection{Group Presentations} \label{sec-group-presentations}
Let $X$ be a set and let $F(X)$ be the free group on $X$. First we note that for any subset $R$ of $F(X)$, there is a smallest normal subgroup of $F(X)$ containing $R$, namely $\bigcup_{x\in F(X)} xRx^{-1}$. We denote it by $\langle\langle R\rangle\rangle$.
\begin{definition}
	Let $X$ be a set and let $R$ be a subset of $F(X)$. The quotient group $$\langle X\mid R\rangle\coloneq F(X)/\langle\langle R\rangle\rangle$$ is called the \textbf{group generated by} $X$ \textbf{and the relations} $R$.  Let $G$ be a group. Then $\langle X\mid R\rangle$ is called a presentation of $G$ if $G\cong\langle X\mid R\rangle$. The elements in $X$ are called \textbf{generators} and the elements of $R$ are called \textbf{relators} of the presentation.
\end{definition} 
\begin{remark}
	\begin{enumerate}[(1)]
		\item If $G$ is generated by $X$ and $R$, then we write $G = \langle X\mid R\rangle$. Some authors may abuse the notation by referring $X$ and $R$ as sets of elements in $G$ (and hence use the equality).
		\item It is sometimes convenient to replace $R$ in $\langle X\mid R\rangle$ by the set of equations $\{r=1\mid r\in R\}$, called \textbf{defining relations} for $G$. A defining relation may even take the form
		$u=v$ where $u,v\in F(X)$, corresponding to the defining relator $uv^{-1}$.
		\item When $X=\{x_1,\dots, x_n\}$ and $R = \{r_1,\dots, r_m\}$, we write $\langle X\mid R\rangle = \langle x_1,\dots, x_n \mid r_1=\cdots = r_m=1 \rangle$.
	\end{enumerate}
\end{remark}
\begin{lemma} \label{lemma-for-von-Dyck}
	Let $F,G$ and $H$ be groups. Let $\nu :F\to G$ and $\alpha:F\to H$ be homomorphisms such that $\Im \nu = G$ and $\Ker \nu \subseteq \Ker \alpha$. Then there exists a homomorphism $\alpha':G\to H$ such that $ \alpha'\circ \nu = \alpha$, i.e., the following diagram commutes.
	% https://q.uiver.app/#q=WzAsMyxbMCwxLCJGIl0sWzIsMCwiRyJdLFsyLDIsIkgiXSxbMCwxLCJcXG51Il0sWzEsMiwiXFxhbHBoYSciLDAseyJzdHlsZSI6eyJib2R5Ijp7Im5hbWUiOiJkYXNoZWQifX19XSxbMCwyLCJcXGFscGhhIiwyXV0=
	\[\begin{tikzcd}
		&& G \\
		F \\
		&& H
		\arrow["{\alpha'}", dashed, from=1-3, to=3-3]
		\arrow["\nu", from=2-1, to=1-3]
		\arrow["\alpha"', from=2-1, to=3-3]
	\end{tikzcd}\]
\end{lemma}
\begin{sketch}
	Define $\alpha':G\to H$ by 
	\begin{equation*}
		\alpha'(g) = \alpha(f)
	\end{equation*}
	where $f \in F$ such that $\nu(f) = g$. Such an $f$ exists since $\Im \nu  = G$. We show that $\alpha'$ is well-defined.  Let $g_1, g_2\in G$ with $g_1 = g_2$. Then there is $f_1,f_2\in F$ such that $\nu(f_1) = g_1$ and $\nu(f_2) = g_2$. Hence $\nu(f_1) = \nu(f_2)$ and $f_1f_2^{-1}\in\Ker \nu\subseteq \Ker \alpha$. Thus we have $\alpha(f_1) = \alpha(f_2)$, i.e., $\alpha'(g_1) = \alpha'(g_2)$. 
	
	Now let $g_1, g_2 \in G$ and $f_1, f_2 \in F$ such that $\nu(f_1) = g_1$ and $\nu(f_2) = g_2$. Since $\nu$ is a homomorphism, we have $\nu(f_1f_2) = \nu(f_1)\nu(f_2) = g_1g_2$, and
	\begin{equation*}
		\alpha'(g_1g_2) = \alpha(f_1f_2) =  \alpha(f_1)\alpha(f_2) = 	\alpha'(g_1) \alpha'(g_2).
	\end{equation*}
	This shows that $\alpha'$ is a homomorphism.
	
	For any $f\in F$, we get $(\alpha'\circ \nu)(f) = \alpha'(\nu(f)) = \alpha(f)$ and hence $\alpha'\circ \nu = \alpha$.
\end{sketch}
\begin{theorem}[von Dyck's Theorem] \label{thm-Van-Dyck}
	Let $G$ and $H$ be groups with presentations $\langle X\mid R\rangle$ and $\langle X\mid S \rangle$ respectively such that $R\subseteq S$. Then there is an epimorphism $G\to H$ 
\end{theorem}
\begin{sketch}
	Without loss of generality, assume that $G = \langle X\mid R\rangle$ and $H =\langle X\mid S\rangle$. Let $\nu:F(X)\to G$ and $\alpha:F(X)\to  H$ be the canonical projections. Since $\Im \nu = G$ and $\Ker \nu = \langle\langle R\rangle\rangle \subseteq \langle\rangle\langle S\rangle\rangle = \Ker \alpha$, it follows from Lemma \ref{lemma-for-von-Dyck} that there is a homomorphism $\alpha':G\to H$ with $\alpha'\circ\nu = \alpha$. To prove the surjectivity of $\alpha'$, note that 
	\begin{equation*}
		H = \Im \alpha = \Im (\alpha'\nu) = \alpha'(\Im \nu) = \alpha'(G) = \Im \alpha'. \qedhere
	\end{equation*}
\end{sketch}
\begin{remark}
	By abusing notations we can write $x$ for $\nu(x)$ in $G$ (resp. $\alpha(x)$ in $H$). Then we see that $\alpha'(x) =x$ for all $x\in X$.  
\end{remark}
\subsection{Nielsen-Schreier Theorem}
\begin{theorem}[Nielsen-Schreier Theorem]
	If $H$ is a subgroup of a free group $G$, then $H$ is free.
\end{theorem}



\paragraph{Main References.} \cite{Lang2002,Bergman2015,Johnson1997,Ribes2010}