\section{Free Groups}
\subsection{Definition}
\begin{definition} \label{def-free-group}
	Let $X$ be a set and let $F$ be a group. We call $F$ a \textbf{free group} on $X$ if $F$ is a free object on $X$ in the category of groups, i.e., there is a set function $u:X\rightarrow F$ such that the pair $(F,u)$ satisfies the following universal property: for every group $G$ and a set function $f:X\rightarrow G$, there is a unique group homomorphism $f':F\rightarrow G$ such that $f'\circ u = f$, as shown in the following commutative diagram.
	\[\begin{tikzcd}
		X && F && F \\
		\\
		&& G && G
		\arrow["u", from=1-1, to=1-3]
		\arrow["f"', from=1-1, to=3-3]
		\arrow["{f'}", dashed, from=1-3, to=3-3]
		\arrow["{f'}", dashed, from=1-5, to=3-5]
	\end{tikzcd}\]
\end{definition}
\begin{proposition}
	Let $u:X\rightarrow F$ be the function defined in Definition \ref{def-free-group}. Then the following propositions hold.
	\begin{enumerate}[(i)]
		\item The elements $u(x)$ ($x\in X$) generate $F$.
		\item The function $u$ is injective.
	\end{enumerate}
\end{proposition}
\begin{sketch}
	(i) Let $G$ be the subgroup of $F$ generated by the elements $u(x)$ ($x\in X$). Let $f:X\rightarrow G$ be the function defined by $f(x) = u(x)$. Then there exists a unique group homomorphism $f':F\rightarrow G$ such that $f'\circ u = f$. Let $\iota:G\rightarrow F$ be the inclusion mapping. Consider the function $\iota\circ f$.  By the universal property, there exists a unique group homomorphism $\varphi:F\rightarrow F$ such that $f''\circ u = \iota\circ f$. Note that left multiplication of $\iota$ on the equality $f'\circ u = f$ yields $(\iota\circ f') \circ u = \iota\circ f$. By the uniqueness, we obtain $f'' =\iota\circ f'$. On the other hand, we see that $1\circ u = \iota\circ f$, where $1:F\rightarrow F$ is the identity mapping. Hence we get $f''= 1$. Thus $\iota\circ f' = 1$. So $\iota$ has a right inverse. This implies that $\iota$ is surjective (recall that surjectivity $\Leftrightarrow$ having right inverse), whence  $F = \iota(G) = G = \langle u(x)\,|\, x\in X\rangle$.
	
	(ii) Let $x,y\in X$ be distinct elements, i.e., $x\neq y$. Let $f:X\rightarrow \mathbb{Z}_2$ be defined by $f(x) = 0$, $f(z) = 1$ for all $z\in X\setminus\{x\}$. By the universal property, there is a unique $f':F\rightarrow \mathbb{Z}_2$ such that $f'\circ u = f$. In particular, $f'(u(x)) = (f'\circ u)(x) = f(x) = 0$ and $f'(u(y)) =  f(y) = 1$. Hence $f'(u(x)) \neq f'(u(y))$. Since $f'$ is well-defined, we get $u(x) \neq u(y)$. Therefore  $u$ is injective.
\end{sketch}
\begin{remark}
	Since $u$ is an injective function, we shall identify $X$ as a subset of $F$. Simply speaking, one usually omits $u$ and writes $x$ for its image $u(x)$. 
\end{remark}
\subsection{Three Ways of Constructing Free Groups}
\begin{theorem}
	For any set $X$, there exists a unique (up to isomorphism) free group on $X$.
\end{theorem}
The uniqueness follows from Proposition \ref{prop-free}. Now we prove the existence in three different ways.
\subsubsection{Groups of Words: a Beginner's Favourite}
\subsubsection{Construction from Equivalence Classes: a Logician's Favourite}
\begin{definition}
	Let $X$ be a set. A \textbf{group-theoretic term} is a finite string of symbols from $X$ using formal group operations $\cdot$, ${}^{-1}$ and $e$, where  parentheses are introduced
	among the symbols.
\end{definition}
This definition is actually motivated from the set of all group-theoretic terms, which makes something more precise.
\begin{definition}
 The set of all \textbf{group-theoretic terms} in the elements of $X$ under the formal group operations $\cdot$, $\iota$, $e$ is a set $T$ satisfying the following: There are some functions
 \[
 \text{symb}_T: X \to T, \quad \cdot_T: T\times T \to T, \quad \iota_T: T \to T, \quad \text{and} \quad e_T: T^0 \to T,
 \]
 where $T^0$ is a set of one distinguished element, such that
\begin{enumerate}[(i)]
	\item  each of these maps is one-to-one;
	\item  their images are disjoint and $T$ is the union of those images; 
	\item $T$ is generated by $\text{symb}_T(X)$ under the operations $\cdot_T$, $\iota_T$, and $e_T$;
	\item $T$ has no proper subset which contains $\text{symb}_T(X)$ and is closed under those operations.
\end{enumerate}
\end{definition}

Now we construct a free group on a given set $X$. Let $T$ be the set of all group-theoretic terms in the elements of $X$ under $\cdot$, $\iota$, $e$. Let $\sim$ be the \textbf{least} relation on $T$ that satisfy:
(Group Axioms)
\begin{align}
	(\forall p, q, r \in T) \quad &(p \cdot q) \cdot r \sim p \cdot (q \cdot r),  \label{eq-G1} \tag{G1}\\
	(\forall p \in T) \quad &(p \cdot e \sim p) \land (e \cdot p \sim p), \tag{G2}\\
	(\forall p \in T) \quad &(p \cdot p^{-1} \sim e) \land (p^{-1} \cdot p \sim e).  \label{eq-G3}\tag{G3}
\end{align}
(Well-definedness)
\begin{align}
	(\forall p, p', q \in T) \quad &(p \sim p') \implies ((p \cdot q \sim p' \cdot q) \land (q \cdot p \sim q \cdot p')) \label{eq-WD}\tag{WD}
\end{align}
(Equivalence Relations)
\begin{align}
	(\forall p \in T) \quad &p \sim p, \label{eq-R1} \tag{R1} \\
	(\forall p, q \in T) \quad &(p \sim q) \implies (q \sim p), \tag{R2}\\
	(\forall p, q, r \in T) \quad &((p \sim q) \land (q \sim r)) \implies (p \sim r).\label{eq-R3} \tag{R3}
\end{align}
This least relation on $T$ can be constructed by forming the set-theoretic intersection of all relations on $T$ satisfying the conditions above. By (\ref{eq-R1})--(\ref{eq-R3}), the relation $\sim$ is an equivalence relation. Let $F$ be the equivalence classes of $\sim$, i.e.,
\begin{equation*}
	F = \frac{T}{\sim} =\{[p]\,|\, p\in T\}.
\end{equation*}
Let $u:X\rightarrow F$ be the function defined by
\begin{equation*}
	u(x) = [x].
\end{equation*}
Define operation $\cdot$, ${}^{-1}$ and $e$ on $F$ by
\begin{equation*}
	[p]\cdot [q] = [p\cdot q].
\end{equation*}
Then the operation is well-defined by (\ref{eq-WD}). It can be verified that $[e]$ is the identity and $[p]^{-1} = [p^{-1}]$ for all $p\in T$. It follows that $F$ is a group by (\ref{eq-G1})--(\ref{eq-G3}). Now we claim that $(F,u)$ satisfies the universal property.



\subsubsection{Crazy Construction from Direct Products: Only Serge Lang's Favourite}
\subsection{Group Presentations}
\subsection{Nielsen-Schreier Theorem}
\begin{theorem}[Nielsen-Schreier Theorem]
	If $H$ is a subgroup of a free group $G$, then $H$ is free.
\end{theorem}



\paragraph{Main References.} \cite{Lang2002,Bergman2015,Ribes2010}