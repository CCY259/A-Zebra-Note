\section{Introduction to Permutation Group Theory}
\subsection{Notations}
A \textbf{permutation group} of a set $\Omega$ is a subgroup of $\Sym(\Omega)$. If $G$ is a permutation group on $\Omega$, then $G$ acts on $\Omega$ via the inclusion map, i.e. $\omega^g = g(\omega)$, and this is a faithful action. The \textbf{degree} of such an action is $|\Omega|$.  Conversely, if $G$ is a faithful action on $\Omega$, then $G$ can be identified as a permutation group of $\Omega$. For simplicity, we reintroduce notations for notions in group actions. Let $G$ act on $\Omega$.  
\begin{align*}
	\omega^G &\coloneq O_G(\omega) =\{\omega^g\mid g\in G\}, \tag{\textbf{Orbit} of $\omega \in \Omega$}
	\\
	G_{\omega} &\coloneq  S_G(\omega) = \{g\in G\mid \omega^g = \omega\}, \tag{\textbf{Point stabilizer} of $\omega\in \Omega$}
	\\
	G_{X} &\coloneq   \{g\in G\mid X^g = X\}, \tag{\textbf{Setwise stabilizer} of $X\subseteq \Omega$}
	\\
	G_{(X)} &\coloneq   \{g\in G\mid x^g = x\text{ for all }x\in X\}. \tag{\textbf{Elementwise stabilizer} of $X\subseteq \Omega$}
\end{align*}
\subsection{Isomorphic Actions}
\begin{definition}
	Let $G$ and $H$ be groups acting on the sets $\Omega$ and $\Delta$, respectively. The two actions (or the pairs $(G,\Omega)$ and $(H,\Delta)$) are said to be \textbf{permutationally isomorphic} if there exist a bijection $\vartheta: \Omega \to \Delta$ and an isomorphism $\chi: G \to H$ such that
	\[ \vartheta(\omega^g) = \vartheta(\omega)^{\chi(g)} \]
	for all  $\omega \in \Omega, g \in G$. In other words, for every $g\in G$ the following diagram commutes.
	% https://q.uiver.app/#q=WzAsNCxbMCwwLCJcXE9tZWdhIl0sWzIsMCwiXFxPbWVnYSJdLFswLDIsIlxcRGVsdGEiXSxbMiwyLCJcXERlbHRhIl0sWzAsMSwiZyJdLFsxLDMsIlxcdmFydGhldGEiXSxbMCwyLCJcXHZhcnRoZXRhIiwyXSxbMiwzLCJcXGNoaShnKSIsMl1d
	\[\begin{tikzcd}
		\Omega && \Omega \\
		\\
		\Delta && \Delta
		\arrow["g", from=1-1, to=1-3]
		\arrow["\vartheta"', from=1-1, to=3-1]
		\arrow["\vartheta", from=1-3, to=3-3]
		\arrow["{\chi(g)}"', from=3-1, to=3-3]
	\end{tikzcd}\]
	If such conditions hold, the pair $(\vartheta, \chi)$ is said to be a \textbf{permutational isomorphism}. Similarly, the pair $(\vartheta, \chi)$ is a \textbf{permutational embedding} of the permutation group $G$ on $\Omega$ into the permutation group $H$ on $\Delta$, if $\chi: G \to H$ is a monomorphism and $(\vartheta, \hat{\chi})$ is a permutational isomorphism, where $\hat{\chi}: G \to \Im \chi$ is obtained from $\chi$ by simply restricting the range of $\chi$.
\end{definition}


\begin{proposition}
	Let $G$ act on a set $\Omega$. Let $\Delta$ be a set and let $\vartheta: \Omega \to \Delta$ be a bijection. Define a $G$-action on $\Delta$ by $\delta^g = \vartheta((\vartheta^{-1}(\delta))^g)$. Then $(\vartheta, \operatorname{Id}_G)$ is a permutational isomorphism from the $G$-action on $\Omega$ to the $G$-action on $\Delta$.
\end{proposition}

\begin{sketch}
	It is easy to see that the $G$-action on $\Delta$ is well-defined and is permutationally isomorphic to the $G$-action on $\Omega$. 
\end{sketch}


\begin{proposition} \label{prop-eqv-for-isom-trans-grp}
	Let $G $ and $H$ be groups acting transitively on $\Omega $ and $\Delta $, respectively. Then the following are equivalent.
	\begin{enumerate}[(1)]
		\item The actions of $G $ and $H$ on $\Omega $ and $\Delta $, respectively, are permutationally isomorphic.
		\item There exist $\omega \in \Omega $ and $\delta \in \Delta $ and an isomorphism $\varphi: G  \to H$ such that $\varphi(G_{\omega}) = H_{\delta}$.
		\item For all $\omega \in \Omega $ and $\delta \in \Delta $, there exists an isomorphism $\varphi: G  \to H$ such that $\varphi(G_{\omega}) = H_{\delta}$.
	\end{enumerate}
\end{proposition}

\begin{sketch}
	(1) $\Rightarrow$ (2) Let $(\vartheta: \Omega  \to \Delta , \varphi: G  \to H)$ be a permutational isomorphism. Let $\omega \in \Omega $ and $g \in G_{\omega}$. Then $\vartheta(\omega) ^{\varphi (g)} = \vartheta(\omega^g) = \vartheta(\omega)$, and so $\varphi(g) \in H_{\vartheta(\omega)}$. Therefore $\varphi (G_{\omega}) \leq H_{\vartheta(\omega)}$. 
	
	On the other hand if $h \in H_{\vartheta(\omega)}$ then there is some $g \in G $ such that $\varphi(g) = h$. Then $\vartheta(\omega) = \vartheta(\omega)^{h} = \vartheta(\omega)^{\varphi(g)} = \vartheta(\omega^{g})$. Since $\vartheta$ is injective, we obtain $\omega = \omega^{g}$. Therefore $g \in G_{\omega}$, and hence $h = \varphi(g) \in \varphi(G_{\omega})$. This shows that $\varphi(G_{\omega} ) = H_{\vartheta(\omega)}$.
	
	(2) $\Rightarrow$ (3) Since both $G$ and $H$ are transitive, $\{G_{\omega'} \mid \omega' \in \Omega \}$ and $\{H_{\delta'} \mid \delta' \in \Delta \}$ are conjugacy classes in $G$ and $H$, respectively, by Proposition \ref{prop-stabilizer}.(ii). Let $\varphi: G  \to H$ be an isomorphism, and $\omega \in \Omega , \delta \in \Delta$ such that $\varphi(G_{\omega}) = H_{\delta}$, and let $\omega' \in \Omega $ and $\delta' \in \Delta $. Then there are $\sigma_1 \in \Inn G$ and $\sigma_2 \in \Inn H$ such that $\sigma_1(G_{\omega'}) = G_{\omega}$ and $\sigma_2(H_{\delta}) = H_{\delta'}$. If $\phi = \sigma_1\varphi\sigma_2$ then clearly $\phi(G_{\omega'} ) = H_{\delta'}$.
	
	(3) $\Rightarrow$ (1) Let $\omega \in \Omega , \delta \in \Delta $, and $\varphi: G  \to H$ be an isomorphism such that $\varphi(G_{\omega} ) = H_{\delta}$. Then define $\vartheta: \Omega  \to \Delta $ by $\vartheta (\omega^g)= \delta^{\varphi(g)}$ for $g \in G $. We show that $\vartheta$ is well-defined. Since $G $ is transitive, we have $\Omega = \{\omega^g\mid g\in G\}$. If $\omega^{g_1} = \omega^{g_2}$ for some $g_1, g_2 \in G $, then, $g_1g_2^{-1} \in G_{\omega}$, and so $\varphi(g_1g_2^{-1}) = \varphi(g_1)\varphi(g_2)^{-1} \in H_{\delta}$. Hence $\delta^{\varphi(g_1)} = \delta^{\varphi(g_2)}$, and so $\vartheta$ is well-defined.  Since $H = \varphi(G)$ is transitive, $\vartheta$ is surjective. For injectivity, suppose that $\vartheta(\omega^{g_1} ) = \vartheta(\omega^{g_2} )$ for some $g_1, g_2 \in G$, then, by the definition of $\vartheta$, we have $\delta^{\varphi(g_1)} = \delta^{\varphi(g_2)}$ and so $\varphi(g_1)\varphi(g_2)^{-1} = \varphi(g_1g_2^{-1}) \in H_{\delta}$. Thus $g_1g_2^{-1} \in G_{\omega}$, and so $\omega^{g_1} = \omega^{g_2}$. Therefore $\vartheta$ is injective, and hence $\vartheta$ is a bijection. 
	
	If $\omega' \in \Omega $ and $g \in G $ then there exists $g' \in G $ such that $\omega' = \omega^{g'}$. Then
	\begin{equation*}
		\vartheta(\omega'^g) = \vartheta((\omega^{g'} )^g) = \delta^{\varphi(g'g)} = \delta^{\varphi(g')\varphi(g)} = \vartheta(\omega^{g'} )^{\varphi(g)} = \vartheta(\omega')^{\varphi(g)}.
	\end{equation*}
	Thus $(\vartheta, \varphi)$ is a permutational isomorphism.
\end{sketch}




\begin{proposition} \label{prop-isom-perm-grp-are-conj}
	Let $\Omega$ be a set and let $G_1,G_2$ be permutation groups on $\Sym(\Omega)$. Then $G_1$ and $G_2$ are permutationally isomorphic if and only if they are conjugate in $\Sym(\Omega)$. Moreover, if $(\vartheta, \varphi)$ is a permutational isomorphism, then $\vartheta \in \Sym(\Omega)$ and $\varphi(g) = \vartheta^{-1}g\vartheta$, for all $g \in G_1$.
\end{proposition}

\begin{sketch}
	Assume that $G_1$ and $G_2$ are permutationally isomorphic, and let $(\vartheta:\Omega\to \Omega, \varphi:G_1\rightarrow G_2)$ be a permutational isomorphism. Then we have $\vartheta(\omega^g) = \vartheta(\omega)^{\varphi(g)}$ for all $g \in G_1$ and $\omega \in \Omega$, i.e., $\omega^g = \vartheta^{-1}(\vartheta(\omega)^{\varphi(g)})$. This shows that $\vartheta^{-1}g\vartheta = \varphi(g)$, and so $G_1$ is conjugate to $\varphi(G_1) = G_2$.
	
	Conversely, suppose that $\vartheta \in \Sym\Omega$ with $\vartheta^{-1}G_1\vartheta = G_2$ and let $\varphi: G_1 \to G_2$ be the isomorphism defined by  $\varphi(g) = \vartheta^{-1}g\vartheta$. Then  $\vartheta(\omega)^{\varphi(g)} = \vartheta(\omega)^{\vartheta^{-1}g\vartheta} = \vartheta((\vartheta^{-1}(\vartheta(\omega)))^{g}) = \vartheta(\omega^g)$ for $g \in G$ and $\omega \in \Omega$. Hence $(\vartheta, \varphi)$ is a permutational isomorphism.
\end{sketch}

 Recall that if $H$ is a subgroup of  a group $G$, then the right coset action of $G$ on the set $\Gamma_H$ of right cosets of $H$  is defined by $(Hx)^g  = Hxg$ for $x,g\in G$. In view of Theorem \ref{thm-statement-of-kernel-of-left-translation}, this action is transitive. In fact, every transitive action is permutationally isomorphic to a coset action. 
\begin{proposition} \label{prop-transitive-action-perm-isom}
	Let $G$ act transitively on $\Omega$ and let $\omega\in \Omega$. Then the $G$-action on $\Omega$ is permutationally isomorphic to the $G$-action on $\Gamma_{G_\omega}$.
\end{proposition}
\begin{sketch}
	Since $G$ is transitive, we have $\omega^G = \Omega$. Let $\vartheta:\Omega \rightarrow \Gamma_{G_\omega}$ be the bijective function as defined in Lemma \ref{lemma-orbit-partition-and-bijection}.(ii). Then $\vartheta((\omega^g)^h) = \vartheta(\omega^{gh}) = G_\omega gh = (G_\omega g)^h = \vartheta(\omega^g)^h$. This shows that $(\vartheta,\operatorname{Id}_G)$ is a permutational isomorphism.
\end{sketch}
\begin{remark}
	In case of a permutation group, or more generally, a faithful action of $G$ on $\Omega$, we can verify that this action is faithful on $\Gamma_{G_\omega}$. Let $\rho$ be the associated homomorphism. Then $\rho(G)\cong G$ and so we can establish a permutational isomorphism between $(G,\Omega)$ and $(\rho(G),\Gamma_{G_\omega})$.
\end{remark}


\begin{corollary}[Frattini's Argument] \label{cor-Frattini-transitive}
	Let $G$ act transitively on $\Omega$ and let $\omega\in\Omega$. Then a subgroup $H$ of $G$ is transitive if and only if $G = G_\omega H$.
\end{corollary}
\begin{sketch}
If $H$ is transitive, then for every $g\in G$, there exists $h\in H$ such that $\omega^g = \omega^h$. By Proposition \ref{prop-transitive-action-perm-isom}, we get $G_\omega g = G_\omega h$. Hence we can write $gh^{-1} = g'\in G$ and thus $g = g'h\in G_\omega H$.
	
For the converse, reverse the argument in the previous paragraph.
\end{sketch}

\subsection{Blocks}
\begin{definition}
	Let $G$ act transitively on $\Omega$. The nonempty subset $\Delta$ of $\Omega$ is called a \textbf{block} if for every $g\in G$, either $\Delta^g=\Delta$ or $\Delta^g \cap \Delta = \emptyset$. All the singletons of $\Omega$ and the set $\Omega$ itself are blocks, and so they are said to be \textbf{trivial}.
\end{definition}



\begin{proposition}
	Let $G$ act transitively on $\Omega$.  Then the following propositions hold.
	\begin{enumerate}[(i)]
		\item If $\Delta$ is a block of $\Omega$, then $G_{\Delta}$ acts transitively on $\Delta$. 
		\item If $\Delta$ is a block, then $|\Delta^g| = |\Delta|$ and $\Delta^g$ is a block for each $g\in G$. 
		\item If $\Delta$ is a subset of $\Omega$, then $\Delta$ is a block if and only if $\{\Delta^g\mid g\in G\}$ forms a partition of $\Omega$.
	\end{enumerate}  
\end{proposition}
\begin{sketch}
	(i) Clearly $G_{\Delta}$ acts on $\Delta$. Let $\delta_1,\delta_2\in\Delta$. Since $G$ is transitive, $\delta_1^g = \delta_2$ for some $g\in G$. Hence $\delta_2\in \Delta^g\cap \Delta$. Since $\Delta$ is a block, we get $\Delta^g = \Delta$. Thus $g\in G_\Delta$.
	
	(ii) Clearly $|\Delta^g| = |\Delta|$.  Let $h\in G$ and assume that  $\Delta^g \cap (\Delta^g)^h \neq \emptyset$. Then $\Delta \cap \Delta^{g h g^{-1}}  = (\Delta^g \cap \Delta^{gh})^{g^{-1}} \neq \emptyset$. Since $\Delta$ is a block, we have  $\Delta = \Delta^{g h g^{-1}}$, whence $\Delta^g = \Delta^{g h}$. Thus $\Delta^g$ is a block.
	
	(iii) Assume that $\Delta$ is a block. Let $\mathcal{P} = \{\Delta^g \mid g \in G\}$. We claim that $\mathcal{P}$ is a partition of $\Omega$. Let $\omega \in \Omega$. Since $G$ is transitive and $\Delta \neq \emptyset$, there is some $g \in G$ such that $\omega^g \in \Delta$. Thus $\omega \in \Delta^{g^{-1}} $ which shows that $\bigcup_{g \in G} \Delta^g = \Omega$.  Let $\Delta^g, \Delta^h \in \mathcal{P}$ and suppose that $\Delta^g \cap \Delta^h \neq \emptyset$. Then $\Delta^g \cap (\Delta^g)^{g^{-1} h} \neq \emptyset$. By (ii), $\Delta^g$ is a block. So $\Delta^g = (\Delta^g)^{g^{-1} h} = \Delta^h$. Therefore $\mathcal{P}$ is a partition of $\Omega$. 
	
	For the converse, note that the definition of partition gives $\Delta g = \Delta$ or $\Delta g \cap \Delta = \emptyset$ for all $g \in G$.
\end{sketch}

\begin{definition}
	Let $G$ act on $\Omega$. An equivalence relation $\sim$ on $\Omega$ is called a $G$-\textbf{congruence} if  $$\omega_1 \sim \omega_2 \iff \omega_1^g \sim \omega_2^g$$ 
	for all $\omega_1,\omega_2\in \Omega$ and $g\in G$. We also say that $G$ preserves the relation.
\end{definition}
\begin{proposition}
	Let $G$ act transitively on $\Omega$.
	\begin{enumerate}[(i)]
		\item If $\sim$ is a $G$-congruence on $\Omega$, then each equivalence class is a block of $\Omega$.
		\item  If  $\Delta$ is a block, then $\Sigma = \{\Delta^g\mid g\in G\}$ is the set of equivalence classes of a $G$-congruence on $\Omega$. Hence $G$ acts transitively on $\Sigma$.
	\end{enumerate}
\end{proposition}
\begin{sketch}
	(i) let $[\omega]$ be an equivalence class of $\sim$. It suffices to show that $[\omega]^g$ is an equivalence class for each $g\in G$. Let $g\in G$. If $\alpha^g\in [\omega]^g$, then $\alpha \sim \omega$ implies $\alpha^g \sim \omega^g$ and so $[\omega]^g\subseteq [\omega^g]$. If $\alpha\in [\omega^g]$, then $\alpha \sim \omega^g$. Hence $\alpha^{g^{-1}} \sim \omega$ and $\alpha^{g^{-1}}\in [\omega]$. This implies $\alpha = (\alpha^{g^{-1}})^g \in [\omega]^g$ and thus $[\omega^g]\subseteq [\omega]^g$. Consequently, $[\omega]^g$ is an equivalence class (namely $[\omega^g]$) which proves that $[\omega]$ is a block. 
	
(ii) Clearly any two sets in $\Sigma$ are pairwise disjoint. Let $\omega\in\Omega$ and let $\alpha\in \Delta$. Since $G$ is transitive, there exists $g \in G$ such that $\alpha^g = \omega$. So $\omega\in \Delta^g$. This shows that $\Sigma$ is a partition of $\Omega$. Now we verify that the equivalence relation $\sim$ on $\Omega$ induced by $\Sigma$ (where $\omega_1\sim\omega_2$ if and only if $\omega_1,\omega_2$ belong to the same set in $\Sigma$) is a $G$-congruence. Let $g\in G$. Assume  that $\omega_1 \sim \omega_2$. Then   $\omega_1, \omega_2 \in \Delta^h$ for some $h \in G$.  Then we get $\omega_1^g,\omega_2^g \in (\Delta^h)^g = \Delta^{hg}$ and so $\omega_1^g\sim\omega_2^g$.
 Assume that $\omega_1^g \sim \omega_2^g$. Then $\omega_1^g, \omega_2^g \in \Delta^{h}$ for some $h\in G$. So $\omega_1,\omega_2\in \Delta^{hg^{-1}}$ and $\omega_1\sim\omega_2$.  Therefore, the relation $\sim$ is a $G$-congruence.

To prove the last statement, note that the action of $G$ on $\Sigma$ is given by $(\Delta^g)^h = \Delta^{gh}$. This action is transitive because $(\Delta^g)^{g^{-1}h} = \Delta^h$. 
\end{sketch}

\begin{definition}
	Let $G$ act transitively on $\Omega$. The set $\Sigma$ of equivalence classes associated to a $G$-congruence on $\Omega$ is called a \textbf{system of blocks} (or a \textbf{system of imprimitivity}). Such a system is said to be \textbf{trivial} if it only contains trivial blocks.
\end{definition}

\subsection{Primitive Actions}

\begin{definition}
	Let $G$ act transitively on $\Omega$. The action (or $G$-set) is said to be \textbf{primitive} (or $G$ is \textbf{primitive} on $\Omega$) if $G$ has no nontrivial blocks; otherwise, it is \textbf{imprimitive}.
\end{definition}

\begin{proposition} \label{prop-block-and-stab}
	Let $G$ acts transitively on $\Omega$. Let $\omega\in \Omega$ be fixed. Then there is a one-to-one correspondence between the set of blocks of $\Omega$ containing $\omega$ and the set of subgroups which contains the stabilizer $G_\omega$ of $\omega$.
\end{proposition}
\begin{sketch}
	Let $\Delta$ be a block containing $\omega \in \Omega$, and consider the set $H_\Delta = \{g \in G \mid \omega^g \in \Delta\}$. We claim that $H_\Delta$ is a subgroup of  $G$. Clearly, $1 \in H_\Delta$. Let $g,g' \in H_\Delta$. Since $\omega$ and $\omega^g$ both lie in $\Delta$, we see that $\Delta^g \cap \Delta\neq \emptyset$ and hence that $\Delta^g = \Delta$. Now we have $\omega^{gg'} = (\omega^g)^{g'} \in \Delta^{g'} = \Delta$ and hence $gg' \in H_\Delta$. Also, for $g \in H_\Delta$ we have $\omega^g \in \Delta$. Since $\omega$ is also in $\Delta$, this means $\Delta \cap \Delta^g \neq \emptyset$, which implies $\Delta^g = \Delta$. Acting on both sides by $g^{-1}$, we get $\Delta^{g^{-1}} = (\Delta^g)^{g^{-1}} = \Delta$. Since $\omega \in \Delta$, it follows that $\omega^{g^{-1}} \in \Delta^{g^{-1}} = \Delta$, and hence $g^{-1} \in H_\Delta$. Therefore $H_\Delta$ is a subgroup of $G$.
	
	Observe that $G_\omega \le H_\Delta$ since for any $s \in G_\omega$, we have $\omega^s=\omega \in \Delta$.  Let $\Sigma$ be the set of blocks of $\Omega$ containing $\omega$ and let $\mathcal{H}$ be the set of subgroups containing $G_\omega$. Let $\theta:\Sigma\rightarrow \mathcal{H}$ be the function defined by $\theta(\Delta)=H_\Delta$. We claim that $\theta$ is bijective.
	
	Let $\Delta$ and $\Delta'$ be distinct blocks in $\Sigma$. Without loss of generality assume that $\Delta'\not\subseteq \Delta$. Then there exists some  $\delta \in \Delta'$ with $\delta \notin \Delta$. Since $G$ acts transitively on $\Omega$, there exists some $g \in G$ such that $\omega^g = \delta$. So $g \in H_{\Delta'}$. Since $\delta\not\in \Delta$, we have $g \notin H_\Delta$, and hence $H_\Delta \ne H_{\Delta'}$. This shows that $\theta$ is injective.
	
	Let $H\in\mathcal{H}$. Consider the subset $C = \{\omega^h \mid h \in H\}$ of $\Omega$. We show that $C$ is a block. Clearly $C$ is non-empty and $C^g = C$ for each $g\in H$. Let $g \in G$ be such that $C^g \cap C\neq \emptyset$. Then there exist $h_1, h_2 \in H$ such that $(\omega^{h_1})^g = \omega^{h_2}$. This gives $\omega^{h_1g} = \omega^{h_2}$, so $\omega^{h_1gh_2^{-1}} = \omega$. Hence $h_1gh_2^{-1} \in G_\omega \le H$, and thus $g \in H$. Consequently $C^g = C$. Therefore $C$ is a block. Note that $\theta(C) = H_C = \{g \in G \mid \omega^g \in C\}$. Clearly $H \le H_C$. Let $g \in H_C$. Then $\omega^g = \omega^h$ for some $h \in H$. Hence $\omega^{gh^{-1}} = \omega$ and thus $gh^{-1} \in G_\omega \le H$, giving $g \in H$. So $\theta(C) = H$, which shows that $\theta$ is surjective.
\end{sketch}
\begin{remark}
	This correspondence is order-preserving, i.e., if $\Delta_1,\Delta_2$ are blocks of $\Omega$ containing $\omega$, then $\Delta_1\subseteq \Delta_2$ if and only if $\theta(\Delta_1)\subseteq \theta(\Delta_2)$. Indeed $\Delta_1\subseteq \Delta_2$ implies that $\omega^g \in \Delta_1\subseteq \Delta_2$ for all $g\in H_{\Delta_1}$. Conversely, suppose that $H_{\Delta_1}\leq H_{\Delta_2}$. Let $\delta\in \Delta_1$. So there exists $g\in G$ such that $\delta = \omega^g$. This implies $g\in H_{\Delta_1}\subseteq H_{\Delta_2}$. So $\delta = \omega^g \in \Delta_2$. This shows that $\theta$ is order-preserving.
\end{remark}

\begin{corollary} \label{cor-prim-iff-stab-is-max}
	Let $G$ act transitively on $\Omega$. Then $G$ is primitive if and only
	if the point stabilizers are maximal subgroups.
\end{corollary}
\begin{sketch}
	Suppose $G$ is primitive on $\Omega$. Let $\omega \in \Omega$.  Since $G$ is primitive, there are  two  blocks containing $\omega$, namely $\{\omega\}$ and $\Omega$.  By Proposition \ref{prop-block-and-stab}, there are only two subgroups of $G$ containing $G_\omega$, namely $G_\omega$ and $G$. So there is no proper subgroup of $G$ which properly contains $G_\omega$. Therefore $G_\omega$ is maximal in $G$.
	
	Conversely, suppose that every stabilizer is a maximal subgroup. Fix $\omega\in \Omega$. Then there are only two subgroups of $G$ containing $G_\omega$, namely $G_\omega$ and $G$. By Proposition \ref{prop-block-and-stab}, we only have two such blocks $\{\omega\}$ and $\Omega$.  Consequently, $\Omega$ can have no other blocks besides itself and its singletons, and so $\Omega$ is primitive.
\end{sketch}

\subsection{Centralizers and Normalizers of Transitive Permutation Groups}
\begin{definition}
	Let $G$ act on a set $\Omega$. We say that $G$ acts \textbf{semiregularly} on $\Omega$ ($G$ or the $G$-set $\Omega$ is \textbf{semiregular}) if nonidentity elements fix no point, i.e., $G_\omega = 1$ for all $\omega\in\Omega$. We say that $G$ acts \textbf{regularly} on $\Omega$ if $G$ is transitive and semiregular.
\end{definition}
Let $H$ be a subgroup of a group $G$ and let $\Gamma_H$ be the set of right cosets of $H$ in $G$. We have the right coset action of $G$ on $\Gamma_H$, i.e., $(Hx)^g = Hxg$. In fact, we can also define another right coset action of $N_G(H)$ on $\Gamma_H$ by left multiplication: $(Hx)^g = Hg^{-1}x$. Note that $g^{-1}$ is required and $G$ is restricted to $N_G(H)$ to ensure that this action is a well-defined right action.
\begin{lemma} \label{lemma-left-right-mult}
	Let $G$ be a group with a subgroup $H$, and put $K \coloneq  N_G(H)$. Let $\Gamma_H$ denote the set of right cosets of $H$ in $G$, and let $\rho$ and $\lambda$ denote the right and left multiplication of $G$ and $K$, respectively, on $\Gamma_H$ as defined above. Then the following hold.
\begin{enumerate}[(i)]
\item $\Ker \lambda = H$ and $\lambda(K)$ is semiregular.
\item The centralizer $C$ of $\rho(G)$ in $\Sym\Gamma_H$ equals $\lambda(K)$.
\item $H \in \Gamma_H$ has the same orbit under $\lambda(K)$ as under $\rho(K)$.
\item If $\lambda(K)$ is transitive, then $K = G$, and $\lambda(G)$ and $\rho(G)$ are conjugate in $\Sym(\Gamma_H)$.
\end{enumerate}
\end{lemma}
\begin{sketch}
	(i) Clearly $(Ha)^{\lambda(x)} = Ha$ for all $a \in G \Leftrightarrow x \in H \Leftrightarrow (Ha)^{\lambda(x)} = Ha$ for some $a \in G$. Thus $\Ker \lambda = H$, and the point stabilizer of each point $Ha \in \Gamma_H$ under the action $\lambda$ is $H$.

(ii) Let $\lambda(y)\in \lambda(K)$ and let $\rho(x)\in \rho(G)$. Then for each $a \in G$,
	\begin{align*}
		(Ha)^{\rho(x)\lambda(y)} &= Hy^{-1}ax = (Ha)^{\lambda(y)\rho(x)}
	\end{align*}
	and so $\rho(x)\lambda(y) = \lambda(y)\rho(x)$. Thus $\lambda(K) \subseteq C$.

Conversely, suppose that $z \in C$ and write $Hc=H^z$ for some $c\in G$. Then for each $a \in G$,
\begin{equation*}
	(Ha)^z = (H^{\rho(a)})^z=H^{\rho(a)z} = H^{z\rho(a)} = Hca.
\end{equation*}
	In particular, for each $a \in H$, we have
\begin{equation*}
	Hc= H^z = (Ha)^z = Hca.
\end{equation*}  Thus $c \in N_G(H) = K$, and $z = \lambda(c^{-1}) \in \lambda(K)$. This shows that $C \subseteq \lambda(K)$.

(iii) Note that
\begin{align*}
	H^{\lambda(K)} &= \{H^{\lambda(x)}\mid x\in K\} 
	\\
	&= \{Hx^{-1}\mid x\in K\} 
	\\
	&= \{Hx\mid x\in K\}
	\\
	&= \{H^{\rho(x)}\mid x\in K\}
	\\
	&= H^{\rho(K)}.
\end{align*}

(iv) If $\lambda(K)$ is transitive, then (iii) shows that each coset $Hx$ ($x \in G$) has the form $H^{\rho(y)} = Hy$ for some $y \in K$. So $xy^{-1} \in H\subseteq K$ and thus $x\in K$. Hence $G = K = N_G(H)$. Now we can define a permutation $t \in\Sym(\Gamma_H)$ by $(Ha)^t \coloneq  Ha^{-1}$. Note that $t$ is well-defined because $H$ is normal in $G$. Finally, the calculation shows that $t^{-1}\lambda(x)t = \rho(x)$ for all $x \in G$.
\end{sketch}

\begin{theorem} \label{thm-centralizer-trans-perm-grp}
	Let $G$ be a transitive subgroup of $\Sym(\Omega)$ and let $\omega\in \Omega$. Let $C$ be the centralizer of $G$ in $\Sym(\Omega)$. Then the following hold.
	\begin{enumerate}[(i)]
		\item $C$ is semiregular, and $C \cong N_G(G_\omega)/G_\omega$.
		\item $C$ is transitive if and only if $G$ is regular.
		\item If $C$ is transitive, then it is conjugate to $G$ in $\Sym(\Omega)$ and hence $C$ is regular.
		\item $C = 1$ if and only if $G_\omega$ is self-normalizing in $G$, i.e., $N_G(G_\omega) = G_\omega$.
		\item If $G$ is abelian, then $C = G$.
		\item If $G$ is primitive and nonabelian, then $C = 1$.
	\end{enumerate}
\end{theorem}
\begin{sketch}
(i) Let $\omega\in\Omega$. Since $G$ is transitive, it follows from Proposition \ref{prop-transitive-action-perm-isom} that $(G,\Omega)$ and $(\rho(G),\Gamma_{G_\omega})$ are permutationally isomorphic. By Lemma \ref{lemma-left-right-mult}.(ii), $C$ is permutationally isomorphic to $C_{\Sym(\Gamma_{G_\omega})}(\rho(G)) = \lambda(K)$, where $K = N_G(G_\omega)$. So $C$ is semiregular by Lemma \ref{lemma-left-right-mult}.(i). Also, Lemma \ref{lemma-left-right-mult}.(i) and the First Isomorphism Theorem show that
	$
		C \cong \lambda(K) \cong K/\Ker \lambda  = N_G(G_\omega)/G_\omega$.
	
	(ii) If $C$ is transitive, then $\lambda(K)$ is transitive. By Lemma \ref{lemma-left-right-mult}.(iv), $N_G(G_\omega) = G$.  Then $G_\omega =G_{\omega^g}$ for all $g\in G$ by Proposition \ref{prop-stabilizer}.(ii). Since $G$ is transitive and faithful, it follows from Proposition \ref{prop-kernel-of-transitive-action} that $G_\omega = 1$. Hence $G$ is regular. 
	
	Conversely, if $G$ is regular, then $K = G$. By Theorem \ref{thm-statement-of-kernel-of-left-translation}.(i), we know that $\lambda(G)$ is transitive. Hence $C$ is transitive.
	
	(iii) If $C$ is transitive, we see that $K = G$ and both $\lambda(G)$ and $\rho(G)$ are conjugate in $\Sym(\Gamma_{G_\omega})$ by Lemma \ref{lemma-left-right-mult}.(iv). Since $G$ and $\rho(G)$ are permutationally isomorphic, $C$ and $G$ are conjugate in $\Sym(\Omega)$. By Proposition \ref{prop-isom-perm-grp-are-conj}, $C$ and $G$ are permutationally isomorphic. From (ii), we know that $G$ is regular and thus $C$ is regular.
	
	(iv) The result follows from (i).
	
	(v) Since $G$ is abelian, it follows that $G \subseteq C$. Let $\omega \in \Omega$.  Since $C$ is semiregular by (i), we have $C_\omega = 1$. By Frattini's argument (Corollary \ref{cor-Frattini-transitive}), we now have $C = GC_\omega = G$.
	
	(vi) Since $G$ is primitive, it follows from Corollary \ref{cor-prim-iff-stab-is-max} that $G_\omega$ is a maximal subgroup. By the argument in (ii), $N_G(G_\omega)$ is a proper subgroup of $G$. So we obtain $N_G(G_\omega) = G_\omega$. Hence the result follows from (iv).
\end{sketch}


\begin{theorem} \label{thm-normalizer-of-perm-grp}
	Let $G$ be a transitive subgroup of $\Sym(\Omega)$, let $N$ be the normalizer of $G$ in $\Sym(\Omega)$ and let $\alpha \in \Omega$. If $\Psi:N\to \Aut G$ is the homomorphism defined by conjugation, and $\sigma \in \Aut (G)$, then
	$ \sigma \in \Im  \Psi$ if and only if $(G_\alpha)^\sigma$ is a point stabilizer for $G$, i.e., $(G_\alpha)^\sigma = G_\beta$ for some $\beta\in\Omega$.
\end{theorem}
\begin{sketch}
	Let $\sigma \in\Im  \Psi$, so $\sigma = \Psi(x)$ for some $x \in N$. Then $$(G_\alpha)^\sigma= (G_\alpha)^{\Psi(x)} = x^{-1}G_\alpha x = G_{\alpha^x}$$
	by  Proposition \ref{prop-stabilizer}.(ii). Write $\beta = \alpha^x$ and we are done. 
	
	Conversely, suppose that $(G_\alpha)^\sigma = G_\beta$ for some $\beta \in \Omega$. By Proposition \ref{prop-eqv-for-isom-trans-grp}, the two permutation groups on $\Omega$ (which are $G$ and $G^\sigma$) are permutationally isomorphic. By Proposition \ref{prop-isom-perm-grp-are-conj}, there exists $t \in \Sym\Omega$ such that $x^\sigma = t^{-1}xt$ for all $x \in G$. Clearly $t \in N$. Hence $\sigma = \Psi(t) \in\Im  \Psi$ as required.
\end{sketch}

\begin{definition}
	Let $G$ be a group. The \textbf{holomorph} of $G$, denoted by $\operatorname{Hol} (G)$, is the semidirect product $G \rtimes \Aut (G)$ with respect to the natural action of $\Aut (G)$ on $G$.
\end{definition}
In the case where $G$ is regular,  the normalizer of $G$ in the symmetric group is the holomorph of $G$.
\begin{corollary}
	Let $G$ be a transitive subgroup of $\Sym (\Omega)$ and let $N$ be the normalizer of $G$ in $\Sym (\Omega)$. If $G$ is regular, then $\Im  \Psi = \Aut (G)$, where $\Psi$ is the homomorphism defined in Theorem \ref{thm-normalizer-of-perm-grp}. Let $\alpha\in \Omega$. In this case $N_\alpha \cong \Aut (G)$, and $N$ is isomorphic to $\operatorname{Hol}(G)$.
\end{corollary}
\begin{sketch}
	Since $G$ is regular, therefore $G_\alpha = 1$, and so $\Im  \Psi = \Aut (G)$ by Theorem \ref{thm-normalizer-of-perm-grp} since $(G_\alpha)^{\sigma} = \{1\}$ is a point stabilizer for $G$. The centralizer $C$ of $G$ in $\text{Sym}(\Omega)$ is transitive by Theorem \ref{thm-centralizer-trans-perm-grp}.(ii). By Frattini's argument (Corollary \ref{cor-Frattini-transitive}), we get $N = CN_\alpha$. Next we show that $C \cap N_\alpha = 1$. Let $x\in C \cap N_\alpha$. Then $\alpha^x = \alpha$. Since $G$ is transitive, for every $\omega\in\Omega$, we have $\omega = \alpha^g$ for some $g\in G$. This implies 
	\begin{equation*}
		\omega^x = (\alpha^g)^x =\alpha^{gx} = \alpha^{xg} = (\alpha^x)^g = \alpha^g = \omega. 
	\end{equation*}
	This proves $C \cap N_\alpha = 1$. Clearly $\Ker\Psi = C$. By the First and Second Isomorphism Theorems, 
\begin{equation*}
	\Aut (G) =\Im  \Psi \cong N/\Ker \Psi = N/C = CN_\alpha/C \cong N_\alpha.
\end{equation*} In fact, the isomorphism $N_\alpha\stackrel{\sim}{\rightarrow}\Aut (G)$ maps $x$ to the conjugation on $G$ by $x$. Note that $G \cap N_\alpha = G_\alpha = 1$ and $G$ is normal in $N$. By Propositions \ref{prop-semidirect-prod-isom} and \ref{prop-internal-semidirect-prod-isom}, we have 
\begin{equation*}
	N = GN_\alpha \cong G \rtimes \Aut (G) = \operatorname{Hol}(G). \qedhere
\end{equation*}
\end{sketch} 

\begin{remark}
It seems to be true that permutational isomorphisms preserve transitivity, primitivity, regularity and faithfulness. More specifically, let $(G,\Omega)$ and $(H,\Delta)$ be permutational isomorphic group actions, then  the following holds.
\begin{enumerate}[(i)]
	\item $(G,\Omega)$ is transitive if and only if $(H,\Delta)$ is transitive.
	\item $(G,\Omega)$ is primitive if and only if $(H,\Delta)$ is primitive.
	\item $(G,\Omega)$ is regular if and only if $(H,\Delta)$ is regular.
	\item $(G,\Omega)$ is faithful if and only if $(H,\Delta)$ is faithful.
\end{enumerate}
These might be too trivial until they are not mentioned anywhere, but it is important to know. Siapa boleh tolong provekan?
\end{remark}

\paragraph{Main References.} \cite{Praeger2018,Dixon1996,Cameron1999,Jacobson1985}