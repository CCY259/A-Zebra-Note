\subsection*{Subgroups}
\begin{enumerate}[(1)]
	\item Simply speaking, given a group with a binary operation $\cdot$, if the restriction of $\cdot$ to a subset $H\subseteq G$ makes $H$ itself a group, then we call $H$ a \textbf{subgroup} of $G$.
	\item We have a analogous criterion for sets to be subspaces of a vector space; so we call this criterion the \textbf{subgroup criterion}. A subset $H$ is a subgroup of  a group if and only if  $$x,y\in H\Rightarrow xy^{-1}\in H.$$ This statement also equivalent to $xy\in H$ and $x^{-1}\in H$ for all $x,y\in H$.
	\item Let $H$ be a subgroup of $G$. Then the identity of $H$ is the identity of $G$.
	\item Clearly $G$ and $\{e\}$ are subgroups of $G$. They are called \textbf{trivial subgroups}. 
	\item Write $H\leq G$ for a subgroup $H$ of $G$. Then we have:
	\begin{enumerate}[(i)]
		\item $K\leq H$ and $H\leq G$ $\Rightarrow$ $K\leq G$;
		\item $H,K\leq G$ and $K\subseteq H$ $\Rightarrow$ $K\leq H$.
	\end{enumerate}
	\item The \textbf{intersection of subgroups} of $G$ is a \textbf{subgroup} of $G$. I think that the index set is not required to be a countable set. So $\bigcap_{\lambda\in\Lambda} H_\lambda$ is also a subgroup of $G$.
	\item The \textbf{union of subgroups} is \textbf{not necessarily a subgroup} of $G$. In fact, 
	\begin{equation*}
		H\cup K\leq G\Leftrightarrow H\subseteq K\text{ or }K\subseteq H.
	\end{equation*}
	\item The product $HK = \{hk\,|\, h\in H,k\in K\}$ is called the \textbf{product} of subgroups $H$ and $K$. It is \textbf{not necessarily a subgroup} of $G$. In fact, we have
	\begin{equation*}
		HK\leq G\Leftrightarrow HK = KH.
	\end{equation*}
\end{enumerate}

\subsection*{Subgroups Defined by Subsets}
Let $S$ be a subset of $G$. There are three classical subgroups that can be defined from $S$:
\begin{enumerate}[(1)]
	\item the \textbf{subgroup} of $G$ \textbf{generated by} $S$, denoted by $\langle S\rangle$; this is the set of all elements of $G$ which can be written as the product of finite number of elements of $S$ or of the inverses of elements of $S$, i.e., every element is of the form
	\begin{equation*}
		s_1s_2\cdots s_n
	\end{equation*}
	where $s_i\in S$ or $s_i^{-1}\in S$. 
	\begin{itemize}
		\item $\langle S\rangle$ is the intersection of all subgroups of $G$ that contain $S$.
		\item $\langle S\rangle$ is the smallest subgroup of $G$ that contains $S$.
		\item If $H = \langle S\rangle$, then  $S$ is called a \textbf{generating set} of $H$. 
		\item If $S$ is a finite set, then $H$ is said to be \textbf{finitely generated}.
		\item  In particular, if $S$ is a singleton, then $H$ is said to be \textbf{cyclic}.
		\item If we involve more than a set, say $S_1,\dots, S_n$, then $\langle S_1,\dots, S_n\rangle$ is the subgroup generated by the union of subsets $S_i$. If we have a collection of subsets $S_\lambda (\lambda\in \Lambda)$, then $\langle S_\lambda \,|\, \lambda\in \Lambda$ is the subgroup generated by $\bigcup_{\lambda\in \Lambda} S_\lambda$.
	\end{itemize}
	\item the \textbf{normalizer} of $S$ in $G$, denoted by $N_G(S)$; this is the set
	\begin{equation*}
		N_G(S) = \{g\in G\,|\, gSg^{-1}= S\}.
	\end{equation*}
	In this case, we say $g$ \textbf{normalizes} $S$.
	\begin{itemize}
		\item If $H$ is a subgroup of $G$, then $H$ is normal in $	N_G(H)$.
		\item $N_G(H)$ is the largest subgroup of $G$ that contains $H$ as a normal subgroup.
	\end{itemize}
	\item the \textbf{centralizer} of $S$ in $G$, denoted by $C_G(S)$; this is the set
	\begin{equation*}
		C_G(S) = \{g\in G\,|\, gs = sg\text{ for all }s\in S\}.
	\end{equation*}
	In this case, we say $g$ centralizes $S$.
	\begin{itemize}
		\item $C_G(G)$ is called the \textbf{center} of $G$, and is denoted by $Z(G)$.
	\end{itemize}
	\item The next subgroup is not depending on $S$, but it has important connection between those subgroups defined by $S$: If $H$ is a subgroup of $G$, then $xHx^{-1}$ (where $x\in G$ is fixed) is called a \textbf{conjugate subgroup} of $H$.
	\begin{itemize}
		\item Let $S\subseteq G$ and $x\in G$. Then
		\begin{align*}
			\langle xSx^{-1}\rangle &=  x\langle S\rangle x^{-1},
			\\
			N_G(xSx^{-1}) &= xN_G(S)x^{-1} ,
			\\
		 C_G(xSx^{-1}) &= xC_G(S)x^{-1}.
		\end{align*}
	\end{itemize}
	As a consequence, $N_G(S)\subseteq N_G(C_G(S))$.
\end{enumerate}