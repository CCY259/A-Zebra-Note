\subsection*{Integral domains}
\begin{enumerate}[(1)]
	\item A nonzero element $x$ is called a \textbf{left zero divisor} (resp. \textbf{right zero divisor}) if $xa = 0$ (resp. $ax = 0$) for some $a\in R$. We call $x$ a \textbf{zero divisor} if it is either left \textbf{or} right zero divisor.
	\item An element $x$ of nonzero $R$ is said to be \textbf{left invertible} (resp. \textbf{right invertible}) if $ax =1$ (resp. $xa = 1$) for some $a\in R$. In this case, $a$ is called a \textbf{left inverse} (resp. \textbf{right inverse}) of $x$.
	\item Left-invertibility \textbf{does not imply} right-invertibility.
	\item If $x$ has left inverse $y$ and right inverse $y'$, then $y=y'$ and so we shall say that $x$ is \textbf{invertible} (or a \textbf{unit}) in $R$ and $y$ is called the inverse of $x$.
	\item The set of all units in $R$ forms a multiplicative group.
	\item $x$ is a \textbf{zero divisor} $\Rightarrow x$ is \textbf{not a unit}. Thus fields contain no zero divisor.
	\item $x$ is not a zero divisor $\Rightarrow$ $x$ is \textbf{left and right cancellable}, i.e., $xa=xb\Rightarrow x = 0$ or $b=c$, and $ax=bx\Rightarrow x = 0$ or $b=c$.
	\item An \textbf{integral domain} $R$ is 
	\begin{enumerate}[(i)]
		\item a commutative nonzero ring with $1$ 
		\item which has \textbf{no zero divisors}.
	\end{enumerate}
	\item If the commutative property is dropped, we will call $R$ a \textbf{domain}.
	\item Cancellation law holds for integral domains.
	\item $R$ is an integral domain and $R$ is finite $\Rightarrow$ $R$ is a field.
\end{enumerate}