\subsection*{Cosets}
\begin{enumerate}[(1)]
	\item Let $H$ be a subgroup of $G$ and $x\in G$. 
	\begin{itemize}
		\item $xH$ is a \textbf{left coset} of $H$ in $G$;
		\item $Hx$ is a \textbf{right coset} of $H$ in $G$.
		\item The number of distinct left cosets of $H$ in $G$ is called the index of $H$ in $G$ and denoted $[G:H]$.
	\end{itemize}
	When the context is clear, we will call a left coset simply a coset.
	\item $xH = yH\Leftrightarrow xy^{-1}\in H$.
	\item $G$ can be \textbf{partitioned} into a \textbf{disjoint union of cosets} of $H$.
	\item Each coset has the same number of elements as in $H$, i.e., $|xH| = |H|$.
	\item $[G:K] = [G:H][H:K]$ for $K\leq H\leq G$.
	\item \textbf{Lagrange's Theorem.} $|G| = [G:H]|H|$.
	\item As corollary, the order of an element of a finite group $G$ must divide the order of $G$.
	\item Using the concept of cosets one can establish 
	\begin{equation*}
		|HK| = \frac{|H||K|}{|H\cap K|}.
	\end{equation*}
	Remark that it has a weaker hypothesis than the Second Isomorphism Theorem since we don't require any subgroup to be normal.
\end{enumerate}