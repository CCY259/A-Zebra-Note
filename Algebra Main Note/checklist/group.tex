\subsection*{Groups} 
\begin{enumerate}[(1)]
	\item We start from defining a set with a binary operation.
	\begin{itemize}
		\item A set $M$ with a binary operation $M\times M \rightarrow M$ is called a \textbf{magma}. We denote it by $(M,\cdot)$ or simply $M$. Note that the closure property is mentioned in the binary operation.
		\item A magma $M$ is called a \textbf{semigroup} if  it satisfies associative law: $(xy)z = x(yz)$ for all $x,y,z\in M$.
		\item A semigroup $M$ is called a \textbf{monoid} if there is an identity element $e\in M$ (or $1$) such that $ex = xe = x$ for all $x\in M$.
		\item A monoid $G$ is called a \textbf{group} if every element in $G$ has an inverse: for all $x\in G$, there exists $x^{-1}\in G$ such that $xx^{-1}  = e = x^{-1}x$.
	\end{itemize}
	\item $\text{Magma}\subset \text{Semigroup} \subset \text{Monoid} \subset \text{Group}$.
	\item There is an optional property: $x,y$ are \textbf{commutative} if $xy = yx$. A commutative group is called an \textbf{abelian group}.
	\item We can drop the bracket for the element $x_1x_2\cdots x_n$ in semigroup   because of the \textbf{general associative law} (so we say the product is \textbf{uniquely determined}). It  means that the ways to take the product while fixing the order of elements does not affect the outcome. Similarly, the product of elements in a commutative semigroup is uniquely determined regardless of the ways to take the product or the order of elements. This is because of the \textbf{general commutative law}.
	\item If elements $x$ and $y$ in a monoid is invertible,  then $(x^{-1})^{-1} = e$ and $(xy)^{-1} = y^{-1}x^{-1}$. The latter is sometimes described as the \textbf{socks-shoes property}.
	\item In a group, define $x^n$ ($n\geq 1$) as the product of $x$ taken $n$ times, $x^0 = e$ and  $x^{-n}$ ($n\geq 1$) as the product of $x^{-1}$ taken $n$ times. Then the following hold:
	\begin{enumerate}[(i)]
		\item $(x^{-1})^m = (x^m)^{-1}$ for all integer $m$;
		\item $x^mx^n = x^{m+n}$ and $(x^m)n = x^{mn}$ for all integers $m,n$.
	\end{enumerate}
\end{enumerate}