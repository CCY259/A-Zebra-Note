

\setcounter{section}{-1}

\part{Miscellaneous}

\section{Zabalang}
This is a collection of all results used in the previous section, but we have not yet learned them in any undergraduate studies.

\subsection{Freeeee}
\begin{definition}
	Let $(\mathcal{C}, U)$ be a concrete category, where $U:\mathcal{C}\rightarrow \mathbf{Set}$ is a forgetful functor. If $\iota: X \rightarrow UF$ a function from a set $X$ to the underlying set of an object $F$ in $\mathcal{C}$, then $F$ is called \textbf{free on $X$} if  for every $A \in \operatorname{obj}\mathcal{C}$ and map $f: X \rightarrow UA$, there exists a unique morphism $f':F\rightarrow A$ such that $Uf' \circ \iota = f$.
\end{definition}
\begin{remark}
	In fact, we call $\iota$ a \textbf{universal arrow} from $X$ to $U$. Note that $\iota$ is often omitted when introducing a free object, but it must exist. In this case, we shall say the pair $(F,\iota)$ satisfies the \textbf{universal property}. We present the definition of this general concept below.
\end{remark}
\begin{definition}
	If $S : \mathcal{D} \to \mathcal{C}$ is a functor and $c$ an object of $\mathcal{C}$, a \textbf{universal arrow} from $c$ to $S$ is a pair $( r, u )$ consisting of an object $r$ of $\mathcal{D}$ and an arrow $u : c \to Sr$ of $\mathcal{C}$, such that to every pair $( d, f )$ with $d$ an object of $\mathcal{C}$ and $f \colon c \to Sd$ an arrow of $\mathcal{C}$, there is a unique arrow $f' \colon r \to d$ of $\mathcal{D}$ with $S f' \circ u = f$. In other words, every arrow $f$ to $S$ factors uniquely through the universal arrow $u$, as in the commutative diagram
	\[\begin{tikzcd}
		c && Sr && r \\
		\\
		&& Sd && d
		\arrow["u", from=1-1, to=1-3]
		\arrow["f"', from=1-1, to=3-3]
		\arrow["{Sf'}", dashed, from=1-3, to=3-3]
		\arrow["{f'}", dashed, from=1-5, to=3-5]
	\end{tikzcd}\]
\end{definition}
\begin{proposition} \label{prop-free}
	Let $\mathbb{C}$ be a concrete category. If $F$ and $F'$ are free objects on sets $X$ and $X'$, respectively with $|X| = |X'|$, then $F\cong F'$.
\end{proposition}


\subsection{ZORN'S LEMMA!!!!}
\begin{definition}
	Let $(A,\leq)$ be a partially ordered set. An element $a \in A$ is \textbf{maximal} in $A$ if for every $c \in A$ which is comparable to $a$, $c \leq a$; in other words, for all $c \in A$, $a \leq c \Rightarrow a = c$
\end{definition}
\begin{definition}
	An \textbf{upper bound} of a nonempty subset $B$ of $A$ is an element $d \in A$ such that $b \leq d$ for every $b \in B$. A nonempty subset $B$ of $A$ that is linearly ordered by $\leq$ is called a \textbf{chain} in $A$.
\end{definition}
\begin{remark}
	Note that if $a$ is maximal, it need not be the case that $c \leq a$ for all $c \in A$ (there may exist $c \in A$ that are not comparable to $a$). Furthermore, a given set may have many maximal elements or none at all (for example, $\mathbb{Z}$ with its usual ordering).
\end{remark}
  
\begin{lemma}[Zorn's Lemma] \label{lemma-Zorn}
	If $A$ is a nonempty partially ordered set such that every chain in $A$ has an upper bound in $A$, then $A$ contains a maximal element.
\end{lemma}

\begin{comment}
	\begin{theorem} \label{thm-submodule-of-free-module-is-free}
		If $M$ is a free module over a principal ideal domain $R$, then any submodule $N$ of $M$ is free.
	\end{theorem}
\end{comment}


\subsection{BILA NAK COUNTABLE OR UNCOUNTABLE, PENANG LIH}
\subsubsection{How We Actually Define the ``Size" of a Set}
\begin{definition}
	Let $S$ and $T$ be sets.
	\begin{enumerate}[(i)]
		\item Two sets $S$ and $T$ have the same \textbf{cardinality}, written
		\[ |S| = |T| \]
		if there is a bijective function (a \textbf{one-to-one correspondence}) between the sets.
		\item If $S$ is in one-to-one correspondence with a \textbf{subset} of $T$, then we write $$|S| \leq |T|.$$
		\item If $S$ is in one-to-one correspondence with a \textbf{proper subset} of $T$ but \textbf{not all} of $T$, then we write $$|S| < |T|.$$The second condition is necessary, since, for instance, $\mathbb{N}$ is in one-to-one correspondence with a proper subset of $\mathbb{Z}$ and yet $\mathbb{N}$ is also in one-to-one correspondence with $\mathbb{Z}$ itself. Hence $|\mathbb{N}| = |\mathbb{Z}|$.
	\end{enumerate}
\end{definition}
\begin{definition}
	\begin{enumerate}[(i)]
		\item A set is \textbf{finite} if it can be put in one-to-one correspondence with a set of the form $\mathbb{Z}_n = \{0, 1, \ldots, n-1\}$, for some nonnegative integer $n$. A set that is not finite is \textbf{infinite}. The \textbf{cardinal number} (or \textbf{cardinality}) of a finite set is just the number of elements in the set.
		\item The \textbf{cardinal number} of the set $\mathbb{N}$ of natural numbers is $\aleph_0$. 
		\item Any set with cardinality $\aleph_0$ is called a \textbf{countably infinite} set and any finite or countably infinite set is called a \textbf{countable} set. An infinite set that is not countable is said to be \textbf{uncountable}.
	\end{enumerate}
\end{definition}

\subsubsection{Tools to ``Compare the Size"}
\begin{theorem}[Schröder--Bernstein theorem] \label{thm-cardinal-Schroder-Bernstein}
	For any sets $S$ and $T$,
	\[ |S| \leq |T| \text{ and } |T| \leq |S| \Rightarrow |S| = |T|. \]
\end{theorem}
\begin{theorem}[Cantor's Theorem]
	If $\mathcal{P}(S)$ denotes the power set of $S$, then
	\[ |S| < |\mathcal{P}(S)|. \]
\end{theorem}
\begin{theorem} \label{thm-cardinal-finite-subsets}
	If $\mathcal{P}_0(S)$ denotes the set of all finite subsets of $S$ and if $S$ is an infinite set, then
	\[ |S| = |\mathcal{P}_0(S)|. \]
\end{theorem}
\subsubsection{Cardinal Arithmetic Is Strange but Not Weirdo}
\begin{definition}
	Let $\kappa$ and $\lambda$ denote cardinal numbers. Let $S$ and $T$ be disjoint sets for which $|S| = \kappa$ and $|T| = \lambda$. 
	\begin{enumerate}[(i)]
		\item The \textbf{sum} $\kappa + \lambda$ is the cardinal number of $S \cup T$.
		\item The \textbf{product} $\kappa\lambda$ is the cardinal number of $S \times T$.
		\item The \textbf{power} $\kappa^\lambda$ is the cardinal number of $S^T$, where $S^T$ is the set of all functions from $T$ to $S$.
	\end{enumerate}
\end{definition}
\begin{theorem}
	Let $\kappa$, $\lambda$ and $\mu$ be cardinal numbers. Then the following properties hold.
	\begin{enumerate}[(i)]
		\item (Associativity)
		\begin{align*}
			\kappa + (\lambda + \mu) &= (\kappa + \lambda) + \mu,
			\\
			\kappa(\lambda\mu) &= (\kappa\lambda)\mu.
		\end{align*}
		\item (Commutativity)
		\begin{align*}
			\kappa + \lambda &= \lambda + \kappa,
			\\
			 \kappa\lambda &= \lambda\kappa.
		\end{align*}
		\item (Distributivity)
		\[ \kappa(\lambda + \mu) = \kappa\lambda + \kappa\mu. \]
		\item (Properties of Exponents)
		\begin{align*}
			\kappa^{\lambda+\mu} &= \kappa^\lambda \kappa^\mu,
			\\
			(\kappa^\lambda)^\mu &= \kappa^{\lambda\mu},
			\\
			(\kappa\lambda)^\mu &= \kappa^\mu \lambda^\mu.
		\end{align*}
	\end{enumerate}
\end{theorem}

\begin{theorem} \label{thm-cardinal-sum-prod}
	Let $\kappa$ and $\lambda$ be cardinal numbers, at least one of which is infinite. Then
	\[ \kappa + \lambda = \kappa\lambda = \max\{\kappa, \lambda\}. \]
\end{theorem}








\begin{theorem} \label{thm-cardinal-union}
	Let $\{A_k \mid k \in K\}$ be a collection of sets, indexed by the set $K$, with $|K| = \kappa$. If $|A_k| \leq \lambda$ for all $k \in K$, then
	\[ \left|\, \bigcup_{k \in K} A_k \, \right| \leq \lambda\kappa. \]
\end{theorem}

\begin{theorem} \label{thm-cardinal-weird}
	Let $\kappa$ be any cardinal number.
	\begin{enumerate}[(i)]
		\item If $|S| = \kappa$, then $|\mathcal{P}(S)| = 2^\kappa$.
		\item $\kappa < 2^\kappa$.
		\item If $\kappa < \aleph_0$, then $\kappa$ is a natural number.
	\end{enumerate}
\end{theorem}
\begin{remark}
	There is a reason why we use $2^\kappa$.
	 In fact $|\mathbb{R}|$, is equal to $2^{\aleph_0}$.  This means we can use $\aleph_0$ for the size of countable sets and $2^{\aleph_0}$ for the size of uncountable sets.
\end{remark}

\begin{theorem} \label{thm-cardinal-aleph_0} The following results hold for $\aleph_0$.
\begin{enumerate}[(i)]
	\item Addition applied a countable number of times to the cardinal number $\aleph_0$ does not yield anything more than $\aleph_0$, i.e., $$\aleph_0 \cdot \aleph_0 = \aleph_0.$$
	\item Multiplication applied a \textbf{finite number} of times to the cardinal number $\aleph_0$ does not yield anything more than $\aleph_0$, i.e., for any positive integer $n\in\mathbb{N}$, $$\aleph_0^n = \aleph_0.$$ 
	The result is not true when we apply a countable number of times. More precisely,
	\begin{equation*}
		\aleph_0^{\aleph_0} = 2^{\aleph_0}.
	\end{equation*}
	\item Addition and multiplication applied a countable number of times to the cardinal number $2^{\aleph_0}$ does not yield more than $2^{\aleph_0}$, i.e., 
	\begin{align*}
		\aleph_0 \cdot 2^{\aleph_0} &= 2^{\aleph_0},
		\\
		(2^{\aleph_0})^{\aleph_0} &= 2^{\aleph_0}.
	\end{align*}
\end{enumerate}
\end{theorem}

\subsubsection{All the Facts That Make You a Weirdo}
The following results hold.
	\begin{enumerate}[(1)]
		\item $ |\mathbb{N}| = |\mathbb{Z}| = |\mathbb{Q}| = \aleph_0 $.
		\item $|\mathbb{R}| = 2^{\aleph_0}$.
		\item We can ``translate" everything from theorems above to some statements. The following are some examples.
		\begin{enumerate}[(i)]
			\item The power set of a countable set is uncountable.
			\item The power set of an uncountable set is uncountable.
			\item The set of all finite subsets of a uncountable set is uncountable.
			\item A countable union of countable sets is countable.
			\item The set of functions from a countable set to a countable set is uncountable.
			\item The set of functions from a countable set to a uncountable set is uncountable.
		\end{enumerate} 
	\end{enumerate}
	
\subsection{Ordinal was not invented by ordinary people}
It wasn't ordinary people, and it wasn't OVALTINE.
\subsubsection{How we actually ``order" elements}
\begin{definition}
	A binary relation $<$ on a set $P$ is a \textbf{partial ordering} of $P$ if:
	\begin{enumerate}
		\item[(i)] $p \not< p$ for any $p \in P$;
		\item[(ii)] if $p < q$ and $q < r$, then $p < r$.
	\end{enumerate}
	$(P,<)$ is called a \textbf{partially ordered set}. A partial ordering $<$ of $P$ is a \textbf{linear ordering} if moreover
	\begin{enumerate}
		\item[(iii)] $p < q$ or $p = q$ or $q < p$ for all $p, q \in P$.
	\end{enumerate}
\end{definition}
\subsubsection{Elements are always ordered nicely, not nicely, whatever}
\begin{definition}
	A linear ordering $<$ of a set $P$ is a \textbf{well-ordering} if every nonempty subset of $P$ has a least element.
\end{definition}
\begin{theorem}[Well-Ordering Theorem] \label{thm-well-ordering}
	Every set can be
	well-ordered.
\end{theorem}

\subsubsection{One, two, three, $\dots$ OMEGA!}
\begin{definition}
	A set $T$ is \textbf{transitive} if every element of $T$ is a subset of $T$.
\end{definition}
\begin{definition}
	A set is an \textbf{ordinal number} (an \textbf{ordinal}) if it is transitive and well-ordered by $\in$.
\end{definition}
We shall denote ordinals by lowercase Greek letters $\alpha, \beta, \gamma, \dots$. The class of all ordinals is denoted by $\operatorname{Ord}$.
\begin{proposition}
	Define
	\[ \alpha < \beta \quad \text{if and only if} \quad \alpha \in \beta. \]
	Then we have the following results.
	\begin{enumerate}[(i)]
		\item $<$ is a linear ordering of the class $\operatorname{Ord}$.
		\item For each $\alpha$, $\alpha = \{ \beta \,|\, \beta < \alpha \}$.
		\item If $C$ is a nonempty class of ordinals, then $\bigcap C$ is an ordinal, $\bigcap C \in C$ and $\bigcap C = \inf C$.
		\item If $X$ is a nonempty set of ordinals, then $\bigcup X$ is an ordinal, and $\bigcup X = \sup X$.
		\item For every $\alpha$, $\alpha \cup \{ \alpha \}$ is an ordinal and $\alpha \cup \{ \alpha \} = \inf \{ \beta \,|\, \beta > \alpha \}$.
	\end{enumerate}
\end{proposition}
We thus define $\alpha + 1 := \alpha \cup \{ \alpha \}$ (the \textbf{successor} of $\alpha$) and denote $\alpha + (n+1) := (\alpha+n)+1$ for all $n\geq 1$. 
\begin{definition}
	Let $\alpha$ be an ordinal. If $\alpha = \beta + 1$ for some ordinal $\beta$, then $\alpha$ is a \textbf{successor ordinal}. If $\alpha$ is not a successor ordinal, then $\alpha = \sup\{\beta \,|\, \beta < \alpha \} = \bigcup \alpha$ is called a \textbf{limit ordinal}. We also consider $0$ a limit ordinal and define $\sup \emptyset = 0$.
\end{definition}

In this fashion, we can generate greater and greater ``numbers'': $\omega,\omega + 1,\omega + 2,\dots, \omega + n,\dots$ for all $n \in \mathbb{N}$. A number following all $\omega + n$ can again be conceived of as a set of all smaller numbers:
\[ \omega 2 := \omega + \omega = \{0, 1, 2, \ldots, \omega, \omega + 1, \omega + 2, \ldots \}. \]
Then we can still generate greater numbers:
\begin{align*}
	\omega 2 + 1 &= \omega + \omega + 1 = \{0, 1, 2, \ldots, \omega, \omega + 1, \omega + 2, \ldots, \omega + \omega\}, \\
	\omega  3 &= \omega + \omega + \omega \\
	&= \{0, 1, 2, \ldots, \omega, \omega + 1, \omega + 2, \ldots, \omega + \omega, \omega + \omega + 1, \ldots \}, \\
	\omega^2 := \omega  \omega &= \{0, 1, 2, \ldots, \omega, \omega + 1, \ldots, \omega  2, \omega \cdot 2 + 1, \ldots, \omega  3, \ldots, \omega  4, \ldots \}.
\end{align*}
After that the whole thing starts over again: $\omega^2 + 1, \omega^2 + 2, \ldots, \omega^2 + \omega, \omega^2 + \omega + 1, \omega^2 + \omega + 2, \ldots, \omega^2 + \omega 2, \omega^2 + \omega 2 + 1, \ldots, \omega^2 + \omega 3, \ldots, \omega^2 + \omega 4, \ldots, \omega^2 2, \ldots, \omega^2 3, \ldots, \omega^3, \ldots,\break \omega^4, \ldots, \omega^\omega, \ldots, \omega^{\omega^\omega}, \ldots, \omega^{\omega^{(\omega^\omega)}}, \ldots$. The next one after all that is $\varepsilon_0$. Then come $\varepsilon_0 + 1, \varepsilon_0 + 2, \ldots, \varepsilon_0 + \omega, \ldots, \varepsilon_0 + \omega 2, \ldots, \varepsilon_0 + \omega^2, \ldots, \varepsilon_0 + \omega^\omega, \ldots, \varepsilon_0 2, \ldots, \varepsilon_0 \omega, \ldots, \varepsilon_0 \omega^\omega,\break \ldots, \varepsilon_0^2, \ldots $.

\subsubsection{Each well ordered set has a unique order type}


\begin{definition}
	If $(P,<)$ and $(Q,<)$ are partially ordered sets and $f: P \rightarrow Q$, then $f$ is \textbf{order-preserving} if $x < y$ implies $f(x) < f(y)$. 
\end{definition}

\begin{definition}
	A bijective function $f:P\to Q$ is an \textbf{isomorphism} of $P$ and $Q$ if both $f$ and $f^{-1}$ are order-preserving. In other words, we say that $(P,<)$ is \textbf{isomorphic} to $(Q,<)$.
\end{definition}

\begin{theorem}[Counting Theorem] \label{thm-counting}
	Every well-ordered set is isomorphic to a unique ordinal number.
\end{theorem}

\subsubsection{Playing Dominoes at The Edge of Universe}
\begin{theorem}[Transfinite Induction] \label{thm-trans-induction}
	Let $P(x)$ be a property. Assume that
	\begin{enumerate}[(i)]
		\item $P(0)$ holds.
		\item $P(\alpha)$ implies $P(\alpha + 1)$ for all ordinals $\alpha$.
		\item For all limit ordinals $\alpha \ne 0$, if $P(\beta)$ holds for all $\beta < \alpha$, then $P(\alpha)$ holds.
	\end{enumerate}
	Then $P(\alpha)$ holds for all ordinals $\alpha$.
\end{theorem}



\paragraph{Main References.} \cite{Roman2008,Halmos1974,Jech2003,Osborne2000,Hungerford1974}




